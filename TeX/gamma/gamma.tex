% file for pdflatex.exe

% gamma.tex v7.1 (c) | Copyright 2022-2023 Daniel E. Janusch

% this file is licensed by https://raw.githubusercontent.com/drizzt536/files/main/LICENSE
% and must be copied IN ITS ENTIRETY under penalty of law.

\documentclass[12pt]{article}
\usepackage{amssymb,latexsym,amsmath}
\usepackage[footskip=1in]{geometry}
\addtolength{\topmargin}{-0.5in}

\begin{document}


% \title{Trigonometric Functions in Terms of Gamma} % old title
\title{The Gamma Function, Its Inverse, and Its Relationship With Trigonometry}
\author{Daniel E. Janusch}
\date{Monday, May 29, 2023, 7:33pm MST}
\maketitle

\begin{section}{Definitions}

	\begin{equation}
		0\in\mathbb N~~(1\text a)\hspace{2em}\mathbb N_k:=
		\left\{x+k:x\in\mathbb N\right\}\forall\,k~~(1\text b)
	\end{equation} % equation 1

	\vspace{-0.065em}\begin{equation}
		\tau:=2\pi
	\end{equation} % equation 2

	\vspace{-0.065em}\begin{equation}
		\text{sgn}(x):=\begin{cases}
			1,x>0\\
			0,x=0\\
			-1,x<0
		\end{cases}\hspace{-1em}=\dfrac x{|x|}~\forall x
	\end{equation} % equation 3

	\vspace{-0.065em}\begin{equation}
		a\,\text{mod}_n\,b:=a\,\text{mod}\,b+n
	\end{equation} % equation 4

	\vspace{-0.065em}\begin{equation}
		\exp(x):=\lim_{n\to\infty}\sum_{k=0}^n\dfrac{x^k}{k!}=e^x~\forall x
	\end{equation} % equation 5

	\vspace{-0.065em}\begin{equation}
		\Gamma(x):=\int_0^\infty\dfrac{t^{x-1}}{\exp t}\text dt=(x-1)!~\forall x>0
	\end{equation} % equation 6

	\vspace{-0.065em}\begin{equation}
		\zeta(x):=\sum_{n=1}^\infty\dfrac1{n^x}\forall x>1
	\end{equation} % equation 7

	\vspace{-0.065em}\begin{equation}
		\eta(x):=\Gamma^{-1}(x)~~(8\text a)\hspace{2em}\Gamma(\,\eta(x)\,)=x~~(8\text b)
	\end{equation} % equation 8

	\noindent(8a) is the compositional inverse rather than the fractional inverse. The set of all
	natural numbers, $\mathbb N$, is defined in part by (1a) and (1b). (7) describes the Riemann
	Zeta Function. $x$ is assumed to be a real number everywhere for simplicity.
\end{section}

\pagebreak\begin{section}{Riemann Zeta Function}
	claim: $\zeta(x)=\dfrac{\zeta(1-x)\tau^x\sec\left(\dfrac{\pi x}2\right)}{2\,\Gamma(x)}$\\
	This proof is important for section 3.\\
	Proof:\\
	know: \zeta(x)=2^x\pi^{x-1}\sin\left(\dfrac{\pi x}2\right)\Gamma(1-x)\zeta(1-x)~~~(\text{reference }1)\\

	\begin{align*}
		\zeta(x) & =2^x\pi^{x-1}\sin\left(\dfrac{\pi x}2\right)\Gamma\left(1-x\right)
		\zeta\left(1-x\right)\\
		\zeta(x) & =2\cdot2^{x-1}\pi^{x-1}\sin\left(\dfrac{\pi x}2\right)\Gamma(1-x)\zeta(1-x)\\
		\zeta(x) & =2\cdot\tau^{x-1}\sin\left(\dfrac{\pi x}2\right)\Gamma(1-x)\zeta(1-x)\\
		\zeta(x) & =2\cdot\tau^{x-1}\cos\left(\dfrac\pi2(1-x)\right)\Gamma(1-x)\zeta(1-x)\\
		\zeta(x) & =\dfrac2{\tau^{1-x}}\cos\left(\dfrac\pi2(1-x)\right)\Gamma(1-x)\zeta(1-x)\\
		\zeta(1-x) & =\dfrac2{\tau^x}\cos\left(\dfrac{\pi x}2\right)\Gamma(x)\zeta(x)\\
		\zeta(x) & =\dfrac{\zeta(1-x)\tau^x}{2\cos\left(\dfrac{\pi x}2\right)\Gamma(x)}\\
	\end{align*}
	\begin{equation}
		\zeta(x) & =\dfrac{\zeta(1-x)\tau^x\sec\left(\dfrac{\pi x}2\right)}{2\,\Gamma(x)}
	\end{equation} % equation 9

	\noindent\blacksquare
\end{section}

\pagebreak\begin{section}{Cosecant}
	claim: $\csc x$ can be written in terms of $\Gamma(x)$

	\noindent Proof:

	\noindent know: \zeta(x)=2^x\pi^{x-1}\sin\left(\dfrac{\pi x}2\right)\Gamma(1-x)
	\zeta(1-x)~~~(\text{reference }1)\\
	
	\noindent know: \zeta(x)=\dfrac{\zeta(1-x)\tau^x\sec\left(\dfrac{\pi x}2\right)}{2\,\Gamma(x)
	}~~~(\text{equation }9)\\

	\begin{align*}
		2^x\pi^{x-1}\sin\left(\dfrac{\pi x}2\right)\Gamma(1-x)\zeta(1-x) & =\dfrac{\zeta(1-x)
		\tau^x\sec\left(\dfrac{\pi x}2\right)}{2\,\Gamma(x)}\\
		2^x\pi^{x-1}\sin\left(\dfrac{\pi x}2\right)\cos\left(\dfrac{\pi x}2\right)\Gamma(1-x) &
		=\dfrac{\tau^x}{2\,\Gamma(x)}\\
		4\cdot\tau^{x-1}\sin\left(\dfrac{\pi x}2\right)\cos\left(\dfrac{\pi x}2\right)\Gamma(1-x)
		& =\dfrac{\tau^x}{\Gamma(x)}\\
		2\sin\left(\dfrac{\pi x}2\right)\cos\left(\dfrac{\pi x}2\right)\Gamma(1-x) & =
		\dfrac{\pi}{\Gamma(x)}\\
		\Gamma(1-x)\Gamma(x)\sin\pi x & =\pi\\
	\end{align*}
	\begin{align} % equations 10-12
		\Gamma(1-x)\Gamma(x)=\pi\csc\pi x\\
		\csc x=\dfrac1\pi\Gamma\left(\dfrac x\pi\right)\Gamma\left(1-\dfrac x\pi\right)\\
		\sin\pi x\ne0\Longrightarrow\zeta(1-x)\ne0
	\end{align}

	\noindent\centerline{Therefore, due to (12), the domains also match.}

	\noindent\blacksquare
\end{section}

\pagebreak\begin{section}{Trig Functions In Terms of Cosecant}
	
	\noindent claim: The rest of the trig functions can be written in terms of cosecant

	\noindent Proof:

	\begin{align} % equations 13-23
		\sin x & =\dfrac1{\csc x}\\
		\cos x & =\dfrac1{\csc\left(\dfrac\pi2-x\right)}\\
		\tan x & =\dfrac{\csc\left(\dfrac\pi2-x\right)}{\csc x}\\
		\sec x & =\csc\left(\dfrac\pi2-x\right)\\
		\cot x & =\dfrac{\csc x}{\csc\left(\dfrac\pi2-x\right)}\\
		\sinh x & =-\dfrac i{\csc x}\\
		\cosh x & =\dfrac1{\csc\left(\dfrac\pi2-ix\right)}\\
		\tanh x & =-\dfrac{i\csc\left(\dfrac\pi2-ix\right)}{\csc ix}\\
		\text{csch}\,x & =i\csc ix\\
		\text{sech}\,x & =\csc\left(\dfrac\pi2-ix\right)\\
		\coth x & =\dfrac{i\csc ix}{\csc\left(\dfrac\pi2-ix\right)}
	\end{align}

	\centerline{$\csc x=\csc x$ by the reflexive property.}

	\noindent\blacksquare
\end{section}

\pagebreak\begin{section}{Trig Functions In Terms of Gamma}

	\noindent These equations are just the previous ones with Gamma plugged in for csc.

	\begin{align} % equations 24-35
		\sin x & =\dfrac\pi{\Gamma\left(\dfrac x\pi\right)\Gamma\left(1-\dfrac x\pi\right)}\\
		\cos x & =\dfrac\pi{\Gamma\left(\dfrac12-\dfrac x\pi\right)\Gamma\left(\dfrac12+\dfrac x
		\pi\right)}\\
		\tan x & =\dfrac{\Gamma\left(\dfrac12-\dfrac x\pi\right)\Gamma\left(\dfrac12+\dfrac x\pi
		\right)}{\Gamma\left(\dfrac x\pi\right)\Gamma\left(1-\dfrac x\pi\right)}\\
		\csc x & =\dfrac1\pi\Gamma\left(\dfrac x\pi\right)\Gamma\left(1-\dfrac x\pi\right)\\
		\sec x & =\dfrac1\pi\Gamma\left(\dfrac12-\dfrac x\pi\right)\Gamma\left(\dfrac12+\dfrac x
		\pi\right)\\
		\cot x & =\dfrac{\Gamma\left(\dfrac x\pi\right)\Gamma\left(1-\dfrac x\pi\right)}{\Gamma
		\left(\dfrac12-\dfrac x\pi\right)\Gamma\left(\dfrac12+\dfrac x\pi\right)}\\
		\sinh x & =-\pi i\big/\,\Gamma\left(\dfrac{ix}\pi\right)\Gamma\left(1-\dfrac{ix}\pi
		\right)\\
		\cosh x & =\pi\big/\,\Gamma\left(\dfrac12-\dfrac{ix}\pi\right)\Gamma\left(\dfrac12+
		\dfrac{ix}\pi\right)\\
		\tanh x & =-i\,\Gamma\left(\dfrac12-\dfrac{ix}\pi\right)\Gamma\left(\dfrac12+
		\dfrac{ix}\pi\right)\big/\,\Gamma\left(\dfrac{ix}\pi\right)\Gamma\left(1-
		\dfrac{ix}\pi\right)\\
		\text{csch}\,x & =\dfrac i\pi\Gamma\left(\dfrac{ix}\pi\right)\Gamma\left(1-
		\dfrac{ix}\pi\right)\\
		\text{sech}\,x & =\dfrac1\pi\Gamma\left(\dfrac12-\dfrac{ix}\pi\right)\Gamma\left(
		\dfrac12+\dfrac{ix}\pi\right)\\
		\coth x & =i\,\Gamma\left(\dfrac{ix}\pi\right)\Gamma\left(1-\dfrac{ix}\pi\right)\big/
		\,\Gamma\left(\dfrac12-\dfrac{ix}\pi\right)\Gamma\left(\dfrac12+\dfrac{ix}\pi\right)
	\end{align}

	\noindent Where ``$\big/$" means divide everything on the left by everything on the right.
\end{section}

\pagebreak\begin{section}{Inverse Gamma Function}\label{sec:inverse gamma}
	
	claim: $\eta(x)=\dfrac1\pi\csc^{-1}\dfrac x\pi\Gamma(1-\eta(x))$\\
	Proof:\\
	know: $\Gamma(x)\,\Gamma(1-x)=\pi\csc\pi x~~~$(equation 10)

	\begin{align} % equations 36-40
		\pi\csc\pi x & =\Gamma(x)\,\Gamma(1-x)\\
		\pi\csc\pi\eta(x) & =x\,\Gamma(1-\eta(x))\\
		\csc\pi\eta(x) & =\dfrac x\pi\Gamma(1-\eta(x))\\
		\pi\eta(x) & =\csc^{-1}\dfrac x\pi\Gamma(1-\eta(x))\\
		\eta(x) & =\dfrac1\pi\csc^{-1}\dfrac x\pi\Gamma(1-\eta(x))
	\end{align}

	\noindent\blacksquare

	\noindent There are 2 conditions to where this $\eta(x)$ is defined.\\
	$\eta(x)$ is \textbf{not} defined (over the real numbers) such that the following is true:\\
	\begin{equation}
		\Gamma\!\left(-\frac12\right)\le x\le\Gamma\!\left(\frac12\right)
	\end{equation}\\ % equation 41
	\begin{equation*}
		|\eta(x)|\le\frac12
	\end{equation*}\\

	$\eta(x)$ is guaranteed to be defined if Equation (41) is false. $\eta(x)$ is defined for
	all integers and in small fields around each integer. The area defined around each integer
	gets smaller as $x$ increases, though it is unclear why without further inspection. If
	$\eta(x)$ is being recursed and $\dfrac x\pi$ is changed to $\dfrac{x^n}\pi$ in the
	innermost function, for some $n\in\mathbb N_2$, either the fields get larger, or there is
	a bug in Desmos' graphing calculations. Reference 2 and Page 9 have much more information.\\
	Going a different way after (37) gives the following equation:\\
	\begin{equation}
		\eta(x)=1-\eta\left(\dfrac\pi x\csc\pi\eta(x)\right)\Longrightarrow\eta(x)=\eta(\pi
		\csc(\pi\csc\pi x)\sin\pi x)
	\end{equation}\\ % equation 42
	This equation seems to be less useful because it has large derivatives everywhere when
	absolute-valued, which gets more severe as it is recursed.
\end{section}

\pagebreak\begin{section}{Miscellaneous}
	(28) implies the following formula:
	\begin{equation}
		\Gamma\left(x+\dfrac12\right)=\dfrac{\pi\sec\pi x}{\Gamma\left(\dfrac12-x\right)}
	\end{equation}\\ % equation 43
	Replacing $x$ with $x+z$ and adjusting from gamma to factorial gives:\\
	\begin{equation}
		\left(x+\frac12+z\right)!=\left[2\left(z\,\text{mod}\,2\right)-1\right]
		\dfrac{\pi\sec\pi x}{\left(-\dfrac32-x-z\right)!}\forall z\in\mathbb Z
	\end{equation}\\ % equation 44
	This only simplifies things if $z$ is an integer because $\sec(x+\tau)=\sec x$,\\
	and $\sec(x+\pi)=-\sec x$. rewriting (10) from gamma to factorial gives:
	\begin{equation}
		(x-1)!(-x)!=\pi\csc\pi x
	\end{equation}\\ % equation 45
	\begin{equation}
		(-x)!=\dfrac{\pi x\csc\pi x}{x!}
	\end{equation}\\ % equation 46
	\begin{equation}
		\left(x+\dfrac1c\right)!=\dfrac{\pi\csc\dfrac{cx+1}c\pi}{\left(-\dfrac{c+1}c-x
		\right)!}\forall c\in\mathbb R
	\end{equation}\\ % equation 47
	\begin{equation}
		x!=\left(x\,\text{mod}_n\,1\right)!\left[\prod_{k\,=\,\text{sgn}(\left|n-x\right|+n-x)}^{
		\left|\left\lfloor x-n\right\rfloor\right|+\text{sgn}\left(n-x-\left|n-x\right|\right)}
		\left[x+k\,\text{sgn}\left(n-x\right)\right]\right]^{\text{sgn}\left(x-n\right)}
	\end{equation}\\ % equation 48
	(48) is useful where $x!$ can be non self-dependantly defined for $x\in[n,n+1)$.\\
	\begin{equation}
		\lim_{x\to\infty}\int_0^x\hspace{-0.5em}\int_0^x\exp\left[\dfrac12\ln\dfrac{4t}u-t-u
		\right]\text d^2tu=\pi
	\end{equation} % equation 49
	\pagebreak\\
	claim: $\sin(x)=-\sin(-x)$:\\
	Proof:\\
	\begin{equation}
		(45)~x\mapsto-x\Longrightarrow(-x)!=\dfrac{-\pi x\csc(-\pi x)}{x!}
	\end{equation}\\ % equation 50
	\centerline{$(-x)!=\dfrac{\pi x\csc\pi x}{x!}\hspace{1.6em}(46)$}\\
	\begin{equation}
		\dfrac{\pi x\csc\pi x}{x!}=\dfrac{-\pi x\csc(-\pi x)}{x!}\Longrightarrow
		\csc x=-\csc(-x)\Longrightarrow\sin(x)=-\sin(-x)
	\end{equation}\\ % equation 51
	\blacksquare\\

	\noindent claim: $\left(\dfrac12\right)!=\dfrac{\sqrt\pi}2$\\
	Proof:\\
	\begin{equation}
		x!=\dfrac{(x+1)!}{x+1}\Longrightarrow\left(-\dfrac12\right)!=2\left(\dfrac12\right)!
	\end{equation}\\ % equation 52
	\begin{equation}
		(-x)!=\dfrac{\pi x\csc\pi x}{x!}\Longrightarrow\left(-\dfrac12\right)!=
		\dfrac\pi{2\left(\dfrac12\right)!}
	\end{equation}\\ % equation 53
	The formula for (53) is just (46). (52) and (53) imply the following:\\
	\begin{equation}
		2\left(\dfrac12\right)!=\dfrac\pi{2\left(\dfrac12\right)!}\Longrightarrow
		4\left(\dfrac12\right)!^2=\pi\Longrightarrow\left(\dfrac12\right)!=\dfrac{\sqrt\pi}2
	\end{equation}\\ % equation 54
	Using this strategy only works if $(c_1-x)=(c_2+x)$ for some integers $c_1$ and $c_2$. This
	limits it to $(z+\frac12)$ for integers $z$. Also, you can only apply (53) an odd number of
	times or the factorials will cancel and leave a trigonometric equation. If you apply (53)
	twice, you get back to where you started because everything new cancels.\\
	\blacksquare\\

	\noindent\begin{equation}
		\left(\dfrac12+n\right)!=\dfrac{\sqrt\pi}2\prod_{k\,=\,\text{sgn}\left(n+\left|n\right|
		\right)}^{\left|n\right|\,-\,\text{sgn}\left(n+\left|n\right|\right)\,+\,\text{sgn}\,n}
		\left(\dfrac12+k\,\text{sgn}\,n\right)^{\text{sgn}\,n}\forall n\in\mathbb Z
	\end{equation} % equation 55
	\pagebreak\begin{equation}
		\Gamma\left(\dfrac1\pi\csc^{-1}\dfrac{\Gamma(11)}\pi\right)-\Gamma(11)\approx
		\!\int_0^1\ln\ln\frac1x~\text dx
	\end{equation}

	\begin{equation}
		\lim_{x\to\infty}\left[\Gamma(x)-\Gamma\!\left(\dfrac1\pi\csc^{-1}
		\dfrac{\Gamma(x)}\pi\right)\right]=\gamma
	\end{equation}

	\begin{equation}
		x\approx\Gamma\left(\dfrac1\pi\csc^{-1}\dfrac x\pi\right)
	\end{equation}

	\begin{equation}
		x\approx\Gamma\circ\eta_1(x)\ni\eta_0(x)=0
	\end{equation}
	(56) is a specific case of (57). (58) and (59) are the same equation with different syntax.
	Equation (60) clarifies what the $n^\text{th}$ $\eta(x)$ means. Equations (56) and (57) don't
	work on Desmos after around $x=11$ because it doesn't work well with large numbers since it is
	based in JavaScript.\\
	\begin{equation}
		\eta_n(x):=\dfrac1\pi\csc^{-1}\!\left[\dfrac x\pi\Gamma(1-\eta_{n-1}(x))\right]
	\end{equation}
	Where $\eta_0(x)$ can be any function. Section~\ref{sec:inverse gamma} uses $\eta_0(x):=x$,
	but $\eta_0(x)=-\left|x^n\right|$ or $\eta_0(x):=c$ for some constant $c$ both
	seem to work well, though the powers of $x$ working better; also being better for integer
	powers $n$.
	It is important to note that $\eta(x)$ only returns values in the range
	$\left[-\frac12,\frac12\right]$. This implies that while $\Gamma(\,\eta(x)\,)=x$ is
	always true, $\eta(\,\Gamma(x)\,)=x$ is not necessarily true, only being valid for small $x$.
	$\Gamma(\,\eta_n(x)\,)$ seems to be trying to approach
	$\displaystyle\left|1-\left(x-\frac12\right)\text{mod}\,2\right|-\frac12$, at least for
	negative $x$; a similar equation to $\displaystyle\cos^{-1}(\sin x)$. It also appears to be
	approaching approximately $\displaystyle\exp\left(-\left[\frac x\pi\right]^{\frac52}\right)$
	for positive $x$, $n>\sim9$, and pretty much any $\eta_0(x)=c$. The following table is for
	the approximate best constant values, (method: eyeballing the graph), and is split by
	$n\bmod4$. It is just for a rought approximation.\vspace{1em}\\
	\centerline{\begin{array}{| |c|c| |c|c| |c|c| |c|c| |}
		\hline
		\eta_n(x) & \text{best }\eta_0 &
		\eta_n(x) & \text{best }\eta_0 &
		\eta_n(x) & \text{best }\eta_0 &
		\eta_n(x) & \text{best }\eta_0 \\

		\hline 0  & 0      & 1  & 0.15   & 2  & 0.42   & 3  & 0.42   \\
		\hline 4  & 0.4    & 5  & 0.4    & 6  & 0.4    & 7  & 0.4    \\
		\hline 8  & 0.394  & 9  & 0.394  & 10 & 0.39   & 11 & 0.39   \\
		\hline 12 & 0.387  & 13 & 0.387  & 14 & 0.385  & 15 & 0.3855 \\
		\hline 16 & 0.3835 & 17 & 0.3835 & 18 & 0.382  & 19 & 0.382  \\
		\hline 20 & 0.3808 & 21 & 0.3805 & 22 & 0.3797 & 23 & 0.3797 \\
		\hline 24 & 0.3786 & 25 & 0.3786 & 26 & 0.3777 & 27 & 0.3777 \\
		\hline 28 & 0.3771 & 29 & 0.3772 & 30 & 0.3767 & 31 & 0.3766 \\
		\hline 32 & 0.3761 & 33 & 0.376  & 34 & 0.3754 & 35 & 0.3754 \\
		\hline
	\end{array}}\vspace{-1em}
\end{section}

\pagebreak\begin{section}{References}

	\noindent- \url{https://en.wikipedia.org/wiki/Riemann\_zeta\_function\#Riemann's\_functional\_equation}\\
	\indent\text{Link for Equation 9}\\

	\noindent- \url{https://www.desmos.com/calculator/qnnylllsvt}\\
	\indent\text{Extra information and graph for Section 6}\\

	\noindent- \url{https://www.github.com/drizzt536/files/tree/main/TeX/gamma}\\
	\indent\text{The files for the most recent version of this pdf and the \LaTeX $\,$code}
	\\
	\\
	\\
	\\
	\\
	\\
	This document is licensed under https://github.com/drizzt536/files/blob/main/LICENSE
	and must be copied IN ITS ENTIRETY under penalty of law.
\end{section}


\end{document}
