% file for pdflatex.exe

% gamma.tex v8.0 (c) | Copyright 2022-2024 Daniel E. Janusch

% this file is licensed by https://raw.githubusercontent.com/drizzt536/files/main/LICENSE
% and must be copied IN ITS ENTIRETY under penalty of law.

\documentclass[12pt]{article}

\usepackage{amssymb, latexsym, amsmath, xcolor, hyperref, xurl}

\usepackage[footskip=1in]{geometry}
\addtolength \topmargin {-0.5in}

\providecommand \dt {\mathrm dt}
\providecommand \dx {\mathrm dx}
\providecommand \dstyle \displaystyle
\providecommand \pgrp [1] {\left( #1 \right)}   % paren group
\providecommand \bgrp [1] {\left[ #1 \right]}   % bracket group
\providecommand \cgrp [1] {\left\{ #1 \right\}} % curly bracket group
\providecommand \setN {\mathbb N}
\providecommand \setZ {\mathbb Z}
\providecommand \setR {\mathbb R}
\providecommand \floor [1]{\left\lfloor #1 \right\rfloor}
\providecommand \abs [1] {\left| #1\right|}
\providecommand \sgn {\mathrm{sgn}\,}
\providecommand \sgnf [1]{\mathrm{sgn}\!\pgrp{#1}}
\DeclareMathOperator \csch {csch}
\DeclareMathOperator \sech {sech}

\providecommand \hpx  [1] {\hspace{#1px}}
\providecommand \vpx  [1] {\vspace{#1px}}
\providecommand \nhpx [1] {\hspace{-#1px}}
\providecommand \nvpx [1] {\vspace{-#1px}}

\providecommand \darkMode 0
\ifnum \darkMode = 0 \else
	\pagecolor{black}
	\color[RGB]{170, 170, 170}
\fi

\begin{document}

\title{The Gamma Function, Its Inverse, and Its Relationship With Trigonometry}
\author{Daniel E. Janusch}
\date{Wednesday, December 11, 2024, 3:40pm MST}
\maketitle

\section{Definitions}

\begin{equation}
	0 \in \setN ~~ (1\text a) \hpx{32}
	\setN_k := \cgrp{x + k : x \in \setN} \forall \,k~~(1\text b)
\end{equation} % equation 1
\nvpx 1
\begin{equation}
	\tau := 2\pi
\end{equation} % equation 2
\nvpx 1
\begin{equation}
	\sgnf{x} := \begin{cases}
		1, x > 0 \\
		0, x = 0 \\
		-1, x < 0
	\end{cases}\nhpx{10} = \dfrac x{|x|} ~ \forall x
\end{equation} % equation 3
\nvpx 1
\begin{equation}
	a \,\text{mod}_n\, b := a \,\text{mod}\, b + n
\end{equation} % equation 4
\nvpx 1
\begin{equation}
	\exp(x) := \lim_{n\to\infty} \sum_{k=0}^n \dfrac{x^k}{k!} = e^x ~ \forall x
\end{equation} % equation 5
\nvpx 1
\begin{equation}
	\Gamma(x) := \int_0^\infty \dfrac{t^{x-1}}{\exp t} \dt = (x-1)! ~ \forall x > 0
\end{equation} % equation 6
\nvpx 1
\begin{equation}
	\zeta(x) := \sum_{n=1}^\infty \dfrac 1{n^x} \forall x > 1
\end{equation} % equation 7
\nvpx 1
\begin{equation}
	\eta(x) := \Gamma^{-1}(x)~~(8\text a) \hpx{28}
	\Gamma(\,\eta(x)\,) = x~~(8\text b)
\end{equation} % equation 8

\noindent (8a) is the compositional inverse rather than the fractional inverse.
The set of all natural numbers, $\setN$, is defined in part by (1a) and
(1b). (7) describes the Riemann Zeta Function. $x$ is assumed to be a real
number everywhere for simplicity.

\pagebreak
\section{Riemann Zeta Function}

claim: $\zeta(x) = \dfrac{\zeta(1 - x)\tau^x\sec \dfrac{\pi x}2}{2\,\Gamma(x)}$\\
This proof is important for section 3.\\
Proof:\\
know: $\zeta(x) = 2^x\pi^{x-1}\sin\!\pgrp{\dfrac{\pi x}2}\Gamma(1 - x)\zeta(1 - x)$
~~(reference 1)\\

\begin{align*}
	\zeta(x) & = 2^x \pi^{x-1} \sin\!\pgrp{\dfrac{\pi x}2} \Gamma(1 - x) \zeta(1 - x)\\
	\zeta(x) & = 2 \cdot 2^{x-1} \pi^{x-1} \sin\!\pgrp{\dfrac{\pi x}2}
		\Gamma(1 - x) \zeta(1 - x)\\
	\zeta(x) & = 2 \cdot \tau^{x-1} \sin\!\pgrp{\dfrac \pi 2 x}
		\Gamma(1 - x) \zeta(1 - x)\\
	\zeta(x) & = 2 \cdot \tau^{x-1} \cos\!\pgrp{\dfrac \pi 2 (1 - x)}
		\Gamma(1 - x) \zeta(1 - x)\\
	\zeta(x) & = \dfrac 2{\tau^{1-x}} \cos\!\pgrp{\dfrac \pi 2 (1 - x)}
		\Gamma(1 - x) \zeta(1 - x)\\
	\zeta(1 - x) & = \dfrac 2{\tau^x} \cos\!\pgrp{\dfrac \pi 2 x} \Gamma(x) \zeta(x)\\
	\zeta(x) & = \dfrac{\zeta(1 - x) \tau^x}{2 \cos\!\pgrp{\dfrac{\pi x}2} \Gamma(x)}\\
\end{align*}
\begin{equation}
	\zeta(x) = \dfrac{\zeta(1 - x) \tau^x \sec \dfrac{\pi x}2}{2\,\Gamma(x)}
\end{equation} % equation 9

\noindent $\blacksquare$

\pagebreak
\section{Cosecant}
claim: $\csc x$ can be written in terms of $\Gamma(x)$\\
Proof:\\
know: $\zeta(x) = 2^x \pi^{x-1} \sin\!\pgrp{\dfrac{\pi x}2} \Gamma(1 - x)
	\zeta(1 - x)$~~~(\text{reference }1)\\
know: $\zeta(x) = \dfrac{\zeta(1 - x) \tau^x \sec \dfrac{\pi x}2}{2\,\Gamma(x)}$
	~~(equation 9)\\

\begin{align*}
	2^x \pi^{x-1} \sin\!\pgrp{\dfrac{\pi x}2} \Gamma(1 - x) \zeta(1 - x) & =
	\dfrac{\zeta(1 - x) \tau^x \sec \dfrac{\pi x}2}{2\,\Gamma(x)}\\
	2^x \pi^{x-1} \sin\!\pgrp{\dfrac{\pi x}2} \Gamma(1 - x) & =
	\dfrac{\tau^x \sec \dfrac{\pi x}2}{2\,\Gamma(x)}\\
	2 \cdot 2^x \pi^{x-1} \sin\!\pgrp{\dfrac{\pi x}2} \cos\!\pgrp{\dfrac{\pi x}2}
		\Gamma(1 - x) & = \dfrac{\tau^x}{\Gamma(x)}\\
	4 \cdot \tau^{x-1} \sin\!\pgrp{\dfrac{\pi x}2} \cos\!\pgrp{\dfrac{\pi x}2}
		\Gamma(1 - x) & = \dfrac{\tau^x}{\Gamma(x)}\\
	2 \cdot 2 \sin\!\pgrp{\dfrac{\pi x}2} \cos\!\pgrp{\dfrac{\pi x}2} \Gamma(1 - x) & =
		\dfrac \tau {\Gamma(x)}\\
	2 \sin\!\pgrp{\pi x} \Gamma(1 - x) & = \dfrac \tau {\Gamma(x)}\\
	\Gamma(1 - x) \Gamma(x) \sin \pi x & = \pi\\
\end{align*}
\nvpx{40}
\begin{align} % equations 10-12
	\Gamma(1 - x) \Gamma(x) = \pi \csc \pi x\\
	\csc x = \dfrac 1\pi \Gamma\!\pgrp{\dfrac x\pi} \Gamma\!\pgrp{1 - \dfrac x\pi}\\
	\sin \pi x \ne 0 \Longrightarrow \zeta(1 - x) \ne 0
\end{align}

\noindent \centerline{(12) shows the domains also match.}

\noindent $\blacksquare$

\pagebreak
\section{Trig Functions In Terms of Cosecant}

\noindent The rest of the trig functions can be written in terms of cosecant\\
\begin{align} % equations 14-24
	\sin x & = \dfrac 1{\csc x}\\
	\cos x & = \dfrac 1{\csc\!\pgrp{\dfrac \pi 2 - x}}\\
	\tan x & = \dfrac {\csc\!\pgrp{\dfrac \pi 2 - x}}{\csc x}\\
	\sec x & = \csc\!\pgrp{\dfrac \pi 2 - x}\\
	\cot x & = \dfrac{\csc x}{\csc\!\pgrp{\dfrac \pi 2 - x}}\\
	\sinh x & = -\dfrac i{\csc x}\\
	\cosh x & = \dfrac 1{\csc\!\pgrp{\dfrac \pi 2 - ix}}\\
	\tanh x & = -\dfrac{i\csc\!\pgrp{\dfrac \pi 2 - ix}}{\csc ix}\\
	\csch x & = i\csc ix\\
	\sech x & = \csc\!\pgrp{\dfrac \pi 2 - ix}\\
	\coth x & =\dfrac{i\csc ix}{\csc\!\pgrp{\dfrac \pi 2 - ix}}\\
	\csc x & = \csc x \text{ by the reflexive property.}
\end{align}

\pagebreak
\section{Trig Functions In Terms of Gamma}

\noindent These equations are just the previous ones with Gamma plugged in for csc.

\begin{align} % equations 25-36
	\sin x & = \dfrac \pi{\Gamma\!\pgrp{\dfrac x\pi} \Gamma\!\pgrp{1 - \dfrac x\pi}}\\
	\cos x & = \dfrac\pi{\Gamma\!\pgrp{\dfrac 12 - \dfrac x\pi} \Gamma\!
		\pgrp{\dfrac 12 + \dfrac x\pi}}\\
	\tan x & = \dfrac{\Gamma\!\pgrp{\dfrac 12 - \dfrac x\pi} \Gamma\!\pgrp{\dfrac 12
		+ \dfrac x\pi }}{\Gamma\!\pgrp{\dfrac x\pi} \Gamma\!\pgrp{1 - \dfrac x\pi}}\\
	\csc x & = \dfrac 1\pi \Gamma\!\pgrp{\dfrac x\pi} \Gamma\!\pgrp{1 - \dfrac x\pi}\\
	\sec x & = \dfrac 1\pi \Gamma\!\pgrp{\dfrac 12 - \dfrac x\pi} \Gamma\!
		\pgrp{\dfrac 12 + \dfrac x\pi}\\
	\cot x & = \dfrac{\Gamma\!\pgrp{\dfrac x\pi}\Gamma\!\pgrp{1 - \dfrac x\pi}}{\Gamma
		\!\pgrp{\dfrac 12 - \dfrac x\pi} \Gamma\!\pgrp{\dfrac 12 + \dfrac x\pi}}\\
	\sinh x & = -\dfrac{\pi i}{\Gamma\!\pgrp{\dfrac{ix}\pi} \Gamma\!
		\pgrp{1 - \dfrac{ix}\pi}}\\
	\cosh x & = \dfrac \pi{\Gamma\!\pgrp{\dfrac 12 - \dfrac{ix}\pi}\Gamma\!
		\pgrp{\dfrac 12 + \dfrac{ix}\pi}}\\
	\tanh x & = -i\,\Gamma\!\pgrp{\dfrac 12 - \dfrac{ix}\pi} \Gamma\!
		\pgrp{\dfrac 12 + \dfrac{ix}\pi}\!\bigg/\,\Gamma\!\pgrp{\dfrac{ix}\pi}
		\Gamma\!\pgrp{1 - \dfrac{ix}\pi}\\
	\csch x & = \dfrac i\pi \Gamma\!\pgrp{\dfrac{ix}\pi} \Gamma\!
		\pgrp{1 - \dfrac{ix}\pi}\\
	\sech x & = \dfrac 1\pi \Gamma\!\pgrp{\dfrac 12 - \dfrac{ix}\pi} \Gamma\!
		\pgrp{\dfrac 12 + \dfrac{ix}\pi}\\
	\coth x & = i\,\Gamma\!\pgrp{\dfrac{ix}\pi} \Gamma\!\pgrp{1 - \dfrac{ix}\pi}
		\!\bigg/\,\Gamma\!\pgrp{\dfrac 12 - \dfrac{ix}\pi} \Gamma\!\pgrp
		{\dfrac 12 + \dfrac{ix}\pi}
\end{align}

\pagebreak
\section{Inverse Gamma Function}\label{sec:inverse gamma}

claim: $\eta(x) = \dfrac 1\pi \csc^{-1}\dfrac x\pi \Gamma(1 - \eta(x))$\\
Proof:\\
know: $\Gamma(x)\,\Gamma(1 - x) = \pi\csc\pi x$~~~(equation 10)

\begin{align} % equations 37-41
	\pi \csc\pi x & = \Gamma(x)\,\Gamma(1 - x)\\
	\pi \csc \pi\eta(x) & = x\,\Gamma(1 - \eta(x))\\
	\csc \pi\eta(x) & = \dfrac x\pi \Gamma(1 - \eta(x))\\
	\pi \eta(x) & = \csc^{-1}\dfrac x\pi \Gamma(1 - \eta(x))\\
	\eta(x) & = \dfrac 1\pi \csc^{-1}\dfrac x\pi \Gamma(1 - \eta(x))
\end{align}

\noindent $\blacksquare$

\noindent There are 2 conditions to where this $\eta(x)$ is defined.\\
$\eta(x)$ is \emph{not} defined (over the real numbers) if the following is true:\\
\begin{equation}
	\Gamma\!\pgrp{-\dfrac 12} \le x \le \Gamma\!\pgrp{\dfrac 12}
\end{equation}\\ % equation 42
\begin{equation*}
	|\eta(x)| \le \dfrac 12
\end{equation*}\\

$\eta(x)$ is guaranteed to be defined if Equation (42) is false. $\eta(x)$ is
defined for all integers and in small fields around each integer. The area defined
around each integer gets smaller as $x$ increases, though it is unclear why without
further inspection. If $\eta(x)$ is being recursed and $\dfrac x\pi$ is changed to
$\dfrac{x^n}\pi$ in the innermost function, for some $n\in\setN_2$, either the
fields get larger, or there is a bug in Desmos' graphing calculations. Reference 2
and Page 9 have much more information.\\
Going a different way after (38) gives the following equation:\\
\begin{equation}
	\eta(x) = 1 - \eta\!\pgrp{\dfrac\pi x \csc\pi\eta(x)} \Longrightarrow
	\eta(x) = \eta(\pi \csc(\pi \csc\pi x) \sin\pi x)
\end{equation}\\ % equation 43
This equation seems to be less useful because it has large derivatives everywhere
when absolute-valued, which gets more severe as it is recursed.

\pagebreak
\section{Miscellaneous}
(29) implies the following formula:
\begin{equation}
	\Gamma\!\pgrp{x + \dfrac 12} = \dfrac{\pi \sec\pi x}{\Gamma\!\pgrp{\dfrac 12 - x}}
\end{equation}\\ % equation 44
Replacing $x$ with $x + z$ and adjusting from gamma to factorial gives:\\
\begin{equation}
	\pgrp{x + \dfrac 12 + z}! = \bgrp{2\pgrp{z \,\text{mod}\, 2} - 1}
	\dfrac{\pi \sec\pi x}{\pgrp{-\dfrac 32 - x - z}!} \forall z \in \setZ
\end{equation}\\ % equation 45
This only simplifies things if $z$ is an integer because $\sec(x + \tau) = \sec x$,\\
and $\sec(x + \pi) = -\sec x$. rewriting (10) from gamma to factorial gives:
\begin{equation}
	(x - 1)! (-x)! = \pi \csc \pi x
\end{equation}\\ % equation 46
\begin{equation}
	(-x)! = \dfrac{\pi x \csc \pi x}{x!}
\end{equation}\\ % equation 47
\begin{equation}
	\pgrp{x + \dfrac 1c}! = \dfrac{\pi \csc \pi\dfrac{cx + 1}c}
		{\pgrp{-\dfrac{c + 1}c - x}!} \forall c \in \setR
\end{equation}\\ % equation 48
\begin{equation}
	x! = \pgrp{x \,\text{mod}_n\, 1}! \bgrp{
		\prod_{
			k\,=\,\sgn\!(\abs{n - x} + n - x)
		}^{
			\abs{\floor{x - n}} - \sgn\!(\abs{x - n} + x - n)
		}
		\bgrp{x + k\,\sgnf{n - x}}
	}^{\sgn\!(x - n)}
\end{equation}\\ % equation 49
(49) is useful where $x!$ can be non self-dependantly defined for $x \in [n, n+1)$.\\
\begin{equation}
	\lim_{x\to\infty} \int_0^x\nhpx 6\int_0^x \exp\bgrp{\dfrac12\ln\dfrac{4t}u - t - u
	}\text d^2tu = \pi
\end{equation} % equation 50

\pagebreak
\noindent claim: $\sin(x) = -\sin(-x)$:\\
Proof:\\
\begin{equation}
	(47)~x \mapsto -x \Longrightarrow (-x)! = \dfrac{-\pi x\csc(-\pi x)}{x!}
\end{equation}\\ % equation 51
\centerline{$(-x)! = \dfrac{\pi x\csc\pi x}{x!}\hpx{22}(47)$}\\
\begin{equation}
	\dfrac{\pi x\csc\pi x}{x!} = \dfrac{-\pi x\csc(-\pi x)}{x!} \Longrightarrow
	\csc x = -\csc(-x) \Longrightarrow \sin(x) = -\sin(-x)
\end{equation}\\ % equation 52
$\blacksquare$\\

\noindent claim: $\pgrp{\dfrac 12}! = \dfrac{\sqrt \pi}2$\\
Proof:\\
\begin{equation}
	x! = \dfrac{(x + 1)!}{x + 1} \Longrightarrow \pgrp{-\dfrac 12}! =
	2\pgrp{\dfrac 12}!
\end{equation}\\ % equation 53
\begin{equation}
	(-x)! = \dfrac{\pi x\csc\pi x}{x!} \Longrightarrow \pgrp{-\dfrac 12}! =
	\dfrac \pi{2\pgrp{\dfrac 12}!}
\end{equation}\\ % equation 54
The formula for (54) is just (47). (53) and (54) imply the following:\\
\begin{equation}
	2\pgrp{\dfrac 12}! = \dfrac\pi{2\pgrp{\dfrac 12}!} \Longrightarrow
	\pgrp{\dfrac 12}!^2 = \dfrac \pi 4 \Longrightarrow
	\pgrp{\dfrac 12}! = \dfrac{\sqrt \pi}2
\end{equation}\\ % equation 55
Using this strategy only works if $(c_1 - x) = (c_2 + x)$ for some integers $c_1$
and $c_2$. This limits it to $(z + \frac 12)$ for integers $z$. Also, you can only
apply (47) an odd number of times or the factorials will cancel and leave a
trigonometric equation. If you apply (47) twice, you get back to where you started
because everything new cancels.\\
$\blacksquare$\\

\begin{equation}
	\pgrp{\dfrac 12 + n}! = \dfrac{\sqrt \pi}2 \prod_{k\,=\,\sgn\!(n + \abs n)
	}^{\abs n\,-\,\sgn\!(n + \abs n)\,+\,\sgn n
	} \bgrp{\dfrac 12 + k\,\sgn n}^{\sgn n} \forall n \in \setZ
\end{equation} % equation 56
\pagebreak
\begin{equation}
	\Gamma\!\pgrp{\dfrac 1\pi\csc^{-1}\dfrac{\Gamma(11)}\pi} - \Gamma(11)\approx
	\!\int_0^1\ln\ln\dfrac1x~\dx
\end{equation} % equation 57

\begin{equation}
	\lim_{x \to \infty}\bgrp{\Gamma(x) - \Gamma\!\pgrp{\dfrac 1\pi\csc^{-1}
	\dfrac{\Gamma(x)}\pi}} = \gamma
\end{equation} % equation 58

\begin{equation}
	x \approx \Gamma\!\pgrp{\dfrac 1\pi \csc^{-1}\dfrac x\pi}
\end{equation} % equation 59

\begin{equation}
	x \approx \Gamma \circ \eta_1(x) \ni \eta_0(x) = 0
\end{equation} % equation 60
(57) is a specific case of (58). (59) and (60) are the same equation with
different syntax. Equation (61) clarifies what the $n^\text{th}~\eta(x)$ means.
Equations (57) and (58) don't work on Desmos after around $x = 11$ because it
doesn't work well with large numbers since it is based in JavaScript. It does
work well in Mathematica, however.\\
\begin{equation}
	\eta_n(x) := \dfrac 1\pi \csc^{-1}\!\bgrp{\dfrac x\pi\Gamma(1 - \eta_{n-1}(x))}
\end{equation}
Where $\eta_0(x)$ can be any function. Section~\ref{sec:inverse gamma} uses
$\eta_0(x) := x$, but $\eta_0(x) = -\abs{x^n}$ or $\eta_0(x) := c$ for some constant
$c$ both seem to work well, though the powers of $x$ work better; also being better
for integer powers $n$. It is important to note that $\eta(x)$ only returns values
in the range $\bgrp{-\frac 12, \frac 12}$. This implies that while
$\Gamma(\,\eta(x)\,) = x$ is always true, $\eta(\,\Gamma(x)\,) = x$ is not
necessarily true, only being valid for small $x$. The following table is for the
approximate best constant values, (method: eyeballing the graph), and is split by
$n\bmod4$. It is just for a rought approximation.\vpx{14}\\
\centerline{$\begin{array}{| |c|c| |c|c| |c|c| |c|c| |}
	\hline
	\eta_n(x) & \text{best }\eta_0 &
	\eta_n(x) & \text{best }\eta_0 &
	\eta_n(x) & \text{best }\eta_0 &
	\eta_n(x) & \text{best }\eta_0 \\

	\hline 0  & 0      & 1  & 0.15   & 2  & 0.42   & 3  & 0.42   \\
	\hline 4  & 0.4    & 5  & 0.4    & 6  & 0.4    & 7  & 0.4    \\
	\hline 8  & 0.394  & 9  & 0.394  & 10 & 0.39   & 11 & 0.39   \\
	\hline 12 & 0.387  & 13 & 0.387  & 14 & 0.385  & 15 & 0.3855 \\
	\hline 16 & 0.3835 & 17 & 0.3835 & 18 & 0.382  & 19 & 0.382  \\
	\hline 20 & 0.3808 & 21 & 0.3805 & 22 & 0.3797 & 23 & 0.3797 \\
	\hline 24 & 0.3786 & 25 & 0.3786 & 26 & 0.3777 & 27 & 0.3777 \\
	\hline 28 & 0.3771 & 29 & 0.3772 & 30 & 0.3767 & 31 & 0.3766 \\
	\hline 32 & 0.3761 & 33 & 0.376  & 34 & 0.3754 & 35 & 0.3754 \\
	\hline
\end{array}$}\vspace{-1em}

\pagebreak
\section{References}

	\begin{enumerate}
		\item \url{https://en.wikipedia.org/wiki/Riemann_zeta_function#Riemann's_functional_equation}\\
			Link for Equation 9
		\item \url{https://www.desmos.com/calculator/qnnylllsvt}\\
			Extra information and graph for Section 6
		\item \url{https://www.github.com/drizzt536/files/tree/main/TeX/gamma}\\
			The files for the most recent version of this PDF and the \LaTeX $\,$code.
	\end{enumerate}
	\vfill
	\small
	\noindent This document is licensed under
	\url{https://raw.githubusercontent.com/drizzt536/files/main/LICENSE}
	and must be copied IN ITS ENTIRETY under penalty of law.
	\nvpx{20}
\end{document}
