% file for pdflatex.exe

\documentclass[12pt]{article}
\usepackage{amssymb,latexsym,amsmath}

\begin{document}

% \title{Trigonometric Functions in Terms of Gamma} % old title
\title{The Gamma Function, Its Inverse, And Its Relationship With Trigonometry}
\author{Daniel E. Janusch}
\date{November 16, 2022} % 8:56pm MST
\maketitle

\begin{section}{Definitions}

	\begin{equation}
		0\in\mathbb N
	\end{equation}

	\begin{equation}
		\mathbb N_k:=\left\{x+k:x\in\mathbb N\right\}
	\end{equation}

	\begin{equation}
		\tau:=2\pi
	\end{equation}

	\begin{equation}
		\Gamma(x):=\int_0^\infty\dfrac{t^{x-1}dt}{\exp\left(t\right)}=(x-1)!~\forall x>0
	\end{equation}

	\begin{equation}
		\zeta(x):=\sum_{n=0}^\infty\dfrac1{n^x}\forall x>1
	\end{equation}

	\begin{equation}
		\eta(x):=\Gamma^{-1}(x)
	\end{equation}

	\begin{equation}
		\eta(\,\Gamma(x)\,)=\Gamma(\,\eta(x)\,)=x
	\end{equation}

	(6) is the compositional inverse rather than the fractional inverse.

	The set of all natural numbers, $\mathbb N$, is defined by (1) and (2).

	(5) describes the Riemann Zeta Function.

	``$\ni$" means ``such that".
\end{section}

\pagebreak
\begin{section}{Riemann Zeta Function}
	claim: $\zeta(x)=\dfrac{\zeta(1-x)\tau^x\sec\left(\dfrac{\pi x}2\right)}{2\,\Gamma(x)}$\\
	This proof is important for part 3.\\
	Proof:\\
	know: \zeta(x)=2^x\pi^{x-1}\sin\left(\dfrac{\pi x}2\right)\Gamma(1-x)\zeta(1-x)\\

	\begin{equation}
		\zeta(x)=2^x\pi^{x-1}\sin\left(\dfrac{\pi x}2\right)\Gamma\left(1-x\right)\zeta\left(1-x\right)
	\end{equation}

	\begin{equation}
		\zeta(x)=2\cdot2^{x-1}\pi^{x-1}\sin\left(\dfrac{\pi x}2\right)\Gamma(1-x)\zeta(1-x)
	\end{equation}

	\begin{equation}
		\zeta(x)=2\cdot\tau^{x-1}\sin\left(\dfrac{\pi x}2\right)\Gamma(1-x)\zeta(1-x)
	\end{equation}

	\begin{equation}
		\zeta(x)=2\cdot\tau^{x-1}\cos\left(\dfrac\pi2(1-x)\right)\Gamma(1-x)\zeta(1-x)
	\end{equation}

	\begin{equation}
		\zeta(x)=\dfrac2{\tau^{1-x}}\cos\left(\dfrac\pi2(1-x)\right)\Gamma(1-x)\zeta(1-x)
	\end{equation}

	\begin{equation}
		\zeta(1-x)=\dfrac2{\tau^x}\cos\left(\dfrac{\pi x}2\right)\Gamma(x)\zeta(x)
	\end{equation}

	\begin{equation}
		\dfrac{\zeta(1-x)\tau^x}{2\cos\left(\dfrac{\pi x}2\right)\Gamma(x)}=\zeta(x)
	\end{equation}

	\begin{equation}
		\zeta(x)=\dfrac{\zeta(1-x)\tau^x\sec\left(\dfrac{\pi x}2\right)}{2\,\Gamma(x)}
	\end{equation}

	\noindent\blacksquare
\end{section}

\pagebreak
\begin{section}{Cosecant}
	claim: $\csc x$ can be written in terms of $\Gamma(x)$

	\noindent Proof:

	\noindent know: \zeta(x)=2^x\pi^{x-1}\sin\left(\dfrac{\pi x}2\right)\Gamma(1-x)\zeta(1-x)\\
	
	\noindent know: \zeta(x)=\dfrac{\zeta(1-x)\tau^x\sec\left(\dfrac{\pi x}2\right)}{2\,\Gamma(x)}\\

	\begin{equation}
		2^x\pi^{x-1}\sin\left(\dfrac{\pi x}2\right)\Gamma(1-x)\zeta(1-x)=\dfrac{\zeta(1-x)\tau^x\sec\left(\dfrac{\pi x}2\right)}{2\,\Gamma(x)}
	\end{equation}

	\begin{equation}
		2^x\pi^{x-1}\sin\left(\dfrac{\pi x}2\right)\cos\left(\dfrac{\pi x}2\right)\Gamma(1-x)=\dfrac{\tau^x}{2\,\Gamma(x)}
	\end{equation}

	\begin{equation}
		4\cdot\tau^{x-1}\sin\left(\dfrac{\pi x}2\right)\cos\left(\dfrac{\pi x}2\right)\Gamma(1-x)=\dfrac{\tau^x}{\Gamma(x)}
	\end{equation}

	\begin{equation}
		2\sin\left(\dfrac{\pi x}2\right)\cos\left(\dfrac{\pi x}2\right)\Gamma(1-x)=\dfrac{\pi}{\Gamma(x)}
	\end{equation}

	\begin{equation}
		\Gamma(1-x)\Gamma(x)\sin\pi x=\pi
	\end{equation}

	\begin{equation}
		\Gamma(1-x)\Gamma(x)=\pi\csc\pi x
	\end{equation}

	\begin{equation}
		\csc x=\dfrac1\pi\Gamma\left(\dfrac x\pi\right)\Gamma\left(1-\dfrac x\pi\right)
	\end{equation}

	\noindent\blacksquare
\end{section}

\pagebreak
\begin{section}{The Remaining Functions}
	
	\noindent claim: The rest of the trig functions can be written in terms of cosecant

	\noindent Proof:

	\begin{equation}\sin x=\dfrac1{\csc x}\end{equation}

	\begin{equation}\cos x=\dfrac1{\csc\left(\dfrac\pi2-x\right)}\end{equation}

	\begin{equation}\tan x=\dfrac{\csc\left(\dfrac\pi2-x\right)}{\csc x}\end{equation}

	% \begin{equation}\csc x=\csc x\end{equation}
	
	\begin{equation}\sec x=\csc\left(\dfrac\pi2-x\right)\end{equation}

	\begin{equation}\cot x=\dfrac{\csc x}{\csc\left(\dfrac\pi2-x\right)}\end{equation}


	\begin{equation}\sinh x=-\dfrac i{\csc x}\end{equation}

	\begin{equation}\cosh x=\dfrac1{\csc\left(\dfrac\pi2-ix\right)}\end{equation}

	\begin{equation}\tanh x=-\dfrac{i\csc\left(\dfrac\pi2-ix\right)}{\csc ix}\end{equation}

	\begin{equation}\text{csch}\,x=i\csc ix\end{equation}

	\begin{equation}\text{sech}\,x=\csc\left(\dfrac\pi2-ix\right)\end{equation}

	\begin{equation}\coth x=\dfrac{i\csc ix}{\csc\left(\dfrac\pi2-ix\right)}\end{equation}

	Therefore, all the trigonometric functions can be written in terms of $\Gamma(x)$.

	\noindent\blacksquare
\end{section}

\pagebreak
\begin{section}{Inverse Gamma Function}
	
	claim: $\eta(x)=\dfrac1\pi\csc^{-1}\dfrac x\pi\Gamma(1-\eta(x))$\\
	Proof:\\
	know: $\Gamma(x)\Gamma(1-x)=\pi\csc\pi x~~~$(equation 21)

	\begin{equation}
		\pi\csc\pi x=\Gamma(x)\Gamma(1-x)
	\end{equation}

	\begin{equation}
		\pi\csc\pi\eta(x)=x\Gamma(1-\eta(x))
	\end{equation}

	\begin{equation}
		\csc\pi\eta(x)=\dfrac x\pi\Gamma(1-\eta(x))
	\end{equation}

	\begin{equation}
		\pi\eta(x)=\csc^{-1}\dfrac x\pi\Gamma(1-\eta(x))
	\end{equation}

	\begin{equation}
		\eta(x)=\dfrac1\pi\csc^{-1}\dfrac x\pi\Gamma(1-\eta(x))
	\end{equation}

	\noindent\blacksquare

	There are 2 conditions to where $\eta(x)$ is defined.

	$\eta(x)$ is not defined over the real numbers such that the following is true:

	\begin{equation}
		|x\,\Gamma(1-\eta(x))|<\pi~~\text{or}~~|x|<\sim3.547373963...
	\end{equation}

	I assume that this value is transcendental or at least algebraic.

	$\eta(x)$ is guaranteed to be defined if (39) is false and $x$ is negative.

	$\eta(x)$ is defined for all integers and around the integers.

	The area around the integer that is defined gets smaller as $x$ increases,

	though I do not know why this is the case.

	If you are recursing $\eta(x)$ and you change the $\dfrac x\pi$ to $\dfrac{x^n}\pi$,

	for some $n\in\mathbb N_2$, it either gets defined for more values,

	or there is a bug in Desmos' graphing calculations.

	These extra values are just expanded ranges around the positive integers.

	It doesn't change the rule described in (39), however.

	See reference 2 for more information.

	going a different way after (35) gives the following equation:

	\begin{equation}
		\eta(x)=1-\eta\left(\dfrac\pi x\csc\pi\eta(x)\right)
	\end{equation}

\end{section}

\pagebreak
\begin{section}{References}
	- https://en.wikipedia.org/wiki/Riemann\_zeta\_function (equation 8)\\
	- https://www.desmos.com/calculator/ieu6ebnfn8 (for part 5, made by me)\\
	- https://www.github.com/drizzt536/files/tree/main/TeX/gamma/ (me)
\end{section}


\end{document}\
