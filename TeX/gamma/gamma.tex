% file for pdflatex.exe

\documentclass[12pt]{article}
\usepackage{amssymb,latexsym,amsmath}
\usepackage[footskip=1in]{geometry}
\addtolength{\topmargin}{-0.5in}

\begin{document}


% \title{Trigonometric Functions in Terms of Gamma} % old title
\title{The Gamma Function, Its Inverse, and Its Relationship with Trigonometry}
\author{Daniel E. Janusch}
\date{December 15, 2022} % ~6:00am MST
\maketitle

\begin{section}{Definitions}

	\begin{equation}
		0\in\mathbb N~~(1a)\hspace{2em}\tau:=2\pi~~(1b)
	\end{equation}

	\vspace{-0.065em}\begin{equation}
		\mathbb N_k:=\left\{x+k:x\in\mathbb N\right\}\forall\,k
	\end{equation}

	\vspace{-0.065em}\begin{equation}
		\text{sgn}(x):=\begin{cases}
			1,x>0\\
			0,x=0\\
			-1,x<0
		\end{cases}
	\end{equation}

	\vspace{-0.065em}\begin{equation}
		a\,\text{mod}_n\,b:=a\,\text{mod}\,b+n
	\end{equation}

	\vspace{-0.065em}\begin{equation}
		\exp(x):=\lim_{n\to\infty}\sum_{k=0}^n\dfrac{x^k}{k!}=e^x~\forall x
	\end{equation}

	\vspace{-0.065em}\begin{equation}
		\Gamma(x):=\int_0^\infty\dfrac{t^{x-1}}{\exp t}\text dt=(x-1)!~\forall x>0
	\end{equation}

	\vspace{-0.065em}\begin{equation}
		\zeta(x):=\sum_{n=1}^\infty\dfrac1{n^x}\forall x>1
	\end{equation}

	\vspace{-0.065em}\begin{equation}
		\eta(x):=\Gamma^{-1}(x)~~(8a)\hspace{2em}\eta(\,\Gamma(x)\,)=\Gamma(\,\eta(x)\,)=x~~(8b)
	\end{equation}

	\noindent(8a) is the compositional inverse rather than the fractional inverse. The set of all natural numbers, $\mathbb N$, is defined in part by (1a) and (2). (7) describes the Riemann Zeta Function. $x$ is assumed to be a real number for simplicity, but It probably doesn't matter.
\end{section}

\pagebreak\begin{section}{Riemann Zeta Function}
	claim: $\zeta(x)=\dfrac{\zeta(1-x)\tau^x\sec\left(\dfrac{\pi x}2\right)}{2\,\Gamma(x)}$\\
	This proof is important for part 3.\\
	Proof:\\
	know: \zeta(x)=2^x\pi^{x-1}\sin\left(\dfrac{\pi x}2\right)\Gamma(1-x)\zeta(1-x)~~~(\text{reference }1)\\

	\begin{equation}
		\zeta(x)=2^x\pi^{x-1}\sin\left(\dfrac{\pi x}2\right)\Gamma\left(1-x\right)\zeta\left(1-x\right)
	\end{equation}

	\begin{equation}
		\zeta(x)=2\cdot2^{x-1}\pi^{x-1}\sin\left(\dfrac{\pi x}2\right)\Gamma(1-x)\zeta(1-x)
	\end{equation}

	\begin{equation}
		\zeta(x)=2\cdot\tau^{x-1}\sin\left(\dfrac{\pi x}2\right)\Gamma(1-x)\zeta(1-x)
	\end{equation}

	\begin{equation}
		\zeta(x)=2\cdot\tau^{x-1}\cos\left(\dfrac\pi2(1-x)\right)\Gamma(1-x)\zeta(1-x)
	\end{equation}

	\begin{equation}
		\zeta(x)=\dfrac2{\tau^{1-x}}\cos\left(\dfrac\pi2(1-x)\right)\Gamma(1-x)\zeta(1-x)
	\end{equation}

	\begin{equation}
		\zeta(1-x)=\dfrac2{\tau^x}\cos\left(\dfrac{\pi x}2\right)\Gamma(x)\zeta(x)
	\end{equation}

	\begin{equation}
		\dfrac{\zeta(1-x)\tau^x}{2\cos\left(\dfrac{\pi x}2\right)\Gamma(x)}=\zeta(x)
	\end{equation}

	\begin{equation}
		\zeta(x)=\dfrac{\zeta(1-x)\tau^x\sec\left(\dfrac{\pi x}2\right)}{2\,\Gamma(x)}
	\end{equation}

	\noindent\blacksquare
\end{section}

\pagebreak\begin{section}{Cosecant}
	claim: $\csc x$ can be written in terms of $\Gamma(x)$

	\noindent Proof:

	\noindent know: \zeta(x)=2^x\pi^{x-1}\sin\left(\dfrac{\pi x}2\right)\Gamma(1-x)\zeta(1-x)~~~(\text{reference }1)\\
	
	\noindent know: \zeta(x)=\dfrac{\zeta(1-x)\tau^x\sec\left(\dfrac{\pi x}2\right)}{2\,\Gamma(x)}~~~(\text{equation }15)\\

	\begin{equation}
		2^x\pi^{x-1}\sin\left(\dfrac{\pi x}2\right)\Gamma(1-x)\zeta(1-x)=\dfrac{\zeta(1-x)\tau^x\sec\left(\dfrac{\pi x}2\right)}{2\,\Gamma(x)}
	\end{equation}

	\begin{equation}
		2^x\pi^{x-1}\sin\left(\dfrac{\pi x}2\right)\cos\left(\dfrac{\pi x}2\right)\Gamma(1-x)=\dfrac{\tau^x}{2\,\Gamma(x)}
	\end{equation}

	\begin{equation}
		4\cdot\tau^{x-1}\sin\left(\dfrac{\pi x}2\right)\cos\left(\dfrac{\pi x}2\right)\Gamma(1-x)=\dfrac{\tau^x}{\Gamma(x)}
	\end{equation}

	\begin{equation}
		2\sin\left(\dfrac{\pi x}2\right)\cos\left(\dfrac{\pi x}2\right)\Gamma(1-x)=\dfrac{\pi}{\Gamma(x)}
	\end{equation}

	\begin{equation}
		\Gamma(1-x)\Gamma(x)\sin\pi x=\pi
	\end{equation}

	\begin{equation}
		\Gamma(1-x)\Gamma(x)=\pi\csc\pi x
	\end{equation}

	\begin{equation}
		\csc x=\dfrac1\pi\Gamma\left(\dfrac x\pi\right)\Gamma\left(1-\dfrac x\pi\right)
	\end{equation}

	\begin{equation}
		\sin\pi x\ne0\Longrightarrow\zeta(1-x)\ne0
	\end{equation}

	\text{Therefore, due to (23), the domains also match.}

	\noindent\blacksquare
\end{section}

\pagebreak\begin{section}{Trig Functions In Terms of Cosecant}
	
	\noindent claim: The rest of the trig functions can be written in terms of cosecant

	\noindent Proof:

	\begin{equation}\sin x=\dfrac1{\csc x}\end{equation}

	\begin{equation}\cos x=\dfrac1{\csc\left(\dfrac\pi2-x\right)}\end{equation}

	\begin{equation}\tan x=\dfrac{\csc\left(\dfrac\pi2-x\right)}{\csc x}\end{equation}

	% \begin{equation}\csc x=\csc x\end{equation}
	
	\begin{equation}\sec x=\csc\left(\dfrac\pi2-x\right)\end{equation}

	\begin{equation}\cot x=\dfrac{\csc x}{\csc\left(\dfrac\pi2-x\right)}\end{equation}


	\begin{equation}\sinh x=-\dfrac i{\csc x}\end{equation}

	\begin{equation}\cosh x=\dfrac1{\csc\left(\dfrac\pi2-ix\right)}\end{equation}

	\begin{equation}\tanh x=-\dfrac{i\csc\left(\dfrac\pi2-ix\right)}{\csc ix}\end{equation}

	\begin{equation}\text{csch}\,x=i\csc ix\end{equation}

	\begin{equation}\text{sech}\,x=\csc\left(\dfrac\pi2-ix\right)\end{equation}

	\begin{equation}\coth x=\dfrac{i\csc ix}{\csc\left(\dfrac\pi2-ix\right)}\end{equation}

	\noindent\blacksquare
\end{section}

\pagebreak\begin{section}{Trig Functions In Terms of Gamma}
	
	\noindent These equations are just the previous ones with Gamma plugged in for csc.

	\begin{equation}\sin x=\pi\big/\,\Gamma\left(\dfrac x\pi\right)\Gamma\left(1-\dfrac x\pi\right)\end{equation}

	\begin{equation}
		\cos x=\pi\big/\,\Gamma\left(\dfrac12-\dfrac x\pi\right)\Gamma\left(\dfrac12+\dfrac x\pi\right)
	\end{equation}

	\begin{equation}
		\tan x=\Gamma\left(\dfrac12-\dfrac x\pi\right)\Gamma\left(\dfrac12+\dfrac x\pi\right)\big/\,\Gamma\left(\dfrac x\pi\right)\Gamma\left(1-\dfrac x\pi\right)
	\end{equation}
	
	\begin{equation}\csc x=\Gamma\left(\dfrac x\pi\right)\Gamma\left(1-\dfrac x\pi\right)\big/\,\pi\end{equation}

	\begin{equation}\sec x=\Gamma\left(\dfrac12-\dfrac x\pi\right)\Gamma\left(\dfrac12+\dfrac x\pi\right)\big/\,\pi\end{equation}

	\begin{equation}
		\cot x=\Gamma\left(\dfrac x\pi\right)\Gamma\left(1-\dfrac x\pi\right)\big/\,\Gamma\left(\dfrac12-\dfrac x\pi\right)\Gamma\left(\dfrac12+\dfrac x\pi\right)
	\end{equation}


	\begin{equation}\sinh x=-\pi i\big/\,\Gamma\left(\dfrac{ix}\pi\right)\Gamma\left(1-\dfrac{ix}\pi\right)\end{equation}

	\begin{equation}
		\cosh x=\pi\big/\,\Gamma\left(\dfrac12-\dfrac{ix}\pi\right)\Gamma\left(\dfrac12+\dfrac{ix}\pi\right)
	\end{equation}

	\begin{equation}
		\tanh x=-i\,\Gamma\left(\dfrac12-\dfrac{ix}\pi\right)\Gamma\left(\dfrac12+\dfrac{ix}\pi\right)\big/\,\Gamma\left(\dfrac{ix}\pi\right)\Gamma\left(1-\dfrac{ix}\pi\right)
	\end{equation}

	\begin{equation}
		\text{csch}\,x=i\,\Gamma\left(\dfrac{ix}\pi\right)\Gamma\left(1-\dfrac{ix}\pi\right)\big/\,\pi
	\end{equation}

	\begin{equation}
		\text{sech}\,x=\Gamma\left(\dfrac12-\dfrac{ix}\pi\right)\Gamma\left(\dfrac12+\dfrac{ix}\pi\right)\big/\,\pi
	\end{equation}

	\begin{equation}
		\coth x=i\,\Gamma\left(\dfrac{ix}\pi\right)\Gamma\left(1-\dfrac {ix}\pi\right)\big/\,\Gamma\left(\dfrac12-\dfrac {ix}\pi\right)\Gamma\left(\dfrac12+\dfrac{ix}\pi\right)
	\end{equation}

	\noindent where ``$\big/$" means divide everything on the left by everything on the right.
\end{section}

\pagebreak\begin{section}{Inverse Gamma Function}
	
	claim: $\eta(x)=\dfrac1\pi\csc^{-1}\dfrac x\pi\Gamma(1-\eta(x))$\\
	Proof:\\
	know: $\Gamma(x)\,\Gamma(1-x)=\pi\csc\pi x~~~$(equation 21)

	\begin{equation}\pi\csc\pi x=\Gamma(x)\,\Gamma(1-x)\end{equation}

	\begin{equation}\pi\csc\pi\eta(x)=x\,\Gamma(1-\eta(x))\end{equation}

	\begin{equation}\csc\pi\eta(x)=\dfrac x\pi\Gamma(1-\eta(x))\end{equation}

	\begin{equation}\pi\eta(x)=\csc^{-1}\dfrac x\pi\Gamma(1-\eta(x))\end{equation}

	\begin{equation}\eta(x)=\dfrac1\pi\csc^{-1}\dfrac x\pi\Gamma(1-\eta(x))\end{equation}

	\noindent\blacksquare

	\noindent There are 2 conditions to where this $\eta(x)$ is defined.\\
	$\eta(x)$ is \textbf{not} defined (over the real numbers) such that the following is true:\\
	\begin{equation}
		|x\,\Gamma(1-\eta(x))|<\pi~~\text{or}~~|x|<3.547373963...~~(V)
	\end{equation}\\
	$V$ is probably transcendental. $\eta(x)$ is guaranteed to be defined if equation (52) is false and $x$ is negative. $\eta(x)$ is defined for all integers and in small fields around each integer. The area defined around each integer gets smaller as $x$ increases, though it is unclear why at this time. If $\eta(x)$ is being recursed and $\dfrac x\pi$ is changed to $\dfrac{x^n}\pi$ in the innermost function, for some $n\in\mathbb N_2$, either the fields get larger, or there is a bug in Desmos' graphing calculations. See reference 2 for more information. going a different way after (48) gives the following equation:\\
	\begin{equation}
		\eta(x)=1-\eta\left(\dfrac\pi x\csc\pi\eta(x)\right)\Longrightarrow\eta(x)=\eta(\pi\csc(\pi\csc\pi x)\sin\pi x)
	\end{equation}\\
	This equation seems to be less useful because it has large derivatives everywhere when absolute valued, which gets more severe as it is recursed.
\end{section}

\pagebreak\begin{section}{Miscellaneous}
	(39) gives the following formulas after some manipulation:
	\begin{equation}
		\Gamma\left(x+\dfrac12\right)=\dfrac{\pi\sec\pi x}{\Gamma\left(\dfrac12-x\right)}
	\end{equation}\\
	replacing $x$ with $x+z$ and adjusting from gamma to factorial gives:\\
	\begin{equation}
		\left(x+\frac12+z\right)!=\left(2\left(z\,\text{mod}\,2\right)-1\right)\dfrac{\pi\sec\pi x}{\left(-\dfrac32-x-z\right)!}\forall z\in\mathbb Z
	\end{equation}\\
	This only simplifies things if $z$ is an integer because $\sec(x+\tau)=\sec x$,\\
	and $\sec(x+\pi)=-\sec x$. rewriting (21) from gamma to factorial gives:
	\begin{equation}
		(x-1)!(-x)!=\pi\csc\pi x
	\end{equation}\\
	\begin{equation}
		(-x)!=\dfrac{\pi x\csc\pi x}{x!}
	\end{equation}\\
	\begin{equation}
		\left(x+\dfrac1c\right)=\dfrac{-\pi\csc\dfrac{cx+1}c\pi}{\left(-\dfrac{c+1}c-x\right)!}\forall c\in\mathbb R
	\end{equation}\\
	\begin{equation}
		x!=(x\,\text{mod}_n\,1)\left[\prod_{k=\text{sgn}(|n-x|+n-x)}^{|\lfloor x-n\rfloor|+\text{sgn}(n-x-|n-x|)}\left(~x+\text{sgn}(n-x)k~\right)\right]^{\text{sgn}(x-n)}
	\end{equation}\\
	(60) is useful where $x!$ is non self-dependantly defined for $x\in[n,n+1]$.
	\pagebreak\\
	Proof sine is odd using gamma:\\
	claim: $\sin(x)=-\sin(-x)$:\\
	Proof:\\
	\begin{equation}
		(57)~x\mapsto-x\Longrightarrow(-x)!=\dfrac{-\pi x\csc(-\pi x)}{x!}
	\end{equation}\\
	\centerline{$(-x)!=\dfrac{\pi x\csc\pi x}{x!}\hspace{1.6em}(57)$}\\
	\begin{equation}
		\dfrac{\pi x\csc\pi x}{x!}=\dfrac{-\pi x\csc(-\pi x)}{x!}\Longrightarrow\csc x=-\csc(-x)\Longrightarrow\sin(x)=-\sin(-x)
	\end{equation}\\
	\blacksquare\\

	\noindent Proof for the value of $\frac12$ factorial\\
	claim: $\left(\dfrac12\right)!=\dfrac{\sqrt\pi}2$\\
	Proof:\\
	\begin{equation}
		x!=\dfrac{(x+1)!}{x+1}\Longrightarrow\left(-\dfrac12\right)!=2\left(\dfrac12\right)
	\end{equation}\\
	\begin{equation}
		(-x)!=\dfrac{\pi x\csc\pi x}{x!}\Longrightarrow\left(-\dfrac12\right)!=\dfrac\pi{2\left(\dfrac12\right)}
	\end{equation}\\
	The formula for (63) comes from (57) and is trivial to convert.\\
	(62) and (63) imply the following:\\
	\begin{equation}
		2\left(\dfrac12\right)=\dfrac\pi{2\left(\dfrac12\right)}\Longrightarrow4\left(\dfrac12\right)^2=\pi\Longrightarrow\left(\dfrac12\right)=\dfrac{\sqrt\pi}2
	\end{equation}\\
	using this strategy only works if $(c_1-x)=(c_2+x)$ for some integers $c_1$ and $c_2$. this limits it to $(z+\frac12)$ for integers $z$. Also, you can only apply (63) an odd number of times or the factorials will cancel and leave a trigonometric equation, which could be interesting later. If you apply (63) twice, you get back to where you started because everything new cancels.\\
	\blacksquare
\end{section}

\pagebreak\begin{section}{References}

	\noindent- \url{https://en.wikipedia.org/wiki/Riemann\_zeta\_function\#Riemann's\_functional\_equation}\\
	\indent\text{Link for equation 8}\\

	\noindent- \url{https://www.desmos.com/calculator/t51xnveadf}\\
	\indent\text{Extra information for part 6, made by me}\\

	\noindent- \url{https://www.github.com/drizzt536/files/tree/main/TeX/gamma}\\
	\indent\text{The files for the most recent version of this pdf and the LaTeX code}
\end{section}


\end{document}
