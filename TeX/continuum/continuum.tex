% file for pdflatex.exe

% countability.tex v2.1 (c) | Copyright 2023 Daniel E. Janusch

% This file is licensed by https://raw.githubusercontent.com/drizzt536/files/main/LICENSE
% and may only be copied IN ITS ENTIRETY under penalty of law.

\documentclass[12pt]{article}
\usepackage{amssymb, latexsym, amsmath, amsthm}
\usepackage[margin=1in]{geometry}

\addtolength{\topmargin}{-0.5in}
\renewcommand{\qedsymbol}{\blacksquare}
\newtheorem{thm}{Theorem}

\begin{document}


\title{On the Continuum Hypothesis and the Correspondence of Infinite Sets with the Natural Numbers}
\author{Daniel E. Janusch\\Dedicated to Maeby, my favorite kitty}
\maketitle

\begin{section}{Background and Theorems}\label{sec:background}
	If a set is ``countable,"  it is bijectable to the natural numbers or a subset of them. Any set
	that is finite will have this property and the only question is if all infinite sets have it.
	$|S|$ denotes the cardinality of $S$ for any set $S$. Iterated set multiplication of the first
	kind gives ordered groups with more than two elements, rather than ordered pairs of ordered pairs.\\
	\begin{align}
		R\times S :=\{(r,s):r\in R\land s\in S\}~~(1 & \text a)
		\hspace{2.5em}
		R\cdot S :=\{r\cdot s:r\in R\land s\in S\}~~~(1\text b)
		\\
		R(\vec v\,):=\left\{\sum_{n=1}^{\dim\vec v}a_n\vec v_n:a_n\in R\right\}~~~(2 & \text a)
		\hspace{2.5em}
		R[\vec v\,]:=\left\{\prod_{n=1}^{\dim\vec v}(a_n+\vec v_n):a_n\in R\right\}~~~(2\text b)
	\end{align}

	\begin{thm}\label{thm:finite unions}
		\emph{Unions of Countable Sets}\\
		\indent\emph{If any pair of sets $R$ and $S$ are countable, $R(a)\cup S(b)$ is countable for all $a$ and $b$.}
	\end{thm}\begin{proof}
		Since $R$ and $S$ are countable, $R(a)$ and $S(b)$ are as well because this just
		multiplies the already present elements by a constant. Interlace the elements of
		$R(a)$ with the elements of $S(b)$ skipping over any duplicates. If either of the
		sets runs out, meaning it is finite, stop interlacing and only pull elements from
		the infinite set. Similar logic works using $R[a]$ and $S[b]$ or combinations of
		those, $R(a)$, and $S(b)$.
	\end{proof}

	\begin{thm}\label{thm:multiplications}
		\emph{Multiplications of Countable Sets}\\
		\indent\emph{If any pair of sets $R$ and $S$ are countable, $R\times S$ and $R\cdot S$ are countable.}
	\end{thm}\begin{proof}
		Put ordered pairs using elements all the of $R$ and $S$ into a table as shown below
		and go along the diagonals getting every element in the product and not ``missing"
		any. The numbers and colons are the indices. The first 25 are labeled if in view.
		Similar logic works for $R\cdot S$. This is also algorithmically viable as shown
		in Section~\ref{sec:algorithm code}.\vspace{0.5em}\\
		\centerline{\begin{array}{|c|c|c|c|c|}
			\hline 1:(R_1, S_1) & 2:(R_1, S_2) & 4:(R_1, S_3) & 7:(R_1, S_4) & \cdots\\
			\hline 3:(R_2, S_1) & 5:(R_2, S_2) & 8:(R_2, S_3) & 12:(R_2, S_4) & \cdots\\
			\hline 6:(R_3, S_1) & 9:(R_3, S_2) & 13:(R_3, S_3) & 18:(R_3, S_4) & \cdots\\
			\hline 10:(R_4, S_1) & 14:(R_4, S_2) & 19:(R_4, S_3) & 25:(R_4, S_4) & \cdots\\
			\hline \vdots & \vdots & \vdots & \vdots & \ddots\\
			\hline
		\end{array}}\vspace{-0.8em}\\
	\end{proof}

	\pagebreak\\
	\begin{thm}\label{thm:subsets}
		\emph{Subsets, Intersections, and Differences of Countable Sets}\\
		\indent\emph{If any set $R$ is countable, and a set $S$ is a loose subset of $R$, $S$ is also countable.}
	\end{thm}\begin{proof}
		For each element in $R$ but not in $S$ ($x\in R\setminus S$), move its index to the beginning
		of $R$ and then remove it. That way, the set is still guaranteed to be countable. Since every set
		along the way is countable, so is $S$. This also implies that set intersections and differences
		are countable because these operations return subsets of the original sets.
	\end{proof}

	\begin{thm}\label{thm:rings}
		\emph{Sum and Product ``Rings" From Countable Sets}\\
		\indent\emph{If any set $R$ is countable, $R(a, b)$ and $R[a, b]$ are also countable
		for all $a$ and $b$.}
	\end{thm}\begin{proof}
		Create 2 new intermediate sets $S_a := \{a\cdot r:r\in R\}$ and $S_b := \{b\cdot r:r\in R\}$.
		$S_a\times S_b$ gives a set with the same countability as $R(a,b)$, only with
		ordered pairs instead of addition. This is countable via Theorem~\ref{thm:multiplications}.
		Similar logic works for $R[a,b]$. They're not actually rings, it's just the same syntax for
		a similar thing.
	\end{proof}

	\begin{thm}\label{thm:powers}
		\emph{Powers of Countable Sets}\\
		\indent\emph{If any set $R$ is countable, $R^n$ is also countable for all natural numbers $n$
		\,($\mathbb N_0$).}
	\end{thm}\begin{proof}
		$R^n$ can be factored as $R\times R^{n-1}$ which is countable if $R^{n-1}$ is countable.
		$R^{n-1}$ can be factored as $R\times R^{n-2}$ which is countable if $R^{n-2}$ is countable,
		et cetera. This can be simplified continuously until it becomes countable if $R^2$ or
		$R\times R$ is countable, which is countable via Theorem~\ref{thm:multiplications}. If $n=0$,
		the output set is all the groups of zero elements from $R$, or just $\O$, which is countable
		since $\left|\O\right|$ is finite. 
	\end{proof}

	\begin{thm}\label{thm:infinite unions}
		\emph{Countably Infinite Unions of Countable Sets}\\
		\indent\emph{If all sets $R_i$ are countable, $\displaystyle\bigcup_{i=1}^{\aleph_0} R_i$ is also
		countable, where $\aleph_0\equiv\left|\mathbb N\right|=\infty$.}
	\end{thm}\begin{proof}
		Give each set a column and put all its elements in that column going down. Do this for every
		set and using the same argument as in Theorem~\ref{thm:multiplications}, one can show that
		every element is indexed and none are skipped. Every column represents a different $R_i$
		and the rows represent a different $\left(R_i\right)_j$ or $R_{i,j}$. The numbers before the
		colons are the output index. The first 85 are shown if in view. Again this is algorithmically viable, even
		though this table has a slightly different numbering than that of Theorem~\ref{thm:multiplications}.
		\vspace{1em}\\
		\centerline{\begin{array}{|c|c|c|c|c|c|c|c|} % 8 columns
			\hline \phantom{0}1:R_{1,1} & \phantom{0}2:R_{2,1} & \phantom{0}6:R_{3,1} & \phantom{0}7:R_{4,1} & 15:R_{5,1} & 16:R_{6,1} & 28:R_{7,1} & \cdots\\
			\hline \phantom{0}3:R_{1,2} & \phantom{0}5:R_{2,2} & \phantom{0}8:R_{3,2} & 14:R_{4,2} & 17:R_{5,2} & 27:R_{6,2} & 30:R_{7,2} & \cdots\\
			\hline \phantom{0}4:R_{1,3} & \phantom{0}9:R_{2,3} & 13:R_{3,3} & 18:R_{4,3} & 26:R_{5,3} & 31:R_{6,3} & 43:R_{7,3} & \cdots\\
			\hline 10:R_{1,4} & 12:R_{2,4} & 19:R_{3,4} & 25:R_{4,4} & 32:R_{5,4} & 42:R_{6,4} & 49:R_{7,4} & \cdots\\
			\hline 11:R_{1,5} & 20:R_{2,5} & 24:R_{3,5} & 33:R_{4,5} & 41:R_{5,5} & 50:R_{6,5} & 62:R_{7,5} & \cdots\\
			\hline 21:R_{1,6} & 23:R_{2,6} & 34:R_{3,6} & 40:R_{4,6} & 51:R_{5,6} & 61:R_{6,6} & 72:R_{7,6} & \cdots\\
			\hline 22:R_{1,7} & 35:R_{2,7} & 39:R_{3,7} & 52:R_{4,7} & 60:R_{5,7} & 73:R_{6,7} & 85:R_{7,7} & \cdots\\
			\hline \vdots & \vdots & \vdots & \vdots & \vdots & \vdots & \vdots & \ddots\\
			\hline
		\end{array}}
	\end{proof}
\end{section}

\pagebreak\begin{section}{Applications of the Theorems}\label{sec:applications}
	\vspace{-1em}\begin{subsection}{Countability of The Integers}\label{subsec:applications.integers}
		Claim: $\mathbb Z$ is a countable set.
		\begin{proof}
			Let $R=\mathbb N_0$, $S=\mathbb N_1$. $R$ and $S$ are countable axiomatically,
			being the naturals themselves. $\mathbb Z\equiv R\cup S(-1)$. This is countable
			via Theorem~\ref{thm:finite unions}.
		\end{proof}
	\end{subsection}

	\vspace{-1em}\begin{subsection}{Countability of the Rationals}\label{subsec:applications.rationals}
		Claim: $\mathbb Q$ is a countable set.
		\begin{proof}
			The rationals are basically just ordered pairs of integers and naturals.
			$R:=\mathbb Z$, $S:=\mathbb N_1$. $R\times S$ defines all of these pairs.
			$R$ is countable via Section~\ref{subsec:applications.integers} and $S$ is
			countable axiomatically. $R\times S$ is thus countable via Theorem~\ref{thm:multiplications}.
			This argument and Theorem~\ref{thm:multiplications} derives from Cantor's argument for the rationals.
		\end{proof}
	\end{subsection}

	\vspace{-1em}\begin{subsection}{Countability of the Reals from Zero to One}\label{subsec:applications.reals 0 to 1}
		Claim: $\left\{x:x\in\mathbb R\land0\le x<1\right\}$ is a countable set.
		\begin{proof}
			Let $R$ be some set. Each $R_i$ can be defined to be the digits of $i$
			reversed with ``0." at the beginning and infinite trailing zeros at the
			end, for any $i\in\mathbb N_0$. For example, $R_{246}=0.642\overline0$ and
			$R_0=0.0\overline0=0$. This set is countable because it was defined to be
			countable; each natural number corresponds to a single element. This set
			contains every real number in the range because every possible sequence
			of digits is in it. The sequence trends upwards, asymptotically approaching 1,
			though it fluctuates wildly along the way.
		\end{proof}
	\end{subsection}

	\vspace{-1em}\begin{subsection}{Countability of the Non-Negative Reals}\label{subsec:applications.non negative reals}
		Claim: $\mathbb R_{\ge0}$ or equivalently $\left\{x:x\in\mathbb R\land x\ge0\right\}$ is a countable set.
		\begin{proof}
			The set of real numbers from zero to one is countable via
			Section~\ref{subsec:applications.reals 0 to 1}. Let $R$ be the same set used in that proof.
			$\mathbb R^+=\displaystyle\bigcup_{i=1}^{\aleph_0}S_i$ where $S_i:=R[i-1]$. This
			is countable via Theorem~\ref{thm:infinite unions}. The following table illustrates this.
			\vspace{0.3em}\\
			\vspace{-0.9em}\centerline{\begin{array}{|c|c|c|c|c|c|c|} % 7 columns
				\hline 0.\overline0 & 1.\overline0 & 2.\overline0 & 3.\overline0 & 4.\overline0 & 5.\overline0 & \cdots\\
				\hline 0.1\overline0& 1.1\overline0& 2.1\overline0& 3.1\overline0& 4.1\overline0& 5.1\overline0& \cdots\\
				\hline 0.2\overline0& 1.2\overline0& 2.2\overline0& 3.2\overline0& 4.2\overline0& 5.2\overline0& \cdots\\
				\hline 0.3\overline0& 1.3\overline0& 2.3\overline0& 3.3\overline0& 4.3\overline0& 5.3\overline0& \cdots\\
				\hline \vdots & \vdots & \vdots & \vdots & \vdots & \vdots & \cdots\\
				\hline 0.7\overline0& 1.7\overline0& 2.7\overline0& 3.7\overline0& 4.7\overline0& 5.7\overline0& \cdots\\
				\hline 0.8\overline0& 1.8\overline0& 2.8\overline0& 3.8\overline0& 4.8\overline0& 5.8\overline0& \cdots\\
				\hline 0.9\overline0& 1.9\overline0& 2.9\overline0& 3.9\overline0& 4.9\overline0& 5.9\overline0& \cdots\\
				\hline 0.01\overline0&1.01\overline0&2.01\overline0&3.01\overline0&4.01\overline0&5.01\overline0&\cdots\\
				\hline \vdots & \vdots & \vdots & \vdots & \vdots & \vdots & \ddots\\
				\hline
			\end{array}}
		\end{proof}
	\end{subsection}

	\pagebreak\begin{subsection}{Countability of the Reals}\label{subsec:applications.reals}
		Claim: $\mathbb R$ is a countable set which implies
		$\left|\mathcal P(\mathbb N)\right|\equiv\left|\mathbb N\right|$.
		\begin{proof}
			$\mathbb R_{\ge 0}$ is countable via Section~\ref{subsec:applications.non negative reals}.
			$\mathbb R_{\ge0}(1, -1)\equiv\mathbb R$. This is countable via Theorem~\ref{thm:rings}.
			This conclusion can be further reinforced by the algorithmic methods later on. For each
			real number $x$, there exists at least 1 sequence of elements in $R$, each element with a
			higher index than the last, where the limit of the sequence equals $x$. This is because
			every sequence of digits is contained by $R$.
		\end{proof}
	\end{subsection}

	\begin{subsection}{Miscellaneous Number Classes}\label{subsec:applications.miscellaneous}
		\begin{subsubsection}{Algebraic Real and Transcendental Real Numbers}
			\label{subsubsec:applications.misc.reals}
			Claim: $\mathbb A$ and $\mathbb T$ are countable sets.
			\begin{proof}
				$\mathbb R$ is a countable set via Section~\ref{subsec:applications.reals}.
				Since the algebraic reals and transcendental reals are both subsets of the
				reals, they are countable via Theorem~\ref{thm:subsets}. They are also
				countable over the complex numbers using Section~\ref{subsec:applications.complex}
				and the same logic.
			\end{proof}
		\end{subsubsection}

		\begin{subsubsection}{Imaginary Numbers}\label{subsubsec:applications.misc.imaginary}
			Claim: $\mathbb I$ is a countable set.
			\begin{proof}
				$\mathbb R$ is a countable set via Section~\ref{subsec:applications.reals}.
				$\mathbb R(\sqrt{-1})\equiv\mathbb I$. This is countable via
				Theorem~\ref{thm:finite unions}, or more precisely $R(\sqrt{-1})\cup\O$
				is countable. The union of any set and the null set is itself.
			\end{proof}
		\end{subsubsection}
	\end{subsection}

	\begin{subsection}{Countability of the Complex Numbers}\label{subsec:applications.complex}
		Claim: $\mathbb C$ is a countable set.
		\begin{proof}
			$\mathbb R$ is a countable set via Section~\ref{subsec:applications.reals}.
			$\mathbb R(1, \sqrt{-1})\equiv\mathbb C$. This is countable via Theorem~\ref{thm:rings}.
			The identity used here stems from the rectangular form of complex numbers.
		\end{proof}
	\end{subsection}
\end{section}

\begin{section}{Cantor's Diagonal Argument}\label{sec:diagonal argument}
	According to Georg Cantor in 1891, If someone is trying to list all the real numbers,
	they can always find a number that is not in the list using his ``Diagonal Argument".
	This argument is basically the same as the following: Suppose someone is trying to make
	a set of every natural number. They first add zero to the set and the set is $\{0\}$,
	then they could say, ``one isn't in the set." When they add one and have $\{0,1\}$,
	they can say ``two isn't in the set," ``then three isn't in the set," ``four isn't in
	the set," et cetera. No matter how many natural numbers they add, they can always find
	one not in it. Using Cantor's same logic, this seems to be saying that the naturals
	cannot be corresponded one to one with the naturals, which is clearly wrong. The
	naturals \emph{are} the naturals.
\end{section}

\pagebreak\begin{section}{Counting the Reals Algorithmically}\label{sec:algorithm code}
	The following Node JS code prints out every real number to the console delimited by a comma and space.
	The only problems are that it prints out both $0$ and $-0$, and the functions return strings. Both of
	these are easily resolvable though. There is a reference to this source code and the C version in
	Section~\ref{sec:references}.

	\noindent\texttt{function R(i, j) \{ return i < 0n || j < 0n ?\\
		$\indent$NaN :\\
		$\indent$\textasciigrave\$\{i\}.\textasciigrave + \textasciigrave\$\{j\}\textasciigrave.split("").reverse().join("");\\
		\}\\
		function sgn(x) \{ return x < 0n ? -1n : BigInt(x > 0n) \}\\
		function triangleNumber(x) \{ return x * (x + 1n) / 2n \}\\
		function isqrt(n) \{\\
		$\indent$if (n < 2n) return n;\\
		$\indent$var x0, x1 = n / 2n;\\
		$\indent$do x0 = x1, x1 = ( x0 + n / x0 ) / 2n;\\
		$\indent$while ( x1 < x0 );\\
		$\indent$return x0;\\
		\}\\
		function greatestTriangleIndex(x) \{\\
		$\indent$return (\\
		$\indent$$\indent$isqrt(1n + 8n*x) +\\
		$\indent$$\indent$sgn( 1n - sgn(1n - isqrt(1n + 8n*x) \% 2n) )\\
		$\indent$) / 2n - 1n;\\
		\}\\
		function generateIndices(x) \{\\
		$\indent$const u = greatestTriangleIndex(x), k = triangleNumber(u);\\
		$\indent$return [x - k, u - x + k]; // these can be swapped, it doesn't matter.\\
		\}\\
		function getReal(index) \{\\
		$\indent$try \{ index = BigInt(index) \} catch \{ return NaN \}\\
		$\indent$let negative = 0;\\
		$\indent$index \% 2n \&\& (negative = 1, index--);\\
		$\indent$let t = generateIndices(index /= 2n);\\
		$\indent$return (negative ? "-" : "") + R(t[0], t[1]);\\
		\}\\
		function generateReals(maxIndex=1n) \{\\
		$\indent$// -1n acts as infinity.\\
		$\indent$try \{ maxIndex = BigInt(maxIndex) \} catch \{ return \}\\
		$\indent$for (var i = 0n; i !== maxIndex; i++) \{\\
		$\indent$$\indent$process.stdout.write( getReal(i) );\\
		$\indent$$\indent$i + 1n !== maxIndex \&\& process.stdout.write(", ");\\
		$\indent$\}\\
		\}\\
		generateReals(-1n);
	}
\end{section}

\pagebreak\begin{section}{References}\label{sec:references}

	\begin{itemize}
		\item\url{https://en.wikipedia.org/wiki/Cantor's\_diagonal\_argument}\\
		\text{Wikipedia page with elaboration on Section~\ref{sec:diagonal argument}}

		\item\url{https://www.digizeitschriften.de/dms/img/?PID=GDZPPN002113910&physid=phys84#navi}\\
		\text{Georg Cantor's 1891 article with the diagonal argument. Same source as on wikipedia.}

		\item\url{https://en.wikipedia.org/wiki/Continuum\_hypothesis}\\
		\text{Wikipedia page with elaboration on Section~\ref{subsec:applications.reals}}

		\item\url{https://www.github.com/drizzt536/files/tree/main/TeX/continuum}\\
		\text{The files for the most recent version of this pdf and the \LaTeX~code}

		\item\url{https://raw.githubusercontent.com/drizzt536/files/main/JavaScript/continuum.js}\\
		\text{The raw JavaScript source code for the Section~\ref{sec:algorithm code}}

		\item\url{https://raw.githubusercontent.com/drizzt536/files/main/C/continuum.c}\\
		\text{The raw C source code for the Section~\ref{sec:algorithm code}}
	\end{itemize}
	\vspace{1em}\\
	Editors:
	\begin{itemize}
		\item Daniel E. Janusch
		\item Valerie Janusch, my mom
	\end{itemize}
	\vspace{3em}\\
	This document is licensed under https://raw.githubusercontent.com/drizzt536/files/main/LICENSE
	and may only be copied IN ITS ENTIRETY under penalty of law.
\end{section}

\end{document}
