% file for pdflatex.exe

% continuum.tex v7.-1 (c) | Copyright 2023 Daniel E. Janusch

% This file is licensed by https://raw.githubusercontent.com/drizzt536/files/main/LICENSE
% and may be copied ONLY IN ITS ENTIRETY under penalty of law.


\documentclass[12pt]{article}
\usepackage{amssymb, latexsym, amsmath, amsthm, upquote}
\usepackage[margin=1in]{geometry}

\addtolength{\topmargin}{-0.5in}
\newtheorem{thm}{Theorem}

\renewcommand{\qedsymbol}{\blacksquare}

\begin{document}


\title{On the Continuum Hypotheses and the Correspondence of Infinite Sets' Elements with the
	Natural Numbers}
\author{Daniel E. Janusch\\Dedicated to Maeby, my favorite kitty}
\maketitle

\begin{section}{Introduction and Countability Theorems}\label{sec:intro}
	Any ``countable" set is equinumerous to an improper subset of the natural numbers, and as
	follows, there exists a bijection between them. Any finite set has this property, an important
	question being if all \emph{infinite} sets do. In all infinite processes, one should
	think of the steps as executing simultaneously rather than separately and successively.
	$|S|$ and $C(S)$ denote the cardinality and countability of a set $S$ respectively.
	Equations (2a) and (2b) describe what will be called ``sum ring operators,"
	``product ring operators," and collectively ``ring operators." Most of the theorems use
	``$\iff$" instead of ``$\implies$" because Theorem~\ref{thm:subsets}.1 works as an
	inversion. Equation~(1a) doesn't nest groups but flattens them in the same way as $R^n$
	normally does. While sets don't have an indexing traditionally, it is easier to pretend
	they can to see if they are countable. $R$, $S$ \& $T$ are sets. $a$ \& $b$ are reals.
	$n$, $k$, $i$ \& $j$ are naturals.\vspace{-1em}\\
	\begin{align}
		R\times S & :=\left\{(r,s):r\in R\land s\in S\right\}~~~~(1\text a)
		\hspace{2em}
		R\cdot S:=\left\{r\cdot s:r\in R\land s\in S\right\}~~~~~\,(1\text b)\\
		R(\vec v\,) & :=\left\{\sum_{n=1}^{\text{dim}\,\vec v}(r_n\cdot\vec
		v_n):r_k\in R\right\}~~~(2\text a)
		\hspace{2.1em}
		R[\,\vec v\,]:=\left\{\prod_{n=1}^{\text{dim}\,\vec v}(r_n+\vec v_n):r_k\in
		R\right\}~~(2\text b)\\
		\aleph_0 & := |\mathbb N|\\
		C(S) & :=\begin{cases}
			1,\,\left[S\text{ countable}\right]\phantom{un}~~(|S|\leqslant\aleph_0)\\
			0,\,\left[S\text{ uncountable}\right]~~(|S|>\aleph_0)\\
		\end{cases}
	\end{align}

	\begin{thm}\label{thm:finite unions}
		\emph{
			Singular Unions of Countable Sets\\
			\indent If any sets $R$ and $S$ are countable, $R(a)\cup S(b)$ is countable
			for all numbers $a$ and $b$.\vspace{0.4em}\\
			\indent1. $C(R)\land C(S)\iff C(R\left(a\right)\cup S\left(b\right))~\forall a,b$
			\hspace{1em}
			2. $C(R)\land C(S)\iff C(R\left(a\right)\cup S[\,b\,])~\forall a,b$\\
			\indent3. $C(R)\land C(S)\iff C(R[\,a\,]\cup S\left(b\right))~\forall a,b$
			\hspace{1em}
			4. $C(R)\land C(S)\iff C(R[\,a\,]\cup S[\,b\,])~\forall a,b$
		}
	\end{thm}\begin{proof}
		$R(a)$ and $S(b)$ exist due to the Axiom Schema of Replacement. They are countable because ring
		operators don't change cardinality. For case 1, Interlace the elements of $R(a)$ with those
		of $S(b)$, skipping over any duplicates. If either set is finite, append its elements to the
		beginning of the other set, excluding duplicates. The same logic works with the other 3 cases.
	\end{proof}

	\pagebreak\begin{thm}\label{thm:multiplications}
		\emph{
			Cartesian Products and Element-Wise Multiplications of Countable Sets\\
			\indent If any sets $R$ and $S$ are countable, both $R\times S$ and
			$R\cdot S$ are countable.\vspace{0.4em}\\
			\indent1. $C(R)\land C(S)\iff C(R\times S)$\\
			\indent2. $C(R)\land C(S)\iff C(R\cdot S)$
		}
	\end{thm}\begin{proof}
		To construct $R\times S$, put all possible ordered pairs into a table as shown. Go along the
		diagonals adding each value to a new set $T$. Every element will be both included and indexed
		in $T$, and thus a bijection exists between $T$ ($R\times S$) and $\mathbb N$. Each table
		element is of the form ``$T$-index : value," the first 25 being labeled if in view. The same
		core logic works for $R\cdot S$. This method is algorithmically viable (see
		Section~\ref{sec:algorithm code}), and the $T$-indices relate strongly to triangle numbers
		(see Section~\ref{sec:algorithm mathematically}).\vspace{0.5em}\\
		\centerline{\begin{array}{|c|c|c|c|c|}
			\hline \phantom{0}1:(R_1, S_1) & \phantom{0}2:(R_1, S_2) & \phantom{0}4:(R_1, S_3) &
			\phantom{0}7:(R_1, S_4) & \cdots\\
			\hline \phantom{0}3:(R_2, S_1) & \phantom{0}5:(R_2, S_2) & \phantom{0}8:(R_2, S_3) &
			12:(R_2, S_4) & \cdots\\
			\hline \phantom{0}6:(R_3, S_1) & \phantom{0}9:(R_3, S_2) & 13:(R_3, S_3) & 18:(R_3, S_4) &
			\cdots\\
			\hline 10:(R_4, S_1) & 14:(R_4, S_2) & 19:(R_4, S_3) & 25:(R_4, S_4) & \cdots\\
			\hline \vdots & \vdots & \vdots & \vdots & \ddots\\
			\hline
		\end{array}}\vspace{-1.1em}
	\end{proof}

	\begin{thm}\label{thm:subsets}
		\emph{
			Subsets, Intersections, and Differences of Countable Sets\\
			\indent If a set $R$ is countable, if any set $S$ is an improper subset of $R$, is
			countable.
			\vspace{0.4em}\\
			\indent1. $C(R)\land S\subseteq R\implies C(S)$\\
			\indent2. $C(R)\land C(S)\implies C(R\cap S)$\\
			\indent3. $C(R)\implies C(R\smallsetminus S)$
		}
	\end{thm}\begin{proof}
		Since subsets of $\mathbb N$ are countable and $R$ bijects $\mathbb N$, all subsets of $R$
		must also be countable due to the Axiom Schema of Replacement. This implies that set
		intersections and differences are countable since these operations return subsets of the left
		input. Case 3 doesn't require $S$ to be countable because its elements cannot appear in the
		output set.
	\end{proof}

	\begin{thm}\label{thm:rings}
		\emph{
			Two-Argument Ring Operators on Countable Sets\\
			\indent If any set $R$ is countable, $R(a, b)$ and $R[a, b]$ are also
			countable for any constants $a, b$.\vspace{0.4em}\\
			\indent1. $C(R)\iff C(R(a, b))~\forall a,b$\\
			\indent2. $C(R)\iff C(R[\,a, b\,])~\forall a,b$
		}
	\end{thm}\begin{proof}
		$R(a,b)$ can be tabulated with axes $n$ and $k$, where the corresponding table box is
		$aR_n + bR_k$. This is countable in the same way as Theorem~\ref{thm:multiplications}
		except with addition as the combining operator instead of multiplication.
		Similar logic works for $R[a,b]$.\vspace{0.5em}\\
		\centerline{\begin{array}{|c|c|c|c|c|c|} % 6 columns
			\hline aR_1 + bR_1 & aR_1 + bR_2 & aR_1 + bR_3 & aR_1 + bR_4 & aR_1 + bR_5 & \cdots\\
			\hline aR_2 + bR_1 & aR_2 + bR_2 & aR_2 + bR_3 & aR_2 + bR_4 & aR_2 + bR_5 & \cdots\\
			\hline aR_3 + bR_1 & aR_3 + bR_2 & aR_3 + bR_3 & aR_3 + bR_4 & aR_3 + bR_5 & \cdots\\
			\hline aR_4 + bR_1 & aR_4 + bR_2 & aR_4 + bR_3 & aR_4 + bR_4 & aR_4 + bR_5 & \cdots\\
			\hline aR_5 + bR_1 & aR_5 + bR_2 & aR_5 + bR_3 & aR_5 + bR_4 & aR_5 + bR_5 & \cdots\\
			\hline \vdots & \vdots & \vdots & \vdots & \vdots & \ddots\\
			\hline
		\end{array}}
	\end{proof}

	\pagebreak\begin{thm}\label{thm:powers}
		\emph{
			Finite Natural Powers of Countable Sets\\
			\indent If any set $R$ is countable, $R^n$ is also countable for all natural
			numbers $n$.\vspace{0.4em}\\
			\indent1. $C(R)\iff C(R^n)~~\forall n\in\mathbb N_0<\infty$
		}
	\end{thm}\begin{proof}
		$R^n$ can be factored as $R\times R^{n-1}$ (eqn.~1a flattens outputs), which is countable if
		$R^{n-1}$ is countable (Theorem~\ref{thm:multiplications}.1). This can be applied recursively
		until it becomes countable if $R^2$ or $R\times R$ is countable, which is countable via
		Theorem~\ref{thm:multiplications}.1. If $n=0$, the output set is all the groups of zero
		elements ($\emptyset$) from $R$; this is countable since $\left|\emptyset\right|$ is finite.
	\end{proof}

	\begin{thm}\label{thm:infinite unions}
		\emph{
			Countably Infinite Unions of Countably Infinite Sets\\
			\indent If all sets $R_n$ are countable, their union is also countable.
			\vspace{0.4em}\\
			\indent1. $C(R_n)\forall n\iff C\!\left(\displaystyle\bigcup_{n=1}^{\aleph_0}R_n\right)$
		}
	\end{thm}\begin{proof}
		To create the union, give each set $R_n$ a column in a table $T$ and put all its elements
		sequentially in that column going down. Do this for every $R_n$, and using the same method
		as in Theorem~\ref{thm:multiplications}, one can show that $T$ bijects $\mathbb N$. Every
		column represents a different $R_n$ and each row represents a different $\left(R_n\right)_k$
		or $R_{n,k}$. This is algorithmically viable (see Section~\ref{sec:algorithm code}). Refer to
		Theorem~\ref{thm:multiplications} for more information on the algorithm or the table format.
		\vspace{0.5em}\\
		\centerline{\begin{array}{|c|c|c|c|c|c|c|c|} % 8 columns
			\hline \phantom{0}1:R_{1,1} & \phantom{0}2:R_{2,1} & \phantom{0}6:R_{3,1} &
			\phantom{0}7:R_{4,1} & 15:R_{5,1} & 16:R_{6,1} & 28:R_{7,1} & \cdots\\
			\hline \phantom{0}3:R_{1,2} & \phantom{0}5:R_{2,2} & \phantom{0}8:R_{3,2} & 14:R_{4,2} &
			17:R_{5,2} & 27:R_{6,2} & 30:R_{7,2} & \cdots\\
			\hline \phantom{0}4:R_{1,3} & \phantom{0}9:R_{2,3} & 13:R_{3,3} & 18:R_{4,3} & 26:R_{5,3} &
			31:R_{6,3} & 43:R_{7,3} & \cdots\\
			\hline 10:R_{1,4} & 12:R_{2,4} & 19:R_{3,4} & 25:R_{4,4} & 32:R_{5,4} & 42:R_{6,4} &
			49:R_{7,4} & \cdots\\
			\hline 11:R_{1,5} & 20:R_{2,5} & 24:R_{3,5} & 33:R_{4,5} & 41:R_{5,5} & 50:R_{6,5} &
			62:R_{7,5} & \cdots\\
			\hline 21:R_{1,6} & 23:R_{2,6} & 34:R_{3,6} & 40:R_{4,6} & 51:R_{5,6} & 61:R_{6,6} &
			72:R_{7,6} & \cdots\\
			\hline 22:R_{1,7} & 35:R_{2,7} & 39:R_{3,7} & 52:R_{4,7} & 60:R_{5,7} & 73:R_{6,7} &
			85:R_{7,7} & \cdots\\
			\hline \vdots & \vdots & \vdots & \vdots & \vdots & \vdots & \vdots & \ddots\\
			\hline
		\end{array}}
	\end{proof}
\end{section}

\begin{section}{Countabilities of Common Sets}\label{sec:applications}
	\begin{subsection}{Countability of the Integers}\label{subsec:applications.integers}
		Claim: $\mathbb Z$ is a countable set; $C(\mathbb Z)\equiv1$.
		\begin{proof}
			$R:=\mathbb N_0$, $S:=\mathbb N_1$. $R$ and $S$ are countable axiomatically
			(axiom of extensionality), being the naturals themselves. $\mathbb Z\equiv
			R(1)\cup S(-1)$; this is countable via Theorem~\ref{thm:finite unions}.1.
		\end{proof}
	\end{subsection}

	\begin{subsection}{Countability of the Rationals}\label{subsec:applications.rationals}
		Claim: $\mathbb Q$ is a countable set; $C(\mathbb Q)\equiv1$.
		\begin{proof}
			$\mathbb Q$ has the same countability as $\mathbb Z\times \mathbb N_1$.
			$C(\mathbb Z)\equiv1$ via Section~\ref{subsec:applications.integers} and
			$C(\mathbb N_1)\equiv1$ axiomatically. Thus $C(\mathbb Z\times \mathbb N_1)\equiv1$
			via Theorem~\ref{thm:multiplications}.1. This and Theorem~\ref{thm:multiplications} derive
			from Cantor's enumeration of countable collections of countable sets.
		\end{proof}
	\end{subsection}

	% TODO: make this redirect to a better place for more information
	\pagebreak\begin{subsection}{Countability of the Reals from Zero to One}
	\label{subsec:applications.reals 0 to 1}
		Claim: $\{x\in\mathbb R:0\leqslant x<1\}$ is a countable set;
		$C(\mathbb R_{0\leqslant x<1})\equiv1$.
		\begin{proof}
			Let $R$ be a set where $\forall n\in\mathbb N_0$, $R_n := $ (``0." $+$ reversed digits
			of $n$), ie: $R_{2760}=0.0672$ (see Section~\ref{sec:algorithm mathematically} for formula). $R$
			is countable by definition, and contains every possible sequence of digits in $[0,1)$. The
			sequence trends towards different values with different sub-series;
			Section~\ref{subsec:applications.reals} explains this further. $R$ is used in the next
			proof.
		\end{proof}
	\end{subsection}

	\begin{subsection}{Countability of the Non-Negative Reals}
	\label{subsec:applications.non negative reals}
		Claim: $\mathbb R_{\geqslant0}$ or equivalently $\left\{x\in\mathbb R:x\geqslant0\right\}$,
		is a countable set; $C(\mathbb R_{\geqslant0})\equiv1$.
		\begin{proof}
			$C(\mathbb R_{0\leqslant x<1})\equiv1$ via Section~\ref{subsec:applications.reals 0 to 1}.
			$\mathbb R_{\geqslant0}=\displaystyle\bigcup_{n=1}^{\aleph_0}S_n$ where $S_n:=R[n-1]$. This
			is countable via Theorem~\ref{thm:infinite unions}; the following table illustrates this.
			$\forall x\in\mathbb R_{\geqslant0}$, there exists a sequence of
			$s_n\in S_{\lceil x\rceil}$, with increasing indices, where
			$\displaystyle\lim_{n\in\mathbb N\to\infty}s_n=x$. This is because $S_n$ contains every
			sequence of digits in $[n-1,n)$. $S$ is used in the next proof.
			\vspace{0.5em}\\
			\centerline{\begin{array}{|c|c|c|c|c|c|c|c|c|c|c|c|c|c|} % 13 columns
				\hline 0 & 0.1 & 0.2 & 0.3 & 0.4 & 0.5 & 0.6 & 0.7 & 0.8 & 0.9 & 0.01 & 0.11 & 0.21 &
				\cdots\\
				\hline 1 & 1.1 & 1.2 & 1.3 & 1.4 & 1.5 & 1.6 & 1.7 & 1.8 & 1.9 & 1.01 & 1.11 & 0.21 &
				\cdots\\
				\hline 2 & 2.1 & 2.2 & 2.3 & 2.4 & 2.5 & 2.6 & 2.7 & 2.8 & 2.9 & 2.01 & 2.11 & 0.21 &
				\cdots\\
				\hline 3 & 3.1 & 3.2 & 3.3 & 3.4 & 3.5 & 3.6 & 3.7 & 3.8 & 3.9 & 3.01 & 3.11 & 0.21 &
				\cdots\\
				\hline 4 & 4.1 & 4.2 & 4.3 & 4.4 & 4.5 & 4.6 & 4.7 & 4.8 & 4.9 & 4.01 & 4.11 & 0.21 &
				\cdots\\
				\hline \vdots & \vdots & \vdots & \vdots & \vdots & \vdots & \vdots & \vdots & \vdots &
				\vdots & \vdots & \vdots & \vdots & \ddots\\
				\hline
			\end{array}}
			
		\end{proof}
	\end{subsection}

	% TODO: make this redirect to a better place for more information
	\begin{subsection}{Countability of the Reals}\label{subsec:applications.reals}
		Claim: $\mathbb R$ is a countable set, which implies $2^{\aleph_0}=\aleph_0$.
		\begin{proof}
			$\mathbb R_{\geqslant0}$ is countable via
			Section~\ref{subsec:applications.non negative reals}.
			$\mathbb R_{\geqslant0}(1, -1)\equiv\mathbb R$. This is countable via
			Theorem~\ref{thm:rings}.1. This conclusion can be further reinforced by
			the algorithmic methods in Section~\ref{sec:algorithm code}.
		\end{proof}
	\end{subsection}

	\begin{subsection}{Countabilities of Miscellaneous Number Classes}
	\label{subsec:applications.miscellaneous}
		\begin{subsubsection}{Algebraic and Transcendental Numbers}
			\label{subsubsec:applications.misc.reals}
			Claim: $\mathbb A$ and $\mathbb T$ are countable sets over both $\mathbb R$ and $\mathbb C$.
			\begin{proof}
				$\mathbb R$ is a countable set via Section~\ref{subsec:applications.reals}.
				Since the algebraic reals and transcendental reals are both subsets of the
				reals, they are countable via Theorem~\ref{thm:subsets}.1. They are also
				countable over the complex numbers using Section~\ref{subsec:applications.complex}
				and the same logic.
			\end{proof}
		\end{subsubsection}

		\begin{subsubsection}{Imaginary Numbers}\label{subsubsec:applications.misc.imaginary}
			Claim: $\mathbb I$ is a countable set; $C(\mathbb I)\equiv1$.
			\begin{proof}
				$\mathbb R$ is a countable set via Section~\ref{subsec:applications.reals}.
				$\mathbb R(\sqrt{-1})\cup\emptyset(1)\equiv\mathbb I$. This is countable via
				Theorem~\ref{thm:finite unions}.1 becuase the union of any set and the null set
				is itself.
			\end{proof}
		\end{subsubsection}
	\end{subsection}

	\begin{subsection}{Countability of the Complex Numbers}\label{subsec:applications.complex}
		Claim: $\mathbb C$ is a countable set; $C(\mathbb C)\equiv1$.
		\begin{proof}
			$\mathbb R$ and $\mathbb I$ are countable sets via Sections \ref{subsec:applications.reals}
			and \ref{subsubsec:applications.misc.imaginary} respectively.
			$\mathbb R(1, \sqrt{-1})\equiv\mathbb C$
			is countable via Theorem~\ref{thm:rings}.1. $\mathbb R\times\mathbb I=\mathbb C$ is
			countable via Theorem~\ref{thm:multiplications}.1.
		\end{proof}
	\end{subsection}
\end{section}

\begin{section}{Generalized Continuum Hypothesis}\label{sec:GCH}
	Claim: GCH has the same truth value as CH, and powersets don't change sizes of infinity.
	\begin{proof}
		Let $S$ be a countably infinite set; it cannot be finite or this proof does not work.
		$P :=\mathcal P(S)$. Let $R_k$ denote the set of elements in $P$ with $k$ sub-elements
		($S^k$); these are countable via Theorem~\ref{thm:powers}. Observe that $P$ has
		$\aleph_0^0$ elements with 0 sub-elements, $\aleph_0^1$ elements with 1 sub-element, et
		cetera, and generally $\aleph_0^k$ elements with $k$ sub-elements, for
		$k\leqslant \aleph_0$. Imagine a $k$-dimensioned table of integers and listing them in
		a $k$-dimensional spiral, or recursively combining two table dimensions into one using
		the Theorem~\ref{thm:multiplications} table method until the table gets to one dimension.
		$\displaystyle P=\bigcup_{i=1}^{\aleph_0}R_{i-1}$ is countable via
		Theorem~\ref{thm:infinite unions}. This contradicts Cantor's Diagonal Argument (see
		Section~\ref{sec:diagonal argument}) and implies that uncountable sets cannot be created
		through powersets of countable sets, and either cannot be created at all, or need a much
		stronger method to create them.
	\end{proof}
\end{section}

\begin{section}{Cantor's Diagonal Argument}\label{sec:diagonal argument}
	According to Georg Cantor in 1891$^\text{[ref \ref{ref:diagonal argument}, \ref{ref:cantor-1891}]}$,
	If someone is trying to list all the real numbers, they can always find a number that is not in the
	list using his ``Diagonal Argument" which accomplishes the same as the following, though for reals
	instead of naturals: suppose someone is trying to make a set $S$ with every natural number
	($\mathbb N_1$). They could first skip 1 and add 2 to $S$. Then they add 3, then 4, 5, 6, et cetera
	forever. No matter how many natural numbers they add, they can always find one not in it;
	$\max(S)+1$. This seems to be implying that there is not a bijection between the natural numbers and
	themselves since in the limit towards infinity, they've used up all infinity natural numbers
	indexing $S$ and yet not all of them are in $S$. Well of course, because they \emph{skipped} 1. Due
	to this possibility of skipping elements, the diagonal argument cannot always accurately depict the
	countability of sets, and it can just as easily skip elements in the set of real numbers.
\end{section}

\begin{section}{Enumerating the Continuum Algorithmically}\label{sec:algorithm code}
	The following JavaScript code prints out real numbers to stdout, delimited by a comma-space pair,
	up to the maximum allowed big integer index. It prints out both $\pm0$, though this is not really a
	problem and is easily resolvable. The more verbatim version as well as the C version of the source
	code can be found at References~\ref{ref:js-code}~and~\ref{ref:c-code}. The complexity of finding
	the $n^\text{th}$ real number with this method is around $\Theta(\log n)$, assuming basic operations
	such as bit shifts, comparisons, addition, and multiplication run in $\mathcal O(1)$. isqrt($n$) is
	less than $\mathcal O(\log_2 n)$, and contains most of the complexity of the formula. The code
	didn't start to slow down noticably on my machine until around $n=10^{100}$, and at $n=10^{10,000}$
	only took 2.4 seconds each (rather than 0.2ms at the start).

	\begin{verbatim}
		│const write = globalThis.toString().slice(8, -1).toLowerCase() === "global" ?
		│    s => process.stdout.write.call(process.stdout, s) : // NodeJS
		│    console.log; // Browser
		│
		│function isqrt(n) { // floored square root
		│    if (n < 2n) return n;
		│
		│    var cur = n >> 1n, prev; // current, previous
		│
		│    do [cur, prev] = [cur + n / cur >> 1n, cur];
		│    while (cur < prev);
		│
		│    return prev;
		│}
		│
		│for (var i = 0n ;;) {
		│    const k = 1n + isqrt(1n + (i << 2n)) >> 1n // T(k-1) < i/2 <= T(k)
		│        , t = i + k*(1n - k) >> 1n; // i/2 - T(k-1)
		│
		│    write( // coordinates: (k, k-t-1)
		│        `${i++ % 2n ? "-" : ""}${t}.` + // integer part
		│        `${k - t - 1n}`.split("").reverse().join("") // fractional part
		│    )
		│}
	\end{verbatim}
\end{section}

\begin{section}{The Algorithm Mathematically}\label{sec:algorithm mathematically}
	The axioms of Zermelo-Fraenkel Set theory required to prove the algorithmic viability of
	continuum enumeration are as follows: Axiom Schema of Specification (for creating Cartesian
	products), Axiom of Union, and Axiom of Infinity. The previous algorithm uses string
	manipulation to reverse integers and find their lengths, but the following formulas for length()
	and reverse() do the same thing mathematically. They are used to find the $n$th real number.\\
	\begin{align}
		\text{length}_b\,x & := \max\!\left(0,1+\lfloor\log_b|x|\rfloor\right)\ni\ln0\leqslant0\\
		\text{reverse}_b\,x & :=\text{sgn}(x)\!\!\sum_{m=0}^{\text{length}_b\,x}\left(\left\lfloor
		\frac{|x|}{b^m}\right\rfloor\text{mod}\,b\right)\!b^{\text{length}_b\,x-m-1}\\
		R(i,j) & := i+\frac{\text{reverse}_{10}\,j}{10^{\text{length}_{10}\,j}}
	\end{align}
	Where $i$ is the integer part, $j$ is the reversed decimal part, and $b$ is the base. $i$ and $j$
	can be swapped for a different enumeration (and inverse) (see Reference~\ref{ref:desmos real-enum}).
	$t$ is the greatest triangle number $T(k)\leqslant n$, where $n,k\in\mathbb N_0$.\\
	\begin{align}
		T(x) &:= \frac{x^2+x}2 = \sum_{j=1}^xj\\
		n &:= \left\lfloor\dfrac N2\right\rfloor\\
		k &:= \left\lceil\frac{\left\lfloor\sqrt{8n+1}\right\rfloor}2\right\rceil-1 =
		\left\lfloor\dfrac{\sqrt{8n+1}-1}2\right\rfloor\\
		t &:= T(k)\\
		u &:= n - t\\
		\mathbb R_N&\,\, = (2[(N+1)\,\text{mod}\,2]-1)R(u, k-u)
	\end{align}
	For each integer $k$, This cycles through the ordered pairs of positive $(i, j)$ where $i + j = k$
	and it makes a unique real number out of each pair. Because of the bijection it creates, there is
	also an inverse of the (Section~\ref{sec:algorithm code}) algorithm. The simplicity and
	near-symmetry of the inverse is striking due to the complexity of the original function.
	\begin{verbatim}
		│function inverse(string) {
		│    const [ipart, fpart] = string.replace("-", "").split(".")
		│        , x = BigInt( ipart )
		│        , y = BigInt( fpart.split("").reverse().join("") );
		│
		│    return (x+y)**2n + 3n*x + y + BigInt(string[0] === "-");
		│}
	\end{verbatim}
	The following is the same formula in math form. $\infty$ can be replaced with the max precision.
	\begin{align}
		x(t) & := \lfloor|t|\rfloor\\
		y(t) & := \text{reverse}_{10}\left((|t|\,\text{mod}\,1)\,10^{\sum\limits_{n=0}^\infty\text{sgn}(10^n|t|\,\text{mod}\,1)}\right)10^{\sum\limits_{n=1}^\infty(\,1-\text{sgn}\lfloor10^n(|t|\,\text{mod}\,1)\rfloor\,)}\\
		\text{inverse}(t) & := \left[x(t) + y(t)\right]^2 + 3\,x(t) + y(t) + 1 - \text{sgn}(1+\text{sgn}\,t)
	\end{align}
\end{section}

% TODO: start updating here
\begin{section}{Further Evidence}\label{sec:further evidence}
	TODO: finish this.
	\begin{itemize}
		\item $\displaystyle\lim_{n\to\infty}x_n=x\Longrightarrow\lim_{n\to\infty}\text{inverse}(x_n)=N$
		(some 10-adic integer index unique to $x$).
		\item $x_n$ approaches $x$ at the same rate that inverse($x_n$) approaches $N$. So for each extra
		(decimal) digit that $x_n$ gains, inverse($x_n$) also gains one (integer) digit. It can gain more
		if the digits after it are zero, but never less.
		\item $\displaystyle
			\text{inverse}(10^n) = \begin{cases}
				100^n + 3 \cdot 10^n,~n \ge 0\\
				\dfrac1{100^{n+1}} + \dfrac1{10^{n+1}},~n < 0
			\end{cases}
		$
		\item $\text{inverse}(b^n)=b^{2n}+3\cdot b^n\forall n\ge0\land b\ge2$
	\end{itemize}
\end{section}

\begin{section}{Conclusion}\label{sec:conclusion}
	Since there exists a bijection between any infinite set and its powerset (GCH), there is no
	set with a cardinality strictly or loosely in between the naturals and reals (CH), because
	they are the same cardinality. This also implies that $\aleph_n=\beth_n=\beth_0$ for all
	natural numbers $n$, which makes sense intuitively since $2^\infty=\infty$. Also if any
	sequence of natural numbers is concatenated, it will always create a new natural number,
	meaning every element in $\mathcal P(\mathbb N_0)$ corresponds to an element in
	$\mathbb N_0$. This all could have disasterous consequences for set theory because CH is
	provably undecidable in some models and yet provably decidable. This means either
	Zermelo-Fraenkel Set Theory with the axiom of choice (ZHC) is unsound, the aforementioned
	models are invalid, or all of the decidability proofs are invalid. The entire foundation of
	aleph numbers, beth numbers, and sizes of infinities could be entirely flawed. CH implies
	the Gimel Hypothesis is true according to Reference~\ref{ref:gimel}, and implies Wetzel's
	problem is false according to Reference~\ref{ref:wetzel}. A truth value for CH or GCH is
	not asserted here bacause the specifics of CH are unclear, and they are important for its
	truth value. If the reals have to be bigger than the naturals, it is false, but if it just
	requires that there is no set in between them, it is true.
	% Liouville numbers are also countable because of this
\end{section}

\pagebreak\begin{section}{References}\label{sec:references}
	\begin{enumerate}
		\item\url{https://en.wikipedia.org/wiki/Cantor's\_diagonal\_argument}\\
		\label{ref:diagonal argument}
		\text{Wikipedia page with background for Section~\ref{sec:diagonal argument}}

		\item\url{https://www.digizeitschriften.de/dms/img/?PID=GDZPPN002113910&physid=phys84#navi}\\
		\label{ref:cantor-1891}
		\text{Georg Cantor's 1891 article with the diagonal argument. Same source as on wikipedia.}

		\item\url{https://en.wikipedia.org/wiki/Continuum\_hypothesis}\\
		\label{ref:continuum}
		\text{Wikipedia page with elaboration on Section~\ref{subsec:applications.reals}}

		\item\url{https://www.github.com/drizzt536/files/tree/main/TeX/continuum}\\
		\label{ref:files}
		\text{The files for the most recent public version of this pdf and the \LaTeX~code}

		\item\url{https://raw.githubusercontent.com/drizzt536/files/main/JavaScript/continuum.js}\\
		\label{ref:js-code}
		\text{The raw JavaScript source code for the Section~\ref{sec:algorithm code}}

		\item\url{https://raw.githubusercontent.com/drizzt536/files/main/C/continuum.c}\\
		\label{ref:c-code}
		\text{The raw C source code for the Section~\ref{sec:algorithm code}}

		\item\url{https://www.desmos.com/calculator/2dk8rm31ds}\\
		\label{ref:desmos real-enum}
		\text{Desmos with real number enumeration 1 formulas for Section~\ref{sec:algorithm mathematically}}

		\item
		\url{https://github.com/drizzt536/files/blob/main/TeX/continuum/real-enum-formula/formula.pdf}\\
		\label{ref:pdf real-enum}
		\text{pdf with an almost fully-expanded real number enumeration 1 formula for
			Section~\ref{sec:algorithm mathematically}}

		\item\url{https://en.wikipedia.org/wiki/Gimel\_function\#The\_gimel\_hypothesis}\\
		\label{ref:gimel}
		\text{Gimel Hypothesis Wikipedia page}

		\item\url{https://en.wikipedia.org/wiki/Wetzel's\_problem}\\
		\label{ref:wetzel}
		\text{Wetzel's Problem Wikipedia page}
	\end{enumerate}\\
	Editors:
	\begin{itemize}
		\item Daniel E. Janusch
		\item Valerie Janusch, my mom
	\end{itemize}
	\vspace{6em}\\
	This document is licensed under https://raw.githubusercontent.com/drizzt536/files/main/LICENSE
	and may be copied ONLY IN ITS ENTIRETY under penalty of law.
\end{section}

\end{document}
