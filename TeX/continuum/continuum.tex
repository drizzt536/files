% file for pdflatex.exe

% countability.tex v1 (c) | Copyright 2023 Daniel E. Janusch

% this file is licensed by https://raw.githubusercontent.com/drizzt536/files/main/LICENSE
% and must be copied IN ITS ENTIRETY under penalty of law.

\documentclass[12pt]{article}
\usepackage{amssymb, latexsym, amsmath, amsthm}
\usepackage[margin=1in]{geometry}

\addtolength{\topmargin}{-0.5in}
\renewcommand{\qedsymbol}{\blacksquare}
\newtheorem{thm}{Theorem}

\begin{document}


\title{On the Continuum Hypothesis and the Correspondence of Infinite Sets With the Natural Numbers}
\author{Daniel E. Janusch\\Dedicated to Maeby, my favorite kitty}
\maketitle

\begin{section}{Background and Theorems}\label{sec:background}
	If a set is ``countable",  each element can be matched one to one with the natural numbers. Any
	set that is finite will have this property, and the only question is if all infinite sets have it.
	$|S|$ denotes the cardinality of $S$ for any set $S$. Iterated set multiplication gives ordered
	groups with more than two elements, rather than ordered pairs of ordered pairs.\\
	\begin{align}
		R\times S & :=\{(r,s):r\in R\land s\in S\}\\
		R(\vec v\,):=\left\{\sum_{n=1}^{\dim\vec v}a_n\vec v_n:a_n\in R\right\}~~~(2 & \text a)
		\hspace{2.5em}
		R[\vec v\,]:=\left\{\prod_{n=1}^{\dim\vec v}(a_n+\vec v_n):a_n\in R\right\}~~~(2\text b)
	\end{align}

	\begin{thm}\label{thm:finite unions}
		\emph{Unions of Countable Sets:}\\
		\indent\emph{If any sets $R$ and $S$ are countable, $R(a)\cup S(b)$ is countable for all $a$ and $b$.}
	\end{thm}\begin{proof}
		Since $R$ and $S$ are countable, $R(a)$ and $S(b)$ are as well, because this just multiplies
		the already present elements by a constant. Since $R(a)$ and $S(b)$ are countable individually,
		just interlace the elements of $R(a)$ with the elements of $S(b)$. If any elements are duplicates,
		just skip them. If either of the sets runs out, meaning it is finite, just stop interlacing, and
		only pull from one set. Similar logic works using $R[a]$ and $S[b]$.
	\end{proof}

	\begin{thm}\label{thm:multiplications}
		\emph{Multiplications of Countable Sets:}\\
		\indent\emph{If any sets $R$ and $S$ are countable, $R\times S$ is countable.}
	\end{thm}\begin{proof}
		Put ordered pairs using elements all the of $R$ and $S$ into a table as shown below. Next,
		go along the diagonals, as shown. This will get every element in the product and not ``miss"
		any. The numbers and colons are the indexes. The first 10 are shown.\vspace{0.5em}\\
		\centerline{\begin{array}{|c|c|c|c|c|}
			\hline 1:(R_1, S_1) & 2:(R_1, S_2) & 6:(R_1, S_3) & 7:(R_1, S_4) & \cdots\\
			\hline 3:(R_2, S_1) & 5:(R_2, S_2) & 8:(R_2, S_3) & (R_2, S_4) & \cdots\\
			\hline 4:(R_3, S_1) & 9:(R_3, S_2) & (R_3, S_3) & (R_3, S_4) & \cdots\\
			\hline 10:(R_4, S_1) & (R_4, S_2) & (R_4, S_3) & (R_4, S_4) & \cdots\\
			\hline \vdots & \vdots & \vdots & \vdots & \ddots\\
			\hline
		\end{array}}\vspace{-0.6em}\\
	\end{proof}

	\pagebreak\\
	\begin{thm}\label{thm:subsets}
		\emph{Subsets of Countable Sets:}\\
		\indent\emph{If any set $R$ is countable, and a set $S\subseteq R$, $S$ is also countable.
		(loose subset)}
	\end{thm}\begin{proof}
		Removing elements from a set won't make it uncountable, so just remove elements from $R$
		until it becomes $S$. Since every set along the way is countable, so is $S$. This also
		implies that intersections and set differences are countable because these operations
		return subsets of the original sets.
	\end{proof}

	\begin{thm}\label{thm:rings}
		\emph{Sum and Product Rings From Countable Sets:}\\
		\indent\emph{If any set $R$ is countable, $R(a, b)$ and $R[a, b]$ are also countable
		for all $a$ and $b$.}
	\end{thm}\begin{proof}
		Create 2 new intermediate sets $S_a := \{ar:r\in R\}$ and $S_b := \{br:r\in R\}$.
		$S_a\times S_b$ gives a set with the same countability as $R(a,b)$, just with
		ordered pairs instead of addition. This is countable via Theorem~\ref{thm:multiplications}.
		Similar logic works for $R[a,b]$, swapping addition and multiplication.
	\end{proof}

	\begin{thm}\label{thm:powers}
		\emph{Powers of Countable Sets:}\\
		\indent\emph{If any set $R$ is countable, $R^n$ is also countable for all natural numbers $n$.}
	\end{thm}\begin{proof}
		$R^n$ can be factored as $R\times R^{n-1}$ which is countable if $R^{n-1}$ is countable.
		$R^{n-1}$ can be factored as $R\times R^{n-2}$ which is countable if $R^{n-2}$ is countable,
		et cetera. This can be simplified continuously until it becomes $R^2$ or $R\times R$, which
		is countable via Theorem~\ref{thm:multiplications}. If $n=0$, the output set is all the groups
		of zero elements from $R$, or just $\O$, which is countable, being finite.
	\end{proof}

	\begin{thm}\label{thm:infinite unions}
		\emph{Infinite Unions of Countable Sets:}\\
		\indent\emph{If all sets $R_i$ are countable, $\displaystyle\bigcup_{i=1}^{\aleph_0} R_i$ is also
		countable, where $\aleph_0\equiv\left|\mathbb N\right|=\infty$.}
	\end{thm}\begin{proof}
		Putting all the elements into a table as shown below, and using the same argument as in
		Theorem~\ref{thm:multiplications}, one can show that every element is indexed and none are
		skipped. Every column represents a different $R_i$, and the row represents a different
		$\left(R_i\right)_j$ or $R_{i,j}$. The number and colon is the index. The first 35 are
		shown. 29 is one to the right of 28.\vspace{1em}\\
		\centerline{\begin{array}{|c|c|c|c|c|c|c|c|} % 8 columns
			\hline 1:R_{1,1} & 2:R_{2,1} & 6:R_{3,1} & 7:R_{4,1} & 15:R_{5,1} & 16:R_{6,1} & 28:R_{7,1}& \cdots\\
			\hline 3:R_{1,2} & 5:R_{2,2} & 8:R_{3,2} & 14:R_{4,2} & 17:R_{5,2} & 27:R_{6,2}& 30:R_{7,2}& \cdots\\
			\hline 4:R_{1,3} & 9:R_{2,3} & 13:R_{3,3} & 18:R_{4,3} & 26:R_{5,3} & 31:R_{6,3} & R_{7,3} & \cdots\\
			\hline 10:R_{1,4} & 12:R_{2,4} & 19:R_{3,4} & 25:R_{4,4} & 32:R_{5,4} & R_{6,4} & R_{7,4} & \cdots\\
			\hline 11:R_{1,5} & 20:R_{2,5} & 24:R_{3,5} & 33:R_{4,5} & R_{5,5} & R_{6,5} & R_{7,5} & \cdots\\
			\hline 21:R_{1,6} & 23:R_{2,6} & 34:R_{3,6} & R_{4,6} & R_{5,6} & R_{6,6} & R_{7,6} & \cdots\\
			\hline 22:R_{1,7} & 35:R_{2,7} & R_{3,7} & R_{4,7} & R_{5,7} & R_{6,7} & R_{7,7} & \cdots\\
			\hline \vdots & \vdots & \vdots & \vdots & \vdots & \vdots & \vdots & \ddots\\
			\hline
		\end{array}}
	\end{proof}
\end{section}

\pagebreak\begin{section}{Applications of the Theorems}\label{sec:applications}
	\begin{subsection}{Countability of The Integers}\label{subsec:applications.integers}
		Claim: $\mathbb Z$ is a countable set.
		\begin{proof}
			Let $R=\mathbb N_0$, $S=\mathbb N_1$. $R$ and $S$ are countable because they
			\emph{are} the naturals.\\
			$\mathbb Z\equiv R\cup S(-1)$. This is countable via Theorem~\ref{thm:finite unions}.
		\end{proof}
	\end{subsection}

	\begin{subsection}{Countability of the Rationals}\label{subsec:applications.rationals}
		Claim: $\mathbb Q$ is a countable set.
		\begin{proof}
			The rationals are really just ordered pairs of integers and naturals.\\
			Let $R=\mathbb Z$, $S=\mathbb N_1$. $R\times S$ defines all of these pairs.
			$R$ is countable via Subsection~\ref{subsec:applications.integers},
			and $S$ is countable because it \emph{is} the naturals. $R\times S$ is
			thus countable via Theorem~\ref{thm:multiplications}.
		\end{proof}
	\end{subsection}

	\begin{subsection}{Countability of the Reals from Zero to One}\label{subsec:applications.reals 0 to 1}
		Claim: $\left\{x:x\in\mathbb R\land0\le x<1\right\}$ is a countable set.
		\begin{proof}
			Let $R$ be some set. Each $R_i$ can be defined to be the digits of $i$
			reversed with ``0." and the beginning and infinite trailing zeros at the
			end, for any $i\in\mathbb N_0$. For example, $R_{246}=0.642\overline0$, and
			$R_0=0.0\overline0=0$. This set is countable because it was defined to be
			countable; each natural number corresponds to a single element in it. This
			set contains every real number in the range because every possible sequence
			of digits is in it. The sequence trends upwards, asymptoting at 1, though
			it fluctuates wildly along the way.
		\end{proof}
	\end{subsection}

	\begin{subsection}{Countability of the Non-Negative Reals}\label{subsec:applications.non negative reals}
		Claim: $\mathbb R_{\ge0}$ or $\left\{x:x\in\mathbb R\land x\ge0\right\}$ is a countable set.
		\begin{proof}
			The set of real numbers from zero to one is countable via
			Subsection~\ref{subsec:applications.reals 0 to 1}. Let $R$ be the same set used there.
			The desired set is $\displaystyle\bigcup_{i=1}^{\aleph_0}S_i$ where $S_i=R[i-1]$. this
			is countable via Theorem~\ref{thm:infinite unions}. The following table shows this.
			\vspace{0.3em}\\
			\vspace{-0.9em}\centerline{\begin{array}{|c|c|c|c|c|c|c|} % 7 columns
				\hline 0.\overline0 & 1.\overline0 & 2.\overline0 & 3.\overline0 & 4.\overline0 & 5.\overline0 & \cdots\\
				\hline 0.1\overline0& 1.1\overline0& 2.1\overline0& 3.1\overline0& 4.1\overline0& 5.1\overline0& \cdots\\
				\hline 0.2\overline0& 1.2\overline0& 2.2\overline0& 3.2\overline0& 4.2\overline0& 5.2\overline0& \cdots\\
				\hline 0.3\overline0& 1.3\overline0& 2.3\overline0& 3.3\overline0& 4.3\overline0& 5.3\overline0& \cdots\\
				\hline \vdots & \vdots & \vdots & \vdots & \vdots & \vdots & \cdots\\
				\hline 0.7\overline0& 1.7\overline0& 2.7\overline0& 3.7\overline0& 4.7\overline0& 5.7\overline0& \cdots\\
				\hline 0.8\overline0& 1.8\overline0& 2.8\overline0& 3.8\overline0& 4.8\overline0& 5.8\overline0& \cdots\\
				\hline 0.9\overline0& 1.9\overline0& 2.9\overline0& 3.9\overline0& 4.9\overline0& 5.9\overline0& \cdots\\
				\hline 0.01\overline0&1.01\overline0&2.01\overline0&3.01\overline0&4.01\overline0&5.01\overline0&\cdots\\
				\hline \vdots & \vdots & \vdots & \vdots & \vdots & \vdots & \ddots\\
				\hline
			\end{array}}
		\end{proof}
	\end{subsection}

	\pagebreak\begin{subsection}{Countability of the Reals}\label{subsec:applications.reals}
		Claim: $\mathbb R$ is a countable set and $\left|\mathcal P(\mathbb N)\right|\equiv\left|\mathbb N\right|$.
		\begin{proof}
			$\mathbb R_{\ge 0}$ or the non-negative real numbers are countable via
			Subsection~\ref{subsec:applications.non negative reals}.\\
			$\mathbb R_{\ge0}(1, -1)\equiv\mathbb R$. This is countable via Theorem~\ref{thm:rings}
			since $\mathbb R_{\ge 0}$ is countable. $\aleph_1=2^{\aleph_0}=\aleph_0$.
		\end{proof}
	\end{subsection}

	\begin{subsection}{Miscellaneous Number Classes}\label{subsec:applications.miscellaneous}
		\begin{subsubsection}{Algebraic Real and Transcendental Real Numbers}
			\label{subsubsec:applications.misc.reals}
			Claim: $\mathbb A$ and $\mathbb T$ are countable sets.
			\begin{proof}
				$\mathbb R$ is a countable set via Subsection~\ref{subsec:applications.reals}.
				Since the algebraic real numbers and transcendental real numbers are both
				subsets of the real numbers, they are countable via Theorem~\ref{thm:subsets}.
			\end{proof}
		\end{subsubsection}

		\begin{subsubsection}{Imaginary Numbers}\label{subsubsec:applications.misc.imaginary}
			Claim: $\mathbb I$ is a countable set.
			\begin{proof}
				$\mathbb R$ is a countable set via Subsection~\ref{subsec:applications.reals}.
				$\mathbb R(\sqrt{-1})\equiv\mathbb I$. This is countable via Theorem~\ref{thm:finite unions},
				or more precisely $R(\sqrt{-1})\cup\O$ is. Any set unioned with the null set is itself.
			\end{proof}
		\end{subsubsection}
	\end{subsection}

	\begin{subsection}{Countability of the Complex Numbers}\label{subsec:applications.complex}
		Claim: $\mathbb C$ is a countable set.
		\begin{proof}
			$\mathbb R$ is a countable set via Subsection~\ref{subsec:applications.reals}.
			$\mathbb R(1, \sqrt{-1})\equiv\mathbb C$. This is countable via Theorem~\ref{thm:rings}.
			The identity used here stems from the rectangular form of complex numbers.
		\end{proof}
	\end{subsection}
\end{section}

\begin{section}{Cantor's Diagonal Argument}\label{sec:diagonal argument}
	According to Georg Cantor in 1891, If someone is trying to list all the real numbers,
	they can always find a number that is not in the list using his ``Diagonal Argument".
	This argument is basically the same as the following: Suppose someone is trying to make
	a set of every natural number. They first add zero to the set and the set is $\{0\}$.
	Then they could say, well one isn't in the set. When they add one and have $\{0,1\}$,
	they can say two isn't in the set, then three isn't in the set, four isn't in the set,
	et cetera. No matter how many natural numbers they add, they can always find one not in
	it. Using Cantor's same logic, this seems to be saying that the naturals cannot be
	corresponded one to one with the naturals, which is clearly wrong, the naturals
	\emph{are} the naturals.
\end{section}


\pagebreak\begin{section}{References}\label{sec:references}

	\noindent- \url{https://en.wikipedia.org//wiki/Cantor's\_diagonal\_argument}\\
	\indent\text{wikipedia page for elaboration on Section~\ref{sec:diagonal argument}}\\

	\noindent- \url{https://en.wikipedia.org/wiki/Continuum_hypothesis}\\
	\indent\text{wikipedia page for elaboration on Subsection~\ref{subsec:applications.reals}}\\

	\noindent- \url{https://www.github.com/drizzt536/files/tree/main/TeX/continuum}\\
	\indent\text{The files for the most recent version of this pdf and the \LaTeX code}
	\\
	\\
	\\
	\\
	This document is licensed under https://github.com/drizzt536/files/blob/main/LICENSE
	and must be copied IN ITS ENTIRETY under penalty of law.
\end{section}
\end{document}
