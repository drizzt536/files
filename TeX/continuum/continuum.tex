% file for pdflatex.exe

% continuum.tex v4.0 (c) | Copyright 2023 Daniel E. Janusch

% This file is licensed by https://raw.githubusercontent.com/drizzt536/files/main/LICENSE
% and may only be copied IN ITS ENTIRETY under penalty of law.

\documentclass[12pt]{article}
\usepackage{amssymb, latexsym, amsmath, amsthm}
\usepackage[margin=1in]{geometry}

\addtolength{\topmargin}{-0.5in}
\newtheorem{thm}{Theorem}

\renewcommand{\qedsymbol}{\blacksquare}

\begin{document}


\title{On the Continuum Hypotheses and the Correspondence of Infinite Sets with the Natural Numbers}
\author{Daniel E. Janusch\\Dedicated to Maeby, my favorite kitty}
\maketitle

\begin{section}{Introduction and Countability Theorems}\label{sec:background}
	Any ``countable" set is equinumerous to a loose subset of the natural numbers, and
	there exists a bijection between them. Any finite set has this property, the main
	question being if all \emph{infinite} sets do. In all infinite processes, one should
	think of the steps as executing simultaneously rather than separately and successively.
	$|S|$ and $C(S)$ denote the cardinality and countability of a set $S$ respectively.
	Equations (2a) and (2b) describe what I will call ``sum ring operators," ``product
	ring operators," and collectively ``ring operators." Most Theorems use ``$\iff$"
	instead of ``$\implies$" because Theorem~\ref{thm:subsets}.1 works as an inversion.
	\vspace{-1em}\\
	\begin{align}
		R\times S & :=\{(r,s):r\in R\land s\in S\}~~~\,(1\text a)
		\hspace{2em}
		R\cdot S:=\{r\cdot s:r\in R\land s\in S\}~~~~~~\,(1\text b)
		\\
		R(\vec v\,) & :=\left\{\sum_{n=1}^{\dim\vec v}(a_n\cdot\vec v_n):a_n\in R\right\}~(2\text a)
		\hspace{2em}
		R[\,\vec v\,]:=\left\{\prod_{n=1}^{\dim\vec v}(a_n+\vec v_n):a_n\in R\right\}~(2\text b)\\
		\aleph_0 & := |\mathbb N|\\
		C(S) & :=\begin{cases}
			1,\,\left[S\text{ countable}\right]\phantom{un}~~(|S|\leqslant\aleph_0)\\
			0,\,\left[S\text{ uncountable}\right]~~(|S|>\aleph_0)\\
		\end{cases}
	\end{align}

	\begin{thm}\label{thm:finite unions}
		\emph{
			Singlular Unions of Countable Sets\\
			\indent If any sets $R$ and $S$ are countable, $R(a)\cup S(b)$ is countable
			for all $a$ and $b$.\vspace{0.4em}\\
			\indent1. $C(R)\land C(S)\iff C(R\left(a\right)\cup S\left(b\right))~\forall a,b$
			\hspace{1em}
			2. $C(R)\land C(S)\iff C(R\left(a\right)\cup S[\,b\,])~\forall a,b$\\
			\indent3. $C(R)\land C(S)\iff C(R[\,a\,]\cup S\left(b\right))~\forall a,b$
			\hspace{1em}
			4. $C(R)\land C(S)\iff C(R[\,a\,]\cup S[\,b\,])~\forall a,b$
		}
	\end{thm}\begin{proof}
		$R$ and $S$ being countable implies $R(a)$ and $S(b)$ are countable as well
		because ring operators keep cardinalities the same. Interlace the elements of
		$R(a)$ with those of $S(b)$, skipping over any duplicates. If either set is
		finite, append its elements to the beginning of the other set sans duplicates.
		Similar logic works using $R[\,a\,]$ or $S[\,b\,]$.
	\end{proof}

	\begin{thm}\label{thm:multiplications}
		\emph{
			Multiplications and Cartesian Products of Countable Sets\\
			\indent If any sets $R$ and $S$ are countable, both $R\times S$ and
			$R\cdot S$ are countable.\vspace{0.4em}\\
			\indent1. $C(R)\land C(S)\iff C(R\times S)$\\
			\indent2. $C(R)\land C(S)\iff C(R\cdot S)$
		}
	\end{thm}\pagebreak\begin{proof}
		To create $R\times S$, put all possible ordered pairs into a table as shown.
		One can go along the diagonals, including every element in the output, and
		not ``missing" any, implying a bijection between $R\times S$ and $\mathbb N$.
		The numbers and colons in the table are the output indices, the first 25 being labeled if in
		view. Similar logic works for $R\cdot S$. This method is also algorithmically
		viable (see Section~\ref{sec:algorithm code}). The output indices relate
		strongly to triangle numbers.\vspace{-0.3em}\\
		\centerline{\begin{array}{|c|c|c|c|c|}
			\hline \phantom{0}1:(R_1, S_1) & \phantom{0}2:(R_1, S_2) & \phantom{0}4:(R_1, S_3) & \phantom{0}7:(R_1, S_4) & \cdots\\
			\hline \phantom{0}3:(R_2, S_1) & \phantom{0}5:(R_2, S_2) & \phantom{0}8:(R_2, S_3) & 12:(R_2, S_4) & \cdots\\
			\hline \phantom{0}6:(R_3, S_1) & \phantom{0}9:(R_3, S_2) & 13:(R_3, S_3) & 18:(R_3, S_4) & \cdots\\
			\hline 10:(R_4, S_1) & 14:(R_4, S_2) & 19:(R_4, S_3) & 25:(R_4, S_4) & \cdots\\
			\hline \vdots & \vdots & \vdots & \vdots & \ddots\\
			\hline
		\end{array}}\vspace{-1.1em}
	\end{proof}

	\begin{thm}\label{thm:subsets}
		\emph{
			Subsets, Intersections, and Differences of Countable Sets\\
			\indent If any set $R$ is countable, any loose subset $S$ of $R$ is also countable.
			\vspace{0.4em}\\
			\indent1. $C(R)\land S\subseteq R\implies C(S)$\\
			\indent2. $C(R)\land C(S)\implies C(R\cap S)$\\
			\indent3. $C(R)\land C(S)\implies C(R\smallsetminus S)$
		}
	\end{thm}\begin{proof}
		To create the subset, take each element in $R$ that is not in $S$ ($x\in R\smallsetminus S$),
		move its index to the beginning of $R$, and remove it. This way, the new $R$ has countability
		guaranteed. With applying this recursively, $S$ must be countable. This also implies that set
		intersections and differences are countable because these operations return subsets of the
		inputs.
	\end{proof}

	\begin{thm}\label{thm:rings}
		\emph{
			Two Argument Ring Operators On Countable Sets\\
			\indent If any set $R$ is countable, $R(a, b)$ and $R[a, b]$ are also countable
			for all $a$ and $b$.\vspace{0.4em}\\
			\indent1. $C(R)\iff C(R(a, b))~\forall a,b$\\
			\indent2. $C(R)\iff C(R[\,a, b\,])~\forall a,b$
		}
	\end{thm}\begin{proof}
		Create 2 new intermediate sets $S_a := \{a\cdot r:r\in R\}$ and $S_b := \{b\cdot r:r\in R\}$.
		$S:=S_a\times S_b$ gives a set with the same cardinality as $R(a,b)$, only with
		ordered pairs instead of addition. $S$ is countable via Theorem~\ref{thm:multiplications}.1.
		Similar logic works for $R[a,b]$.
	\end{proof}

	\begin{thm}\label{thm:powers}
		\emph{
			Finite Natural Powers of Countable Sets\\
			\indent If any set $R$ is countable, $R^n$ is also countable for all natural
			numbers $n$.\vspace{0.4em}\\
			\indent1. $C(R)\iff C(R^n)~~\forall n\in\mathbb N_0<\infty$
		}
	\end{thm}\begin{proof}
		$R^n$ can be factored as $R\times R^{n-1}$ which is countable if $R^{n-1}$ is countable. This can be applied recursively until it becomes countable if $R^2$ or $R\times R$ is
		countable, which is countable via Theorem~\ref{thm:multiplications}.1. If $n=0$, the
		output set is all the groups of zero elements from $R$, or just $\emptyset$, which is countable
		since $\left|\emptyset\right|$ is finite.
	\end{proof}

	\begin{thm}\label{thm:infinite unions}
		\emph{
			Countably Infinite Unions of Countably Infinite Sets\\
			\indent If all sets $R_i$ are countable, their union is also countable.\vspace{0.4em}\\
			\indent1. $C(R_i)\forall i\iff C\!\left(\displaystyle\bigcup_{i=1}^{\aleph_0}R_i\right)$
		}
	\end{thm}\begin{proof}
		To create the union, give each set $R_i$ a column in a table and put all its elements
		sequentially in that column going down. Do this for every $R_i$ and using the same
		argument as in Theorem~\ref{thm:multiplications}, one can show that this set bijects
		the naturals. Every column represents a different $R_i$ and the rows represent a
		different $\left(R_i\right)_j$ or $R_{i,j}$. The numbers before the colons are the
		output index, the first 85 shown if in view. This is algorithmically viable
		(see Theorem~\ref{thm:multiplications}).\vspace{0.5em}
		\centerline{\begin{array}{|c|c|c|c|c|c|c|c|} % 8 columns
			\hline \phantom{0}1:R_{1,1} & \phantom{0}2:R_{2,1} & \phantom{0}6:R_{3,1} & \phantom{0}7:R_{4,1} & 15:R_{5,1} & 16:R_{6,1} & 28:R_{7,1} & \cdots\\
			\hline \phantom{0}3:R_{1,2} & \phantom{0}5:R_{2,2} & \phantom{0}8:R_{3,2} & 14:R_{4,2} & 17:R_{5,2} & 27:R_{6,2} & 30:R_{7,2} & \cdots\\
			\hline \phantom{0}4:R_{1,3} & \phantom{0}9:R_{2,3} & 13:R_{3,3} & 18:R_{4,3} & 26:R_{5,3} & 31:R_{6,3} & 43:R_{7,3} & \cdots\\
			\hline 10:R_{1,4} & 12:R_{2,4} & 19:R_{3,4} & 25:R_{4,4} & 32:R_{5,4} & 42:R_{6,4} & 49:R_{7,4} & \cdots\\
			\hline 11:R_{1,5} & 20:R_{2,5} & 24:R_{3,5} & 33:R_{4,5} & 41:R_{5,5} & 50:R_{6,5} & 62:R_{7,5} & \cdots\\
			\hline 21:R_{1,6} & 23:R_{2,6} & 34:R_{3,6} & 40:R_{4,6} & 51:R_{5,6} & 61:R_{6,6} & 72:R_{7,6} & \cdots\\
			\hline 22:R_{1,7} & 35:R_{2,7} & 39:R_{3,7} & 52:R_{4,7} & 60:R_{5,7} & 73:R_{6,7} & 85:R_{7,7} & \cdots\\
			\hline \vdots & \vdots & \vdots & \vdots & \vdots & \vdots & \vdots & \ddots\\
			\hline
		\end{array}}
	\end{proof}
\end{section}

\begin{section}{Countabilities of Common Sets Using the Theorems}\label{sec:applications}
	\begin{subsection}{Countability of the Integers}\label{subsec:applications.integers}
		Claim: $\mathbb Z$ is a countable set.\\
		\indent$C(\mathbb Z)\equiv1$,
		\hspace{2em}
		$|\mathbb Z|\equiv\aleph_0$
		\begin{proof}
			$R:=\mathbb N_0$, $S:=\mathbb N_1$. $R$ and $S$ are countable axiomatically
			(axiom of extensionality), being the naturals themselves. $\mathbb Z\equiv
			R(1)\cup S(-1)$. This is countable via Theorem~\ref{thm:finite unions}.1.
		\end{proof}
	\end{subsection}

	\begin{subsection}{Countability of the Rationals}\label{subsec:applications.rationals}
		Claim: $\mathbb Q$ is a countable set.\\
		\indent$C(\mathbb Q)\equiv1$
		\begin{proof}
			$R:=\mathbb Z$, $S:=\mathbb N_1$. $\mathbb Q$ has the same countability as the set of all
			the ordered pairs of integers with naturals. $R\times S$ defines all of these pairs. $R$
			is countable via Section~\ref{subsec:applications.integers} and $S$ is countable
			axiomatically. $R\times S$ is thus countable via Theorem~\ref{thm:multiplications}.1.
			This argument and Theorem~\ref{thm:multiplications} derives from Cantor's enumeration of
			countable collections of countable sets.
		\end{proof}
	\end{subsection}

	\begin{subsection}{Countability of the Reals from Zero to One}\label{subsec:applications.reals 0 to 1}
		Claim: $\{x\in\mathbb R:0\leqslant x<1\}$ is a countable set.
		\hspace{2em}$C(\mathbb R_{0\leqslant x<1})\equiv1$
		\begin{proof}
			Let $R$ be a countable set where, for any $i\in\mathbb N_0$, $R_i$ can be defined to be
			the digits of $i$ reversed with ``0." at the beginning (see Section~\ref{sec:further algorithms}
			for formula). For example, $R_{246}=0.642\overline0$. This set is countable because it
			was defined to be countable, and it contains every real number in the range because every
			possible sequence of digits is in it and indexed. The sequence trends upwards,
			asymptotically approaching one, though it fluctuates wildly along the way.
			Section~\ref{subsec:applications.reals} explains this further. $R$ is used throughout the
			sections about $\mathbb R$.
		\end{proof}
	\end{subsection}

	\pagebreak\begin{subsection}{Countability of the Non-Negative Reals}
	\label{subsec:applications.non negative reals}
		Claim: $\mathbb R_{\geqslant0}$ or equivalently $\left\{x\in\mathbb R:x\geqslant0\right\}$,
		is a countable set.\hspace{2em}$C(\mathbb R_{\geqslant0})\equiv1$
		\begin{proof}
			$C(\mathbb R_{0\leqslant x<1})=1$ via Section~\ref{subsec:applications.reals 0 to 1}.
			$\mathbb R_{\geqslant0}=\displaystyle\bigcup_{i=1}^{\aleph_0}S_i$ where
			$S_i:=R[i-1]$. This is countable via Theorem~\ref{thm:infinite unions}.1.
			The following table illustrates this. $S$ is used in the next proof.\vspace{0.5em}\\
			\centerline{\begin{array}{|c|c|c|c|c|c|c|} % 7 columns
				\hline 0.\overline0 & 1.\overline0 & 2.\overline0 & 3.\overline0 & 4.\overline0 & 5.\overline0 & \cdots\\
				\hline 0.1\overline0& 1.1\overline0& 2.1\overline0& 3.1\overline0& 4.1\overline0& 5.1\overline0& \cdots\\
				\hline 0.2\overline0& 1.2\overline0& 2.2\overline0& 3.2\overline0& 4.2\overline0& 5.2\overline0& \cdots\\
				\hline 0.3\overline0& 1.3\overline0& 2.3\overline0& 3.3\overline0& 4.3\overline0& 5.3\overline0& \cdots\\
				\hline \vdots & \vdots & \vdots & \vdots & \vdots & \vdots & \cdots\\
				\hline 0.7\overline0& 1.7\overline0& 2.7\overline0& 3.7\overline0& 4.7\overline0& 5.7\overline0& \cdots\\
				\hline 0.8\overline0& 1.8\overline0& 2.8\overline0& 3.8\overline0& 4.8\overline0& 5.8\overline0& \cdots\\
				\hline 0.9\overline0& 1.9\overline0& 2.9\overline0& 3.9\overline0& 4.9\overline0& 5.9\overline0& \cdots\\
				\hline 0.01\overline0&1.01\overline0&2.01\overline0&3.01\overline0&4.01\overline0&5.01\overline0&\cdots\\
				\hline \vdots & \vdots & \vdots & \vdots & \vdots & \vdots & \ddots\\
				\hline
			\end{array}}
		\end{proof}
	\end{subsection}

	\begin{subsection}{Countability of the Reals}\label{subsec:applications.reals}
		Claim: $\mathbb R$ is a countable set which implies $2^{\aleph_0}\equiv\aleph_0$.
		\begin{proof}
			$\mathbb R_{\geqslant0}$ is countable via Section~\ref{subsec:applications.non negative reals}.
			$\mathbb R_{\geqslant0}(1, -1)\equiv\mathbb R$. This is countable via
			Theorem~\ref{thm:rings}.1. This conclusion can be further reinforced by
			the algorithmic methods in Section~\ref{sec:algorithm code}. For all real
			numbers $x$, there exists at least one sequence of elements in an $S_i$,
			each element with a higher index than the last, where the limit of the sequence
			equals $|x|$. This is because every sequence of digits in $[i-1,i)$ is contained by $S_i$.
		\end{proof}
	\end{subsection}

	\begin{subsection}{Countabilities of Miscellaneous Number Classes}
	\label{subsec:applications.miscellaneous}
		\begin{subsubsection}{Algebraic and Transcendental Numbers}
			\label{subsubsec:applications.misc.reals}
			Claim: $\mathbb A$ and $\mathbb T$ are countable sets over $\mathbb R$ and $\mathbb C$.
			\begin{proof}
				$\mathbb R$ is a countable set via Section~\ref{subsec:applications.reals}.
				Since the algebraic reals and transcendental reals are both subsets of the
				reals, they are countable via Theorem~\ref{thm:subsets}.1. They are also
				countable over the complex numbers using Section~\ref{subsec:applications.complex}
				and the same logic.
			\end{proof}
		\end{subsubsection}

		\begin{subsubsection}{Imaginary Numbers}\label{subsubsec:applications.misc.imaginary}
			Claim: $\mathbb I$ is a countable set.\hspace{2em}$C(\mathbb I)\equiv1$
			\begin{proof}
				$\mathbb R$ is a countable set via Section~\ref{subsec:applications.reals}.
				$\mathbb R(\sqrt{-1})\cup\emptyset(1)\equiv\mathbb I$. This is countable via
				Theorem~\ref{thm:finite unions}.1 becuase the union of any set and the null set
				is itself.
			\end{proof}
		\end{subsubsection}
	\end{subsection}

	\begin{subsection}{Countability of the Complex Numbers}\label{subsec:applications.complex}
		Claim: $\mathbb C$ is a countable set.
		\begin{proof}
			$\mathbb R$ is a countable set via Section~\ref{subsec:applications.reals}.
			$\mathbb R(1, \sqrt{-1})\equiv\mathbb C$. This is countable via Theorem~\ref{thm:rings}.1.
			The identity used here stems from the rectangular form of complex numbers.
		\end{proof}
	\end{subsection}
\end{section}

\begin{section}{Generalized Continuum Hypothesis}\label{sec:GCH}
	Claim: GCH has the same truth value as CH, and powersets don't change sizes of infinity.
	\begin{proof}
		Let $S$ be a countable set. $R_i := S^{i-1}$, and is countable via Theorem~\ref{thm:powers}.1.
		$\displaystyle Q:=\bigcup_{i=1}^{\aleph_0}R_i$ is countable via
		Theorem~\ref{thm:infinite unions}.1. $P :=\mathcal P(S)$. $n := |S|$. $|P|=|Q|$.
		This is because $P$ has
		$n^0$ elements with 0 sub-elements,
		$n^1$ elements with 1 sub-element,
		$n^2$ elements with 2 sub-elements, and has generally
		$n^k$ elements with $k$ sub-elements, for $k\leqslant n$.

		$S^0$ has $n^0$ elements,
		$S^1$ has $n^1$ elements, and generally
		$S^k$ has $n^k$ elements, and when taking the union of all of $S^i$, where
		$0\leqslant i\leqslant n$, the resulting set has the same cardinality as $P$. This expands
		to countably infinite sets as well, replacing $n$ with $\aleph_0$. Since $Q$ is countable,
		$P$ is countable as well, meaning $|S|\equiv|\mathcal P(S)|$ for all countable sets $S$.

		There is also no reason ``uncountable" sets can't be used for $S$, but the powerset
		would end up having the same cardinality as $S$ anyway. This is because
		Theorems~\ref{thm:powers}.1~and~\ref{thm:infinite unions}.1 can be adapted to biject to
		whatever set is inputed to the theorem, instead of $\mathbb N$, This implies that
		uncountable sets cannot be created through powersets of countable sets, and either cannot
		be created at all, or need a much stronger method to create them.
	\end{proof}
\end{section}

\begin{section}{Cantor's Diagonal Argument}\label{sec:diagonal argument}
	According to Georg Cantor in 1891$^\text{[references \ref{ref:diagonal argument},
	\ref{ref:cantor-1891}]}$, If someone is trying to list all the real numbers, they can
	always find a number that is not in the list using his ``Diagonal Argument". This
	argument basically accomplishes the same as the following, though for reals instead of
	naturals: Suppose someone is trying to make a set $S$ with every natural number. They
	first add zero to the set and the set is $\{0\}$, then they could say, ``one isn't in
	the set." When they add one and have $\{0,1\}$, they can say ``two isn't in the set,"
	then ``three isn't in the set," ``four isn't in the set," et cetera. No matter how many
	natural numbers they add, they can always find one not in it; $\max(S)+1$. This seems
	to be implying that there is not a bijection between the natural numbers and themselves,
	which is clearly wrong. Every set is bijective onto itself via the identity function.
	$f(x) := x$ for  $f:X\mapsto X$.
\end{section}

\begin{section}{Enumerating the Continuum Algorithmically}\label{sec:algorithm code}
	The following Node JS code prints out real numbers to stdout delimited by a comma-space
	pair. With infinite time and memory, it will have printed every real number. The problems
	are that it prints out both positive and negative zero, and the functions return strings.
	Both of these are easily resolvable though. There is a reference to this source code and
	the C version in Section~\ref{sec:references} References~\ref{ref:c-code}~and~\ref{ref:js-code}.
	The subscripts are just so it looks nicer. If \texttt{process.stdout.write} is replaced with
	\texttt{console.log}, then it will work in vanilla JavaScript in versions beginning with
	ECMAScript 6. The downside being \texttt{console.log} adds a trailing newline character at
	the end of each call, which isn't ideal for printing a large quantity of small strings each
	using an individual function call. The complexity of finding the $n$th real number with this
	method is less than or equal to $\mathcal O\!\left(n\log^2n\right)$, emphasis on ``less",
	because after over 16 million iterations on my machine, it hadn't slowed down noticably.

	\noindent\texttt{\\
	function isqrt(n) \{ // $\lfloor$$\sqrt\texttt{n}$$\rfloor$ for non-negative big integers n.\\
	$\indent$if (n < 2) return n;\\
	$\indent$var x$_0$, x$_1$ = n / 2n;\\\\
	$\indent$do $x$_0$ = x$_1$,\\
	$\indent\indent$x$_1$ = x$_0$ + n / x$_0$ >> 1n; // Newton's method for f(x) = x\textasciicircum 2+a\\
	$\indent$while ( x$_1$ < x$_0$ );\\\\
	$\indent$return x$_0$;\\
	\} // code continued\\
	// iterate over the natural numbers (up to 0.5*2\textasciicircum 2\textasciicircum 30)\\
	for (var runningIndex = 0n ;;) \{\\
	$\indent$let positive = true, currentIndex = runningIndex++;\\\\
	$\indent$// if the current index is odd, return a negation the previous index's value\\
	$\indent$if (currentIndex \% 2n) positive = false, currentIndex--;\\
	$\indent$currentIndex /= 2n; // divide index by 2 so all integers can be reached\\\\
	$\indent$const c = isqrt(1n + 8n*currentIndex) // intermediate value\\
	$\indent$// the index the next comments refer to\\
	$\indent\indent$, u = (c + c \% 2n) / 2n - 1n\\
	$\indent$// the largest triangle number with an integer index that is less ...\\
	$\indent$// than or equal to the current index. but subtracted from the current index\\
	$\indent\indent$, k = currentIndex - u * (u + 1n) / 2n;\\\\
	$\indent$// generate real number from indices k and u-k and print it\\
	$\indent$process.stdout.write(\textasciigrave\$\{positive~?~""~:~"-"\}\$\{k\}.\textasciigrave~+\\
	$\indent\indent$\textasciigrave\$\{u - k\}\textasciigrave.split("").reverse().join("") + ", "\\
	$\indent$);\\
	\}
	}
\end{section}

\begin{section}{Further Algorithms}\label{sec:further algorithms}
	The axioms of Zermelo-Fraenkel Set theory required to prove the algorithmic viability of
	continuum enumeration are as follows: Axiom Schema of Specification, Axiom of Union, Axiom
	of Infinity, and Axiom of Choice (for creating sets of ordered pairs). Without the Axiom of
	Choice, Theorems~\ref{thm:multiplications},~\ref{thm:rings},~and~\ref{thm:infinite unions}
	don't work, although the algorithm from Section~\ref{sec:algorithm code} is separate from
	set theory and assumes no axioms. The algorithm uses string manipulation to reverse the
	numbers and find their lengths, but the following equations do the same thing
	mathematically. length() and reverse() are only defined for natural number inputs.
	\begin{align}
		\text{length}\,n & := 1+\left\lfloor\log_bn\right\rfloor=\left\lceil\log_b(n+1)\right\rceil\\
		\text{reverse}\,n & :=\sum_{k=0}^{\text{length}\,n}\left(\left\lfloor\frac n{b^k}\right\rfloor\text{mod}\,b\right)\!b^{\text{length}\,n-k-1}\\
		R_{i,j} & :=p\left(i+\frac{\text{reverse}\,j}{b^{\text{length}\,j}}\right)
	\end{align}
	Where $i$ is \texttt k, $p$ is \texttt{(positive~?~1~:~-1)}, $j$ is \texttt{(u - k)}, and $b$
	is the base. An interesting formula used in the algorithm in Section~\ref{sec:algorithm code} is for
	the greatest triangle
	number $t=T(k)$ less than or equal to a natural number $n$, with a natural number index $k$.
	\begin{align}
		T(n) := \frac{n^2+n}2 & = \sum_{j=1}^nj\\
		t = T\left(\left\lceil\frac{\left\lfloor\sqrt{8n+1}\right\rfloor}2\right\rceil-1\right)
		& = T\left(\left\lfloor\dfrac{\sqrt{8n+1}}2\right\rceil-1\right)
	\end{align}
	Because of the bijection, there is also an inverse of the Section~\ref{sec:algorithm code}
	algorithm.\\\\
	\noindent\texttt{function inverse(string) \{ // assume valid input\\
		$\indent$const match = /\textasciicircum(-?)($\backslash$d+)$\backslash$.($\backslash$d\+)\$/.exec(string)\\
		$\indent\indent$, i = BigInt(match[2]) // the x coordinate index, or the integer part\\
		$\indent\indent$, j = BigInt(match[3]); // the y coordinate index, or the decimal part\\
		$\indent$return i*(i+1n) + (2n*i+j)*(j+1n) + BigInt(match[1] === "-");\\
		\}
	}\vspace{-1em}
\end{section}

\begin{section}{Conclusion}\label{sec:conclusion}
	Since there exists a bijection between any infinite set and its powerset, there is no set
	with a cardinality strictly or loosely in between the naturals and reals (CH and GCH), because
	they are the same cardinality. This also implies that $\beth_n=\beth_0$ for all natural numbers
	$n$, which makes sense intuitively because $2^\infty=\infty$. Also if any sequence of natural
	numbers is concatenated, it will always create a new natural number, meaning every element in
	$\mathcal P(\mathbb N_0)$ corresponds to an element in $\mathbb N_0$. This all could have
	disasterous consequences for set theory or mathematics as a whole because CH is provably
	undecidable in some models and yet provably decidable. This means either Zermelo-Fraenkel Set
	Theory with the axiom of choice (ZHC) is unsound, the aforementioned models are invalid, or at
	least one of the proofs is invalid. The entire foundation of aleph and beth numbers and sizes of
	infinities could be deeply flawed to the core. CH implies the Gimel Hypothesis is true according
	to Reference~\ref{ref:gimel}, and Wetzel's problem is false according to Reference~\ref{ref:wetzel}.
	A truth value for CH or GCH is not asserted here bacause the specifics of CH are unclear, and they can
	change its truth value. If the reals have to be bigger than the naturals, it is false, but if it
	just needs there to be no set in between them, it is true.
\end{section}

\begin{section}{References}\label{sec:references}
	\begin{enumerate}
		\item\url{https://en.wikipedia.org/wiki/Cantor's\_diagonal\_argument}\\
		\label{ref:diagonal argument}
		\text{Wikipedia page with background for Section~\ref{sec:diagonal argument}}

		\item\url{https://www.digizeitschriften.de/dms/img/?PID=GDZPPN002113910&physid=phys84#navi}\\
		\label{ref:cantor-1891}
		\text{Georg Cantor's 1891 article with the diagonal argument. Same source as on wikipedia.}

		\item\url{https://en.wikipedia.org/wiki/Continuum\_hypothesis}\\
		\label{ref:continuum}
		\text{Wikipedia page with elaboration on Section~\ref{subsec:applications.reals}}

		\item\url{https://www.github.com/drizzt536/files/tree/main/TeX/continuum}\\
		\label{ref:files}
		\text{The files for the most recent public version of this pdf and the \LaTeX~code}

		\item\url{https://raw.githubusercontent.com/drizzt536/files/main/JavaScript/continuum.js}\\
		\label{ref:js-code}
		\text{The raw JavaScript source code for the Section~\ref{sec:algorithm code}}

		\item\url{https://raw.githubusercontent.com/drizzt536/files/main/C/continuum.c}\\
		\label{ref:c-code}
		\text{The raw C source code for the Section~\ref{sec:algorithm code}}

		\item\url{https://en.wikipedia.org/wiki/Gimel\_function\#The\_gimel\_hypothesis}\\
		\label{ref:gimel}
		\text{Gimel Hypothesis Wikipedia page}

		\item\url{https://en.wikipedia.org/wiki/Wetzel's\_problem}\\
		\label{ref:wetzel}
		\text{Wikipedia page for Wetzel's Problem}
	\end{enumerate}
	\vspace{1em}\\
	Editors:
	\begin{itemize}
		\item Daniel E. Janusch
		\item Valerie Janusch, my mom
	\end{itemize}
	\vspace{3em}\\
	This document is licensed under https://raw.githubusercontent.com/drizzt536/files/main/LICENSE
	and may only be copied IN ITS ENTIRETY under penalty of law.
\end{section}

\end{document}
