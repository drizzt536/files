\documentclass[12pt]{article}

\usepackage{amsmath, amssymb, latexsym, xcolor}

\usepackage[
	top    = 0.5in,
	left   = 1in,
	right  = 1in,
	bottom = 1in,
]{geometry}

\definecolor{lightgray}{RGB}{170, 170, 170}

\color{lightgray}
\pagecolor{black}

\providecommand \dstyle \displaystyle
\providecommand \dQ {\text dQ}
\providecommand \dt {\text dt}

\begin{document}
\newgeometry{
	top    = 0in,
	left   = 1in,
	right  = 1in,
	bottom = 1in,
}

\title{PS 250 Exam 2 formulas}
\author{Daniel E. Janusch}
\maketitle

\begin{equation}
	C = \dfrac Q V
\end{equation}

\begin{equation}
	\hfill \begin{array}{ccc}
		\vspace{10px}
		\text{equivalences} & \text{Series} & \text{Parallel} \\

		\vspace{10px}
		\text{Capacitor} & \left[\sum \dfrac 1 {C_i}\right]^{-1} & \sum C_i \\
		\text{Resistor} & \sum R_i & \left[\sum \dfrac 1 {R_i}\right]^{-1}
	\end{array} \hfill
\end{equation}

\begin{equation}
	U = \dfrac 1 2 \dfrac{Q^2} C = \dfrac 1 2 C V^2 = \dfrac 1 2 Q V
\end{equation}

\begin{equation}
	u = \dfrac {\varepsilon_0} 2 \dfrac {V^2}{d^2} = \dfrac {\varepsilon_0} 2 E^2~~\text{(energy density inside a capacitor)}
\end{equation}

\begin{equation}
	\text{dielectric: } C = K C_0,~K = \dfrac{C}{C_0} = \dfrac{\varepsilon}{\varepsilon_0}
\end{equation}

\begin{equation}
	I = \dfrac \dQ \dt = |q| n A v_d \ni n \sim 20^{28} ~\left[{\rm m}^{-3}\right]
\end{equation}

\begin{equation}
	J = \dfrac I A = \dfrac E \rho ~~ \text{(current density)}
\end{equation}

\begin{equation}
	|\vec v_{\rm electron}| \sim 10^6, ~~ |\vec v_d| \sim 10^{-4}
\end{equation}

\begin{equation}
	R = \dfrac V I = \dfrac {\rho L} A
\end{equation}

\begin{equation}
	P = V I = \dfrac{V^2} R = I^2 R
\end{equation}

% equal and opposite parallel plates (capacitor)
\begin{equation}
	\text{parallel plate: }
		C = \dfrac Q V = \dfrac{\varepsilon A} d,
		E = \dfrac V d = \dfrac Q {C d} = \dfrac Q {\varepsilon A},
		d = \dfrac{\varepsilon A} C = \dfrac{\varepsilon A V} Q
\end{equation}

\begin{equation}
	\text{Junction/Current rule: } \sum I_{\rm in} = \sum I_{\rm out}
\end{equation}

\begin{equation}
	\text{Loop/Voltage rule: } \sum V = 0 ~~ \text{(around a closed loop)}
\end{equation}

\end{document}
