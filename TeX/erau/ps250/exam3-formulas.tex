\documentclass[12pt]{article}

\usepackage{amsmath, amssymb, latexsym, xcolor, graphics}

\usepackage[
	top    = 0.5in,
	left   = 1in,
	right  = 1in,
	bottom = 1in,
]{geometry}

\definecolor{lightgray}{RGB}{170, 170, 170}

\color{lightgray}
\pagecolor{black}

\newcommand \dstyle \displaystyle
\newcommand \hpx [1]{\hspace{#1px}}
\newcommand \vpx [1]{\vspace{#1px}}
\newcommand \nhpx [1]{\hspace{-#1px}}
\newcommand \nvpx [1]{\vspace{-#1px}}

\renewcommand \implies {\ensuremath{\hpx 4 \raisebox{-0.14px}{\scalebox{1.051} =} \nhpx{5.2} \Rightarrow \hpx 3}}

\providecommand \dstyle \displaystyle
\renewcommand \d {\mathrm d}
\providecommand \dQ {\d Q}
\providecommand \dA {\d A}
\providecommand \dt {\d t}

\begin{document}
\newgeometry{
	top    = 0in,
	left   = 1in,
	right  = 1in,
	bottom = 1in,
}

\title{PS 250 Exam 3 formulas}
\author{Daniel E. Janusch}
\maketitle

% TODO: what are sensible values of magnetism?

\begin{equation}
	1\,{\rm T} = 1\,\dfrac{\rm N}{\rm A\,m}
\end{equation}

\begin{equation}
	\mu_0 = 4\pi \times 10^{-7} ~ \left[\dfrac{\rm T\,m}{\rm A}\right] ~~ (\text{permeability of free space})
\end{equation}

\begin{equation}
	n = \text{charge density}
\end{equation}

\begin{equation}
	1\,{\rm G} = 10^{-4}\,{\rm T}
\end{equation}

\begin{equation}
	\vec F = F_E + F_B = q (\vec E + \vec v \times \vec B) \implies |\vec F| = q (E + v B \sin \theta)
\end{equation}

\noindent \text{this one assumes magnetic field is orthogonal to motion:}

\begin{equation}
	F_B = (\text{\# of charges}) \left(\dfrac{\rm force}{\rm charge}\right)
		= (n A \ell)(q v B)
		= (n q v_d A) (\ell B)
		= I \ell B
\end{equation}

\begin{equation}
	\vec F_B = q \vec v \times \vec B = I \vec \ell \times \vec B
\end{equation}

\begin{equation}
	\vec \mu = N I \vec A ~~ (\text{magnetic moment})
\end{equation}

\begin{equation}
	% this one can be multiplied by N for N loops.
	\vec \tau = N I \vec A \times \vec B = \vec \mu \times \vec B ~~ (\text{torque})
\end{equation}

\begin{equation}
	U = \int \tau \d\theta = -\vec \mu \cdot \vec B ~~ (\text{magnetic potential energy})
\end{equation}

\begin{equation}
	\text{point charge: } B = 10^{-7} \dfrac{q \vec v \times \hat r}{r^2}
\end{equation}

\begin{equation}
	% R is the radius of the arc the particle goes in.
	\text{point charge: } m v = |q| B R
\end{equation}

\begin{equation}
	\oint \vec B \cdot \d\ell = \mu_0 I_{\rm encl}
\end{equation}

\begin{equation}
	\mathcal E = - N \dfrac{\d\Phi_B} \dt = - N \dfrac{\d(A B \cos \theta)} \dt = B \ell v \sin \theta = I R
\end{equation}

\begin{equation}
	% r hat is from the source to the test location
	\vec B = 10^{-7} \int \dfrac{I \d\ell \times \hat r}{r^2}
\end{equation}

\newpage
\restoregeometry

\begin{equation}
	%  B_\perp
	\Phi_B = \int B \cos \theta \dA = \int \vec B \cdot \d\vec A = A B \cos \theta ~~ [{\rm Wb}={\rm T}\,{\rm m}^2]
\end{equation}

\begin{equation}
	\oint \vec B \cdot \d\vec A = 0
\end{equation}

\begin{equation}
	\text{velocity selector: } F_E = F_B \implies v = \dfrac E B
\end{equation}

\begin{equation}
	% magnetic force between two wires
	% this is really F_i / L_i, but usually L_1 = L_2.
	% F_1 / L_1 = F_2 / L_2.
	\dfrac F L = \dfrac {\mu_0 I_1 I_2} {2 \pi r}
\end{equation}

\text{magnetic field outside a solenoid is zero}

\begin{equation}
	\text{cylindrical solenoid: } B = \mu_0 n I = \mu_0 \dfrac N \ell I
\end{equation}

\begin{equation}
	\text{toroidal solenoid: } B = \dfrac{\mu_0}{2 \pi} N \dfrac I r, r_{\rm min} < r < r_{\rm max}
\end{equation}

\begin{equation}
	\text{distance $r$ away from a radius-$R$ wire center: } B = \dfrac{\mu_0}{2 \pi} \begin{cases}
		\vpx 5
		\dfrac {I r} {R^2}, r < R \\
		\dfrac I r, r \ge R
	\end{cases}
\end{equation}

\begin{equation}
	\text{torus of current at $a < r < b$: } B = \dfrac {\mu_0} {2 \pi} \dfrac I r \begin{cases}
		0, r < a \\
		\dfrac {r^2 - a^2} {b^2 - a^2}, a < r < b \\
		1, b < r \land \vec I \text{ into/out of page} \\
		0, b < r \land \vec I \text{ around loop} \\
	\end{cases}
\end{equation}

% NOTE: ch. 27 p 926 (actual p 906)
% TODO: ch. 28 p 962 (actual p 942)
% TODO: ch. 29 p 997 (actual p 977)

% TODO: realistic unit ranges
	% B ~ [10^-6, 10]
	% \Phi_B ~ [?, ?]
	% F_B ~ [10^-15, 10^15]
	% emf ~ [10^-3, 10^3]

% TODO: hall effect stuff
	% electric field magnitude and direction
	% drift velocity
	% hall emf

% simplifications for B
	% outside a toroid (r < r_a, or r > r_b): B = 0
	% outside a long ideal solenoid: B \approx 0
	% inside a hollow region: often B = 0 (via ampere's law)
	% drift velocity


\end{document}
