% TODO: go through and replace \varepsilon with \mathcal E for EMF.

\documentclass[12pt]{article}
\usepackage{amsmath, amssymb, latexsym, xcolor, graphics, wasysym, circuitikz}

\usepackage[
	top    = 0.5in,
	left   = 0.5in,
	right  = 0.5in,
	bottom = 1.0in,
]{geometry}

\definecolor{lightgray}{RGB}{170, 170, 170}
\color{lightgray}
\pagecolor{black}

\newcommand \capacitor {\tikz \draw (0, 0) to[C] (1, 0);}
\newcommand \resistor {\tikz \draw (0, 0) to[R] (1, 0);}
\newcommand \inductor {\tikz \draw (0, 0) to[L] (1, 0);}
\newcommand \battery {\tikz \draw (0, 0) to[battery1] (1, 0);}
\newcommand \acpower {\tikz \draw (0, 0) to[sV] (0, 1);}

\newcommand \negphantom [1]{\hspace{-\widthof{#1}}}
\newcommand \replace [2]{\negphantom{#1}\phantom{#2}}
\newcommand \dstyle \displaystyle
\newcommand \hpx [1]{\hspace{#1px}}
\newcommand \vpx [1]{\vspace{#1px}}
\newcommand \nhpx [1]{\hspace{-#1px}}
\newcommand \nvpx [1]{\vspace{-#1px}}
\newcommand \kvalue {\dfrac 1{4 \pi \varepsilon_0}}
% \! == \nhpx 2
% \! == \nhpx 2

\renewcommand \implies {\hpx 4 \raisebox{-0.14px}{\scalebox{1.051} =} \nhpx{5.2} \Rightarrow \hpx 3}

\newcommand \ihat {\hat{\nhpx 1 \textit\i}}
\newcommand \jhat {\hat{\nhpx 1 \textit\j}}
\newcommand \khat {\hat{\textit k}}
\newcommand \encl {\mathrm{encl}}

\renewcommand \d {\mathrm d} % normally, \d is a text mode under dot diacritic thing
\newcommand \dl {\d l}
\newcommand \dq {\d q}
\newcommand \dr {\d r}
\newcommand \dt {\d t}
\newcommand \dy {\d y}
\newcommand \dA {\d A}
\newcommand \dE {\d E}
\newcommand \dQ {\d Q}
\newcommand \dI {\d I}

\begin{document}

\newgeometry{
	top    = 0.0in,
	left   = 0.5in,
	right  = 0.5in,
	bottom = 1.0in,
}

\title{PS 250 Notes}
\author{Daniel E. Janusch}
\maketitle

magnetism is just electricity moving.

% TODO: change this to a `tabular`
$\begin{array}{c|c|c}
{\rm Force:} & {\rm gravity} & {\rm electromagnetic}\\
{\rm conserved~quantity:} & {\rm mass} & {\rm charge}\\
\hline
{\rm electron} & 9.108 \times 10^{-31} {\rm kg} & -1.602 \times 10^{-19} {\rm C}\\
{\rm proton} & 1.673 \times 10^{-27} {\rm kg} & +1.602 \times 10^{-19} {\rm C}\\
{\rm neutron} & 1.675 \times 10^{-27} {\rm kg} & 0 {\rm C}\\
\end{array}$

fundamental charge: $|e| = 1.602 \times 10^{-19}$

$1 C$ = 1 coulomb = $6.242 \times 10^{18} |e|$

$\varepsilon_0 = 8.85 \times 10^{-12}\dfrac{{\rm C}^2}{{\rm N}~{\rm m}^2}$

$|\vec F| = \dfrac{G m_1 m_2}{r^2}$ (gravitation)

$|\vec F| = \kvalue \dfrac{|q_1 q_2|}{r^2}$ (coulombs law)

whenever $4\pi$, shows up, something with a sphere is happening.

$\kvalue = k = 8.988\times 10^9 \dfrac{{\rm N}~{\rm m}^2}{{\rm C}^2}$

ratio of $F_G$ and $F_E$ for two $\alpha$ particles (helium particles): $\dfrac{F_E}{F_G} = \dfrac{\dfrac 1{4\pi\varepsilon_0}\dfrac{q^2}{r^2}}{G\dfrac{m^2}{r^2}} = \dfrac 1 {4\pi\varepsilon_0 G} \dfrac{q^2}{m^2} \approx 3.1 \times 10^{35}$

for planetary scale (Earth): $\dfrac{F_E}{F_G} \approx 6.1 \times 10^{-9}$

$g$ is really the gravitational field strength (as well as the gravitational acceleration)

electrostatics is a conservative force.

$\mu = 10^{-6}$ (micro)

$n = 10^{-9}$ (nano)

$p = 10^{-12}$ (pico)

3 $\alpha$ particles, with 1m in-between them (in a line):

$F_A = F_{BA} + F_{CA}$

$F_A = \kvalue \dfrac{q_A q_B}{r_{AB}^2} + \kvalue \dfrac{q_A q_C}{r_{AC}^2} = 1.15nN (-\ihat)$

problem 3 (see the PNG): split forces into components. the y-components cancel out.

$F = 2 \cdot \kvalue \dfrac{q^2}{r^2}$

$F_x = F \cos \alpha$

$\vec F_x = 0.46 N \, \ihat$

(NOTE: a lightning bolt is up to 100C).

check your math to make sure your answer values are in the right ballpark.

\newpage
\restoregeometry

Electric field: $\vec E = \dfrac{\vec F}{q_0}$, $\vec F = q_0 \vec E$.

tells you what sort of force a charge would feel given a net charge.

electric fields are the first things to be called fields.

$\vec F_G = m\vec g$ is kind of the same thing as this (by analogy).

$\vec E = \kvalue \dfrac q{r^2}\hat r$

$\vec F = \kvalue \dfrac {q q_0}{r^2}\hat r$

the point is to get away from ``point charge".

charge points away from a more positive charge, and points towards more negative charge.

units for electric field are $\dfrac{\rm N}{\rm C}$ because it is a force per unit charge.

electric field is also often given in volts per meter.

volt\,:\,energy\,::\,electric field\,:\,force

you can add electric field contributions to get a net electric field thing, like how you can get net force.

$\dstyle E_{\rm total} = \sum_i \kvalue \dfrac{q_i}{r_i^2}\hat r_i$

this isn't a very good way to do this, but it works conceptually.

It is better to use continuous integrals rather than discrete sums.

$\dstyle E_{\rm total} = \int \dfrac 1{4\pi \varepsilon_0} \dfrac \dq{r^2} = \kvalue \int \dfrac 1{r^2}\dq$

\begin{tabular}{p{50px}|p{35px}|p{112px}|p{28px}|p{52px}}
	dimension & symbol & name & units & in integral\\ \hline
	0 & $Q,q$ & charge & C & d$q$\\ \hline
	1 & $\lambda$ & linear charge density & C/m & $\lambda$d$l$\\ \hline
	2 & $\sigma$ & surface charge density & C/${\rm m}^2$ & $\sigma$d$A$\\ \hline
	3 & $\rho$ & volume charge density & C/${\rm m}^3$ & $\rho$d$V$
\end{tabular}

charge from far away things can be approximated as a point charge.

Example 21.9: ring of charge; distance $x$ from the center into the 3rd dimension: $E = \kvalue \dfrac{Qx}{\left(x^2 + a^2\right)^{3/2}}\ihat$

Example 21.10: electric field a distance $x$ from the center of a segment of charge. the y field contributions all cancel.

$\d\vec E = \dE_X \ihat = \kvalue \dfrac \dQ {r^2} \cos \alpha~\ihat$.

$\dE_x = \kvalue \dfrac{\dQ x}{\left(x^2 + y^2\right)^{3/2}}$

$\dQ = \lambda \dl = \dfrac Q{2a} \dl$. (in terms of the variables in the problem, it is $\dy$)

$\dstyle \int \dE_x = \kvalue \dfrac {Qx}{2a} \int \dfrac 1{\left(x^2 + y^2\right)^{3/2}}\dy$

$\dstyle \int \dfrac \dy {\left(x^2 + y^2\right)^{3/2}} = \dfrac 1{x^2} \dfrac y{\sqrt{x^2 + y^2}}$

$E_x = \left. \kvalue \dfrac {Qx}{2a} \dfrac 1{x^2} \dfrac y{\sqrt{x^2 + y^2}} \right|_{y=-a}^{y=a} = \kvalue \dfrac Q{x\sqrt{x^2 + a^2}}$

$\hfill a \ll x \implies \vec E \approx \kvalue \dfrac Q{x^2} \hat r \hfill a \gg x \implies \vec E \approx \kvalue \dfrac {2 \lambda} x \hat r\hfill$

$Q = \lambda 2a$ (at least for this problem)

\newpage

Evaluate $E$ a distance $x$ (out of the disk, into the third dimension) from the center of a uniformly-charged disk. you can either do a double integral, or use the ring result and do a single integral.

$\vec E = \kvalue \dfrac{Qx}{\left(x^2 + a^2\right)^{3/2}} \ihat$, for a ring.

$\dE_x = \kvalue \dfrac{x \, \dQ}{\left(x^2 + r^2\right)^{3/2}}$

$\dQ = \sigma \dA = \sigma 2\pi r\dr$

$\dE_x = \kvalue \dfrac{x \sigma 2\pi r\dr}{\left(x^2 + r^2\right)^{3/2}} \dr$

$\dstyle \int \!\! \dE_x = \dfrac{x \sigma}{4 \varepsilon_0} \int_0^R \!\! \dfrac{2r}{\left(x^2 + r^2\right)^{3/2}}\dr.~~\text{v-sub}~x^2 + r^2 \mapsto r \implies \int \dE_x = \dfrac{x \sigma}{4 \varepsilon_0} \int_{x^2}^{x^2 + R^2} \hspace{-10px} r^{-3/2}\dr$

$E_x = \dfrac \sigma{2 \varepsilon_0} \! \left[1 - \dfrac x{\sqrt{x^2 + R^2}}\right] \ihat$

$\hfill R \ll x \implies \vec E \stackrel?\approx \vec 0 \hfill R \gg x \implies \vec E \approx \dfrac \sigma {2 \varepsilon_0} \hfill$

two parallel sheets make a constant electric field because of the $R \gg x$ approximation one.

This can be used to make a capacitor.

Gauss's Law for electricity depends on electric flux ($\Phi$)

$\Phi_E = \vec E \cdot \vec A$ ($A$ is the surface the electric field passes through)

And in  general, $\dstyle \Phi_E = \iint \! \vec E \cdot \d\vec A$

there isn't a better way to do it than by using an area vector. It is just a normal vector.

Electric flux is a measure of the electric field over a surface.

$+$flux if electric fields is exiting the object

$-$flux if electric fields is entering the object

for any charge external to a 3D surface, the net flux through that surface is \underline{0} since whatever flux enters must leave again.

Gauss's Law for Electricity: $\dstyle \Phi_E = \oiint \vec E \cdot \d\vec A = EA = \dfrac{q_\encl}{\varepsilon_0}$

NOTE: $q_\encl = q_{\rm enclosed}$

The $\circ$ on the integral means through a closed surface (surface integral). there can be folds and dents, but no gaps, etc.

Electric flux units is $\dfrac{{\rm N}\,{\rm m}^2}{\rm C} = \dfrac{{\rm kg}\,{\rm m}^3}{{\rm C}\,{\rm s}^2}$

$\varepsilon_0$ is the vacuum permittivity constant.

For a positive charge, the electric force will be in the \emph{same direction} as the electric field, and for a negative charge, the electric force will be in the \emph{opposite direction} as the electric field. NOTE: the electric force points from high to low

Q: how do you know what the direction of the electric field is?

Electric Potential (Voltage):

$W = - \Delta U (= mgh$ for gravity).

this is directly analogous to what is happening in a uniform electric field.

$W = - \Delta U = \vec F \cdot \vec r$ still holds. $W = q_0 E d$ for two parallel plates that have opposite charge and a distance $d$ apart.

$\dstyle W = - \Delta U = \int_a^b \vec F(r) \cdot \vec \dr$ (2 point charges)

$\dstyle W = \int_{r_a}^{r_b}\kvalue \dfrac{q q_0}{r^2} \dr$ (this gives $W = -\Delta U$)

\newpage

Electrostatic force is a conservative force.

\begin{equation*} \vec F = q_0 \vec E \implies \vec E = \dfrac{\vec F}{q_0} \end{equation*}
\begin{equation*} U = q_0 V \implies V = \dfrac U{q_0} \end{equation*}

$\dstyle \Delta V = -\int_{r_a}^{r_b} \!\!\! \vec E \cdot \vec \dr$

$\dstyle V = \kvalue \sum_i \dfrac{q_i}{r_i} = \kvalue \int \dfrac \dq r = \dfrac U{q_0}$

$r$ is the distance from the charge to the place you are calculating $V$. 

$\vec E = - \vec \nabla V$

Self Capacitance: the ability to hold charge $C = \dfrac Q V$.

There are two types of capacitance. self capacitance and mutual capacitance.

Mutual Capacitance is what you are probably talking about.

The units of $C$ are $1 {\rm F} = 1\,{\rm farad} = 1\dfrac{\rm C}{\rm V} = 1 \dfrac{{\rm C}^2}{\rm N \, m}$

For two opposite parallel infinite plates in open air: $C = \dfrac{\varepsilon_0 A} d$

If $1 {\rm F}$ is a lot, a more realistic value is $c\;\mu {\rm F}$ or even $c\;p{\rm F}$.

If instead of equal length parallel plates, you have a short one and a long one, you have
a battery. I forgot which one is the positive and which is the negative. Capacitors don't
actually have to use parallel plates, that is just what the symbol is. Most commonly (or
just the ones in the lab idk), they are like parallel plates but wrapped around or something.

this is the symbol for a capacitor: \capacitor. the lines have even length.

Two capacitors that are in sequence are ``in series".

Charge is not supposed to jump across the plates in a capacitor. If they do, it will pop,
smoke, and no longer be usable.

Equivalent capacitance in sequence formula: $\dstyle C_{\rm eq} = \left[\sum_i \dfrac 1 {C_i}\right]^{-1}$

Equivalent capacitance in parallel formula: $\dstyle C_{\rm eq} = \sum_i C_i$

The charges for parallel capacitors can be different, whereas for capacitors in series,
the charge has to be the same. The change in voltage has to be the same in either case.

If you have two capacitors in parallel, you are basically just creating a larger area,
and thus a larger capacitance.

The rules for resistors in parallel and in series are the same as in capacitors, but flipped.

Capacitance goes down when you put them in series.

$\dstyle U = \dfrac 1 2 \dfrac{Q^2} C = \dfrac 1 2 C V^2 = \dfrac 1 2 Q V$

$u$ = energy density

For two parallel plates (inside a capacitor), $u = \dfrac {\varepsilon_0} 2 \dfrac {V^2}{d^2} = \dfrac {\varepsilon_0} 2 E^2$.

this is going to come back up later.

NOTE: next will be resistors, and then inductors.

\newpage

$C = K C_0$, where $K = \dfrac C {C_0} = \dfrac \varepsilon {\varepsilon_0}$

$K$ (Dielectric constant) is a dimensionless quantity.

In general, most formulas should use $\varepsilon$ instead of $\varepsilon_0$.

The smallest possible $K$ is 1 (vacuum condition).
The permittivity of air is $K \approx 1.00059$ at standard atmosphere.
There is a table in chapter 24 of the text book with values of $K$.
In most cases, $K$ is a number less than 10.
For specially created materials, you can get $K$ on the order of like 300.

Most capacitors will have a Dielectric constant measurably different from air.

For a capacitor where the dielectric constant isn't uniform, you can pretend it is parallel capacitors.

moving charge is both current and the source of magnetism.

Current is defined as the flow of charge. $\dstyle I = \dfrac \dQ \dt = \dot Q$

1 Ampere/Amp ($A$) = 1 $\dfrac Cs$. flow of charge.

electron movement is mostly a ``random walk". The drift velocity is how much they actually
move along in a direction. The regular velocity is just them bouncing around.

$\left|\vec v_{\rm electron} \right| \sim 10^6 \rm m/s$.

drift velocity: $\left|\vec v_{\rm drift} \right| \sim 10^{-4} \rm m/s$

current goes the opposite way as the charge carries (usually). It isn't strictly necessary that electrons are the charge carriers for current, but they usually are.

$I = |q| nAv_d$. $v_d$ is the drift velocity. $A$ is the cross-sectional area, $n$ is concentration
of free charge carriers (electrons per cubic meter that are free to move around; volume density
of free electrons).

$n \approx 10^{28} \left[{\rm m}^{-3}\right]$, for good conductors, but varies from material to material.

Alternating current (AC) has a sinusoidal oscillation.

Direct current (DC) has no oscillation. direct charge movement.

AC is used for the power grid, but normal batteries and stuff use DC.

AC doesn't make sense without magnetism.

Current density: $J = \dfrac I A \left[\dfrac{\rm A}{{\rm m}^2}\right]$

Current is always a scalar, but current density can be a vector quantity.

Current isn't a vector because the direction doesn't really matter at all, but the current density
is more of a real-world physical definition.

A realistic diameter of a wire is 1.02mm. They are usually given in gauge though.

Most metalic conductors have a free electron density of around $10^{28}$ per cubic meter.

Current densities can be very large, even when the current is small.

$I$ is sensible to be between the scale of like $10^{-3}$A and $10$A in normal scenarios.

In more than 15A, the circuit breaker will trip in households.

Resistivity: $\rho = \dfrac E J \left[\dfrac{\rm V\, m}{\rm A}\right]$ (Ohm's Law). High for insulating materials, low for conductors.

In circuits, electric field values will be small, $E = 10^{-3}$ would be a reasonable value.

Ohm's Law is like Hooke's law, where it doesn't always work, but it is still useful.

Resistance: $R = \dfrac V I = \dfrac{\rho L} A$

Resistivity is a property of a material. Resistance is a property of a circuit element.

$V = I R$ is also sometimes referred to as Ohm's Law. This is the one we usually mean.

Resistance is also in Ohms, denoted by $\Omega$. the symbol for a resistor is \resistor.

Electromotive force (EMF) $\varepsilon$, is the \emph{voltage} a battery supplies.

As a 9V battery wears down, it is still 9V, but the resistance changes? $\varepsilon = I r$ ($r$ = internal resistance)

Batteries aren't a source of energy, they are just a source of potential difference.

\newpage

The symbol for a battery is \battery.

In a battery, the longer terminal is positive and the shorter terminal is negative.

Ammeter: amp measuring tool. Voltmeter: volt measuring tool

$\Omega {\rm A} = {\rm V}$p

parallel plate:
$C = \dfrac Q V = \dfrac{\varepsilon A} d$,
$E = \dfrac V d = \dfrac Q {C d} = \dfrac Q {\varepsilon A}$,
$d = \dfrac{\varepsilon A}{C} = \dfrac{\varepsilon A V} Q$.

% C = Q^2/W = Q/(E d)

capacitance is always positive.

in a series circuit, charge is conserved across all capacitors.

capacitors in parallel have the same voltage. the voltage for each is the same as $V_{\rm in}$.

Kirchhoff's Rules:
Junction rule: $\dstyle \sum I = 0$.
Current entering a junction is positive.
Current exiting a junction is negative.
A junction is any circuit node with 3 or more connections.
$I$ flowing inwards = $I$ flowing outwards.

Loop rule: $\dstyle \sum V = 0$.
Any path you take on a closed loop in the circuit should get you back to the same voltage.

NOTE: normally go from the negative to the positive, which would have a positive EMF ($\varepsilon$).

NOTE: there is a concept of negative resistance, but we will not be using it.

If you walk along conventional current, $V = -IR$ for a resistor,
or $V = IR$ if you walk against conventional current.

Normally, you get positive voltage from the battery, and negative voltage from resisters.

rechargable batteries will have the current flowing the wrong way when they are recharging.
You can also have bad circuit design where the current will be flowing the wrong way.

NOTE: conventional current flow is that positive charge comes out of the longer (+) side of the battery.

It is usually easier to apply the junction rule before everything else.

NOTE: ${\rm A} \cdot \Omega = {\rm V}$

Q: Does a capacitor change the voltage? Yes, $V_{\rm drop} = \dfrac Q C$

Starting at the negative terminal of the biggest battery and following conventional current flow likely
makes the problem easier for the loop rule. Not a rule, but kind of a guideline.

Every system of resistors has an $R_{\rm eq}$ if you measure across the whole system.

negative current just means current flowing the opposite direction that you defined.

RC Circuits: Circuits with both resistors and capacitors in them.

Grounding circuits is important for some reason.

When you walk from the negative terminal to the positive terminal of a battery,
voltage goes up by $\varepsilon$.

You can't assume that a circuit with capacitors will reach equilibrium almost immedately,
like you can with resistors.

charging RC circuit equation: $\varepsilon - IR - \dfrac Q C = 0$.

formulas for charging RC circuits:

$\dstyle Q(t) = C \varepsilon \! \left(1 - e^{-t / \tau}\right) = Q_f \! \left(1 - e^{t / \tau}\right)$

$\dstyle I(t) = \dfrac \varepsilon R e^{-t/\tau} = I_0 e^{-t/\tau}$
\hspace{20px}
$V(t) = V_0 \left(1 - e^{-t/\tau}\right)$

Characteristic time scale:
$\tau = RC$ (RC time constant.)

discharging RC circuit equation: $\dfrac Q C - I R = 0$.

$Q(t) = Q_0 e^{-t/\tau}$, $I(t) = \dfrac{Q_0} \tau e^{-t/\tau} = \dfrac{V_0} R e^{-t/\tau}$

$V(t) = V_0 e^{-t/\tau}$

$P = V I$.

\newpage

capacitors will sometimes supply power, and sometimes disperse power.

batteries supply power, and resistors disperse power.

$R = \rho \dfrac L A$

{\Large Magnetism:}

The root cause of magnetism is because of a combination of electricity and relativity.

A magnetic dipole is set up a bit differently than an electric dipole.

magnetic field creates closed loops exiting the north and entering the south.

the field lines on magnetic fields never end.

an electric mono-pole is just a single charge. a magnetic mono-pole doesn't exist.

The symbol for magnetism is $\vec B$.

NOTE: for an Electric Field: $\vec F = q \vec E$ (effect of net charge)

For Magnetism: $\vec F = q \vec v \times \vec B$, $|F| = q v B \sin \theta$ (effect of charge motion)

Sometimes you get both, $F = q \, (\vec E + \vec v \times \vec B)$

1 $\dfrac {\rm N}{\rm A \, m}$ = 1 Tesla = 1 T.

The Tesla is the SI unit, but the CGS units for magnetism is 1 Gauss = 1 G.

NOTE: Surface magnetism on Earth is around 1.5 G.

NOTE: 1 G = $10^{-4}$ T.

Reasonable values will be like a few Gauss.

The moving charge in a magnet is in the motion of the electrons.
Billions of them added together leads to a net magnetic effect.
That is just for a stationary magnet, it is different for an electromagnet.

the convention used for magnetic force direction in 2D points with the magnetism
when drawn by hand is: dots imply out of the board, xs imply into the board.

$\vec F = |q| |v| B = m \dfrac{v^2} R \implies R = \dfrac{m v}{|q| B}$

% $\omega = \dfrac v R = \dfrac{|q| B} m$, $f = \dfrac \omega {2 \pi}$
% $T = \dfrac{2 \pi R} v$ (period for full rotation)

this formula assumes the magnetic field is orthogonal to the motion. The value of $v$
is the one that ties to the force: $v_x$, $v_y$, etc., whatever that happens to be.

\emph{hitch}: distance orthogonal to circle a helix moves per full revolution of the circle.

$q (v_x \ihat + v_z \khat) \times B \ihat = v_z \khat \times B \ihat = \ldots$

massive? Like low taper fade?

Q: what is the difference between an ``external" and ``internal" magnetic field?

If you have a capacitor inside an \emph{external} magnetic field, you will have force from both
the electric field and the magnetic field. Make sure the signs are correct, it could be $-E$.
$v = E/B$ (velocity selector).

Point charges in a magnetic field move in a circle?

$F$ = (number of charges) $\times$ (force per charge).

$F = (n A l) (q v B)$

$F = (n q v_d A) (l B) = I l B$

$n$ is defined the same as before. this is for a line of charges or something like that.
It also assumes $90^\circ$ between the field and the velocity.

$F = I \vec l \times \vec B$.
$|\vec l| = l$, and the direction is the direction of the current.

$\vec F = q \vec v \times \vec B = I \vec \ell \times \vec B$

You can get a net effect even if the magnetic force is 0. (torque)

$\vec \tau = \vec r \times \vec F \implies \tau = I \vec A \times \vec B$,
where $\vec A$ is perpendicular to the actual area.

$\vec \mu := I \vec A \text{ (magnetic moment)} \implies \vec \tau = \vec \mu \times \vec B$

Bohr Magneton?

You can make a motor with electrons rotating around an external magnetic field.

\newpage

$\dstyle U = \int \! \tau \, \d\theta
= \int \! \mu B \sin \theta \, \d\theta
= -\mu B \cos \theta
= -\vec \mu \cdot \! \vec B$

For $N$ loops of copper wire, you will have approximately $N$ times the torque as for one loop:

$\tau \to N \tau$ and $\mu \to N \mu$

A single particle can't speed up or slow down due to the magnetic effect because it works at right angles,
it can only change direction.

Hall Effect: $J_x = \dfrac I A = n q v_d = \dfrac{n q E_z}{B_y}$ and $n \stackrel?= \dfrac{J_x B_y}{q E_z}$.

$P = V I = \dfrac{V^2} R = I^2 R$

% $\vec v \times \vec B = \vec E$?

High potential $\to$ low potential $\implies$ negative voltage (e.g. resistor).

Q: does a circuit create a magnetic field since charge is moving through the crcuit?

You need like at least $10^5$ before you can expect measurable magnetic field.

Point charge:
$\vec E = \kvalue \dfrac q {r^2} \hat r$,
$\vec F_E = q \vec E$

$\vec B = \dfrac{\mu_0}{4\pi} \dfrac{q \vec v \times \hat r}{r^2}$,
$\vec F_B = q \vec v \times \vec B$

$B = \dfrac{\mu_0}{4\pi} \dfrac{q \vec v \sin \phi}{r^2}$

thumb points towards positive charge, magnetic field curls around. right hand rule.

magnetic field rotates around the charge, instead of just eminating out of it like electric field does.

$\mu_0 = 4\pi \times 10^{-7} \dfrac{\rm T\,m}{\rm A}$: permeability of free space. (NOTE: not permittivity)

$c^2 = \dfrac 1 {\mu_0 \varepsilon_0} \implies \mu_0 = \dfrac 1 {c^2 \varepsilon_0}$

for a force from magnetism, use the velocity of the thing being acted on, rather than the thing doing the acting.

For two particles moving in opposite directions with the same speed, $\dfrac{F_B}{F_E} = \dfrac{v^2}{c^2}$.

Law of Biot and Savart:

% I can't tell if this is supposed to use \vec p or \hat r.
$\dstyle \int \d\vec B = \int \dfrac{\mu_0}{4\pi} \dfrac{I \d\ell \times \hat r}{r^2}
	\implies \vec B = \dfrac{\mu_0}{4\pi} \int \dfrac{I \d\ell \times \hat r}{r^2}$

Line of current, distance $x$ perpendicular:
$\dstyle \vec B = -\dfrac{\mu_0 I}{4\pi} \dfrac{2a}{x \sqrt{x^2 + a^2}} \khat$

$\dstyle a \gg x \implies \dstyle \lim_{a\to\infty}B = -\dfrac{\mu_0 I}{2\pi r}$

loop of current, distance $x$ perpendicular in the direction to where $I$ is clockwise:
$B = \dfrac{\mu_0 I}{4\pi} \dfrac a {\left(x^2 + a^2\right)^{3/2}} 2\pi a$

$a \gg x \implies B = \dfrac{\mu_0 I}{2a}$. Or multiply by $N$ for with $N$ loops (to an extent)

Electricity:
$\dstyle \Phi_E = \int \vec E \cdot \d\vec A = EA\cos \theta. \hpx{38}
\oint \vec E \cdot \d\vec A = EA = \dfrac{Q_\encl}{\varepsilon_0}$

Magnetism: $\dstyle \Phi_B = \int \vec B \cdot \d\vec A = |\vec B| |\vec A| \cos \theta. \hpx{20}
\oint \vec B \cdot \d\vec A = BA = 0$

Ampere's Law: $\dstyle \oint \vec B \cdot \d\ell = \mu_0 I_\encl$

If a wire is modeled as a cylinder (radius $r$) or a line, you get the same magnetic field.

Inside: $B = \dfrac{\mu_0}{2 \pi} I \dfrac r {R^2}$
Outside: $B = \dfrac{\mu_0}{2 \pi} \dfrac I r$

Ferromagnetic materials have strong magnetic properties and have high permeability (e.g. iron).

copper isn't very magnetic, so you can approximate $\mu$ it to $\mu_0$ for copper.

Inside a cylindrical solenoid: $B = \mu_0 \dfrac N L I = \mu_0 n I$ ($n$ is a density of something)
Outside a cylindrical solenoid: $B = 0$

Inside the enpty part of a Toroidal solenoid: $B = 0$ (no enclosed current)
Inside a Toroidal solenoid: $B = \dfrac{\mu_0} {2 \pi r} N I$ (with $N$ loops)
Outside a Toroidal solenoid: $B = 0$ (enclosed current cancels out)

The $r$ is a variable, but sometimes you can just pretend it is just always the average radius instead.

Faraday's Law of Electromagnetic Induction:
$\varepsilon_{\rm induced} = - N \dfrac{\d\Phi_B} \dt$~~~~(for $N$ loops)

if $A$ (thumb) and $I$ (curl) is right handed, $\varepsilon$ is positive, otherwise it is negative.

Lenz's law: Faraday's law is always a back-reaction of the other stuff that is happening (pushback against change).

For this stuff, you can just use $\Phi_B = \vec B \cdot \vec A$ instead of the full integral.

You can drop the negative on the $\varepsilon$ calculation, calculate the magnitude, and then
get the direction from Lenz's law.

$\varepsilon_{\rm ind} = IR$ ~~~~ (induced current is an Ohm's law thing)

$\dstyle \varepsilon_{\rm ind} = \oint (\vec v \times \vec B) \cdot \d\vec \ell
= \int \vec E \cdot \d\vec \ell$

Another Faraday's Law form:
$\dstyle \int \vec E_{\rm ind} \cdot \d\vec \ell = - \dfrac{\d\Phi_B} \dt$

$\dstyle \oint \vec E \cdot \d\vec\ell = \varepsilon_{\rm ind} = - \dfrac{\d\Phi_B} \dt$

Coiled Wire Solenoid: $E = \dfrac{A \mu_0 n}{2 \pi r} \dfrac \dI \dt$

charge on a capacitor: $Q(t) = \varepsilon \Phi_E \implies I(t) = \varepsilon \dfrac {\d\Phi_E} \dt$.

electric field $E$, current $I_C$ (conventional current). $I_D = I(t)$ (displacement current).

Electrostatic Ampere's Law: $\dstyle \oint \vec B \cdot \d\vec\ell = \mu_0 I_\encl$

Complete Ampere's Law: $\dstyle \oint \vec B \cdot \d\vec\ell = \mu_0 (I_C + I_D)_\encl
= \mu_0 \! \left(I_C + \varepsilon \dfrac {\d\Phi_E} \dt\right)_\encl$

{\Large Maxwell's Equations in Vacuum (Integral forms):}

~~~~~ - Gauss 1: $\dstyle \oint \vec E \cdot \d\vec A = \dfrac {Q_\encl}{\varepsilon_0}$

~~~~~ - Gauss 2: $\dstyle \oint \vec B \cdot \d\vec A = 0$

~~~~~ - Faraday: $\dstyle \oint \vec E \cdot \d\vec\ell = -\dfrac {\d\Phi_B} \dt$

~~~~~ - Ampere: $\dstyle \oint \vec B \cdot \d\vec\ell = \mu_0 \! \left(I_C + \varepsilon \dfrac {\d\Phi_E} \dt\right)_\encl$

\vpx{20}
{\Large Maxwell's Equations in Vacuum (Derivative forms):}

~~~~~ - Gauss 1: $\dstyle \nabla \cdot \vec E = \dfrac \rho {\varepsilon_0}$

~~~~~ - Gauss 2: $\nabla \cdot \vec B = 0$

~~~~~ - Faraday: $\nabla \times \vec E = -\dfrac {\partial \vec B}{\partial t}$

~~~~~ - Ampere: $\nabla \times \vec B = \mu_0 \varepsilon_0 \dfrac{\partial \vec E}{\partial t} + \mu_0 \vec J$

\newpage

inner and outer solenoid (solenoid 2 is inside solenoid 1):

$\varepsilon_1 = - N_1 N_2 \dfrac {A_2} \ell \mu_0 \dfrac {\dI_2} \dt = - M \dfrac {\dI_2} \dt$

$\varepsilon_2 = - N_1 N_2 \dfrac {A_2} \ell \mu_0 \dfrac {\dI_1} \dt = - M \dfrac {\dI_1} \dt$

$\Phi = \mu_0 \dfrac {N} \ell I$

$M = \dfrac {N_2 \Phi_{B2}}{I_1} = \dfrac {N_1 \Phi_{B1}}{I_2}$

$M$ is a mutual inductance. The unit is a Henry (H). $H = \dfrac{\rm J}{{\rm A}^2}$

$M = \dfrac {N_2 \Phi_{B2}}{I_1} = \dfrac{N_1 B_2 A_2}{I_2} = \dfrac{N_1 \mu_0 N_2 I_2}{I_2 L} A_2 = \dfrac{\mu_0 A_2 N_1 N_2} L$

Mutual inductance should be small, like how Farads are a lot of capacitance.

We will probably have $\mu$H and mH and stuff.

Capacitance means mutual capacitance, but inductance means self inductance
% TODO: this might be wrong. C = Q / V may or may not be self capacitance.

$L$ is for self inductance because it was named after John L. Inductor. And $I$ is already for current.

$L = \dfrac{N \Phi_B} I$

Inductors are given in circuits by \inductor.

Capacitor: $V = \dfrac Q C$

Resistor: $V = I R$

Inductor: $V = L \dfrac \dI \dt$

Toroidal solenoid of radius $R  =\dfrac{r_{\rm min} + r_{\rm max}} 2$ (thin wire so $r_{\rm min} \approx r_{\rm max}$):

$L = \dfrac {\mu_0 N^2 A} {2 \pi R}$

$\dstyle P = V I = - L \dfrac \dI \dt I \implies \int P \dt = \int L I \dI \implies U = \dfrac 1 2 L I^2$

$u = \dfrac 1 2 \dfrac{L I^2}{2 \pi R A} = \dfrac 1 {2 \mu_0} \left[\dfrac{\mu_0 N I}{2 \pi R}\right]$

For capacitors: $U = \dfrac {Q^2} {2 C}, u = \dfrac 1 2 \varepsilon_0 E^2$

For inductors: $U = \dfrac 1 2 L I^2, u = \dfrac {B^2} {2 \mu_0}$

Capacitors and inductors are similar in that the both store something. Capacitors store electric field, and inductors store magnetic field.

{\Large R-L circuit} (battery then resistor then inductor):

$\varepsilon - I R - L \dfrac \dI \dt = 0 \implies \dfrac \dI \dt = \dfrac \varepsilon L - \dfrac {I R} L$

$\tau = \dfrac L R$

$\dstyle I(t) = \dfrac \varepsilon R (1 - e^{-t/\tau})$

$\dstyle \dfrac \dI \dt = \dfrac \varepsilon L e^{-t/\tau}$

\vpx{20}
The inductor makes it so, instead of changing the current instantaneously, it happens over a longer period of time.

\newpage

{\Large L-C circuit:}

A battery is not required. just assume the capacitor is already fully charged.

$V_L = -L \dfrac \dI \dt = -L \dfrac {\d^2 Q}{\dt^2}$

$L \dfrac {\d^2 Q}{\dt^2} = - \dfrac {-1} {LC} Q$

$Q(t) = Q_0 \cos(\omega t + \phi)$

$\omega = \sqrt{\dfrac 1 {LC}}$

$I(t) = - \omega Q_0 \sin(\omega t + \phi)$

usually $\phi = 0$ is used.

{\Large L-R-C circuit:}

$-I R - L \dfrac \dI \dt - \dfrac Q C = 0$

$\dfrac {\d^2 Q}{\dt^2} + \dfrac R L \dfrac \dQ \dt + \dfrac 1 {L C} Q = 0$

For $R < 2 \sqrt{\dfrac L C}$, $Q(t) = Q_0 \exp\!\left(-\frac R {2L} t\right) \cos\!\left(t \sqrt{\dfrac 1 {LC} - \dfrac{R^2}{4 L^2}}\right)$. (under damped)

For $R = 2 \sqrt{\dfrac L C}$, $Q(t) = Q_0 \exp\!\left(-\frac R {2L} t\right)$. (critically damped)

For $R > 2 \sqrt{\dfrac L C}$, $Q(t) = Q_0 \exp\!\left(-\frac R {2L} t\right) \exp\!\left(-t \sqrt{\dfrac{R^2}{4 L^2} - \dfrac 1 {L C}}\right)$. (over damped)


AC: $V(t) = V_0 \cos \omega t$. (sin also works, and may vary between textbooks).

an AC ``battery" has its own circuit diagram component picture thing. A circle with $\sim$ in it.

It usually just means plugging into a wall outlet, or signal generator or something.

60Hz is used in the U.S. 50Hz in Europe.

the Average current is 0, but that isn't really accurate, because we wouldn't be using it if AC power had
an average of 0 power. Instead, the RMS is used. $I_{\rm RMS} = \sqrt{\frac 1 {b - a} \int_a^b I^2(t) \dt}$.
\raisebox{-10px}{\acpower}.
\vpx{-5}

$(I^2)_{\rm av} = \dfrac {I_{\rm max}^2} 2$

$I_{\rm RMS} = \sqrt {(I^2)_{\rm max}} = \dfrac {I_{\rm max}} {\sqrt 2}$.

$V_{\rm RMS} = \dfrac {V_{\rm max}} {\sqrt 2}$. $V_{\rm RMS} = 170\,\rm V$ (in the U.S.)

A 110 V voltage in the U.S. means they are using rectified current.

circuit with AC power source and resistor:

$\varepsilon_0 \cos \omega t = IR \implies I(t) = \dfrac {\varepsilon_0} R \cos \omega t$

Phasor diagram?

circuit with AC power source and capacitor:

$\varepsilon_0 \cos \omega t = \dfrac Q C
\implies Q(t) = C \varepsilon_0 \cos \omega t
\implies I(t) = - \omega C \varepsilon_0 \sin (\omega t + 90^\circ)$

circuit with AC power source and inductor:

$\varepsilon_0 \cos \omega t = L \dfrac \dI \dt
\implies \dfrac \dI \dt = \dfrac {\varepsilon_0} L \cos \omega t
\implies I(t) = \dfrac {\varepsilon_0} {\omega L} \sin (\omega t - 90^\circ)$

I don't know where the $I(t)$ phase angles came from.

$\omega L = X_L = $ inductive reactance

$\dfrac 1 {\omega C} = X_C = $ capacitive reactance

\newpage

Inductors are good for low pass filters and capacitor are good for high pass filters.

Anything you can do with a DC circuit, you can, in principle, set it up for an AC circuit.

Q: can you swap the order of circuit components in series? (e.g. R\,C\,L vs. L\,R\,C)
% I think you can swap them. not sure though

$V_L = I_0 X_L = I_0 \omega L$. $90^\circ$ ahead (+) of $I$

$V_R = I_0 R$. in phase with $I$

$V_C = I_0 X_C = I_C \dfrac 1 {\omega C}$. $90^\circ$ behind (-) of $I$

$I(t) = I_0 \cos (\omega t)$

$V(t) = V_0 \cos(\omega t + \phi)$

impedence: $Z = \sqrt{R^2 + (X_L - X_C)^2}$.

this is the magnitude of the impedence, but it is actually $R + (X_L - X_C) i$

$\phi$ is called ``phase".

$X_L - X_C = R \tan \phi \implies \phi = \tan^{-1}\left(\dfrac{X_L - X_C} R\right)$

$V_0 = I_0 Z$, $V_{\rm RMS} = I_{\rm RMS} Z$.

L, R, C $\longrightarrow$ +, 0, - ~~~~ ($V$ phase offsets from $I$, in $\pi/2$)

$\vec E$ is a gradient field, but only in electrostatics. $\vec B$ is never a gradient field

Hall effect voltage: $V_G = \dfrac {B I d} {n e}$. $d$ is the thickness of the material.

For a charged particle, the magnetic force behaves like a centripetal force.

To find the radius of the circle, set $F_C = F_B$,
$\dfrac {m v^2} R = B q v \implies m v = B q R$

Two parallel wires with current in the same direction attract each other.
parallel wires with opposite current repel.

$B$ into the page is negative flux, $B$ out of the page is positive flux.

Magnetic flux is in units of Weber (Wb = ${\rm T}\,{\rm m}^2$)

increase flux $\Rightarrow$ support magnetic field

decrease flux $\Rightarrow$ go against magnetic field

you only care about the magnetic field \emph{inside} the loop.

induced current opposes the change in flux (does the opposite thing as it).

the $\varepsilon$ and $I$ have the same direction.

Power: $P(t) = V(t) I(t)$. these two are in phase with each other.

for R: $P_{\rm av} = V_{\rm RMS} I_{\rm RMS} = \dfrac {V_0} {\sqrt 2} \dfrac {I_0} {\sqrt 2} = \dfrac 1 2 V_0 I_0$.

for L and C: instantaneously, the power is not zero, but the average is zero ($P_{\rm av} = 0$)

L-R-C: $P(t) = V(t) I(t) = V_0 \cos(\omega t + \phi) + I_0 \cos \omega t
= V_0 I_0 \cos \phi \cos^2 \omega t - V_0 I_0 \sin \phi \cos(\omega t) \sin(\omega t)$

$P_{\rm av} = \dfrac 1 2 V_0 I_0 \cos \phi$.

$\cos \phi$ is called the power factor.

$I_0 = \dfrac {V_0} Z \implies P_{\rm av} = \dfrac 1 2 \dfrac {V_0^2} {\sqrt{R^2 + (X_L - X_C)^2}} \cos \phi$.

To maximize power output, set $X_L = X_C
\implies \omega L = \dfrac 1 {\omega C}
\implies \omega = \dfrac 1 {\sqrt {L C}}$.

You will have the same resonance peak ($P$ max value) if you change $R$, but a larger $R$, the peak will be less drastic.

transformers. there are step up ($N_2 > N_1$) transformers, and step down ($N_2 < N_1$) transformers.

$N_1$ is the number of inductor coils on the source side, and $N_2$ is the number of inductor coils on the other side.

transformers are from faraday's law. it only works for AC because in DC, magnetic flux doesn't change.

\newpage

the voltage is \emph{higher} in the part with more turns. if $N_2 = k N_1$, then $V_2 = k V_1$, etc.

the current is \emph{lower} in the part with more turns. if $N_2 = k N_1$, then $I_2 = \frac 1 k I_1$, etc.

It takes about 5-6 time constants for a capacitor to charge in an RC circuit.

The EMF induced by an inductor is the same as the voltage across it. So if the voltage drop is 2 V, the EMF is -2 V.

% RL circuit:
	% increasing current => positive emf
	% decreasing current => negative emf
	% this is because an inductor tries to oppose the change in current.
	% The power delivered by the battery is the same as the total power consumed by the resistor and inductor. (conservation of energy)
\end{document}
