\ifx \combinedDocuments \undefined
\documentclass[12pt]{article}

\usepackage[
	top    = 0.50in,
	left   = 1.25in,
	right  = 1.25in,
	bottom = 1.00in,
]{geometry}

\usepackage{amsmath, amssymb, latexsym, xcolor, graphicx}

\definecolor{lightgray}{RGB}{170, 170, 170}
\color{lightgray}
\pagecolor{black}

\begin{document}

\newgeometry{
	top    = 0.00in,
	left   = 1.25in,
	right  = 1.25in,
	bottom = 1.00in,
}

\title{PS 161 Final Exam Formulas}
\author{Daniel E. Janusch}
\date{December 10, 2024}
\maketitle
\fi

\begin{equation}
	Y = \dfrac{\text{stress}}{\text{strain}} = \dfrac{F/A}{\Delta L/L_0} = \dfrac{F L_0}{A\Delta L}~~[\text{Pa}]
\end{equation}

\begin{equation}
	\lambda = \dfrac mL ~~~~~~~~~~
	\sigma = \dfrac mA ~~~~~~~~~~
	\rho = \dfrac mV
\end{equation}

\begin{equation}
	p = \rho g h = \dfrac{F_1}{A_1} = \dfrac{F_1}{A_2}~~[\text{Pa}]
\end{equation}

\begin{equation}
	F_{\text{pressure}} = p A
\end{equation}

\begin{equation}
	F_{\text{buoyant}} = \rho g V
\end{equation}

\begin{equation}
	R = A v~~(\text{flow rate})
\end{equation}

\begin{equation}
	\rho_1 R_1 = \rho_2 R_2~~(\text{mass flux})
\end{equation}

\begin{equation}
	\rho_1 = \rho_2~~(\text{incompressible})
\end{equation}

\begin{equation}
	M\!E = p + \dfrac 12 \rho v^2 + \rho g y
\end{equation}

\begin{equation}
	U_g = -\dfrac{GMm} r
\end{equation}

\begin{equation}
	F_g = -\dfrac{GMm}{r^2}
\end{equation}

\begin{equation}
	v_{\text{orbit}} = \sqrt{\dfrac{GM}{R}}
\end{equation}

\begin{equation}
	v_{\text{escape}} = v_{\text{orbit}} \sqrt 2
\end{equation}

\begin{equation}
	T = \dfrac{2 \pi r}v = \dfrac{2 \pi r^{1.5}}{\sqrt{G M}}
\end{equation}

\begin{equation}
	R_S = \dfrac{2 G M}{c^2}
\end{equation}

\pagebreak
\restoregeometry

Simple Harmonic Motion (SHM):
\begin{equation}
	x(t) = A \cos(\omega t + \phi) \ni \omega^2 = \dfrac k m
\end{equation}

\begin{equation}
	\omega = 2 \pi f
\end{equation}

\begin{equation}
	E = \dfrac{m v^2 + k x^2}2
\end{equation}

\begin{equation}
	v = \pm \omega \sqrt{A^2 - x^2}
\end{equation}

\begin{equation}
	\theta(t) = \theta_0 \cos(\omega t + \phi)
\end{equation}

\begin{equation}
	\text{Angular SHM: } \omega^2 = \dfrac \kappa I \ni \kappa = \text{torsion constant}
\end{equation}

\begin{equation}
	\text{Small $\theta$ Simple Pendulum: } \omega^2 = \dfrac g L
\end{equation}

\begin{equation}
	\text{Physical Pendulum: } \omega^2 = \dfrac {mgd} I
\end{equation}

\ifx \combinedDocuments \undefined
\end{document}
\fi
