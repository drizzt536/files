\documentclass[12pt]{article}

\usepackage[
	top    = 0.5in,
	left   = 1in,
	right  = 1in,
	bottom = 1in,
]{geometry}

\usepackage{amsmath, amssymb, latexsym, xcolor}

\newcommand \dstyle \displaystyle
\newcommand \hpx [1]{\hspace{#1px}}
\newcommand \nhpx [1]{\hspace{-#1px}}
\renewcommand \d [1] {\mathrm{d}{#1}}
\newcommand \dx {{\d x}}
\newcommand \dy {{\d y}}
\newcommand \dydx {\dfrac{\dy}{\dx}}
\newcommand \ndydx [1]{\dfrac{\mathrm{d}^{#1} y}{\dx^{#1}}}
\newcommand \dd [2]{\dfrac{\d{#1}}{\d{#2}}}
\newcommand \ddn [3]{\dfrac{\mathrm{d}^{#3}{#1}}{\d{#2}^{#3}}}

\definecolor{lightgray}{RGB}{170, 170, 170}
\color{lightgray}
\pagecolor{black}

\begin{document}

\newgeometry{
	top    = 0in,
	left   = 1in,
	right  = 1in,
	bottom = 1in,
}

\title{MA 345 Exact Equations Method Justification}
\author{Daniel E. Janusch}
\maketitle

\vspace{-10px}

\begin{equation}
	M(x,y) \dx + N(x,y) \dy = 0
\end{equation}

Given an exact ODE (1), where $M_y = N_x$, the following equation holds for the solution:

\begin{equation}
	\int M \dx \uplus \int N \dy = c
\end{equation}

where $a \uplus b$ means add and ignore duplicates. Let $f(x, y)$ be the solution to the differential equation. 

$\dstyle M_1(x) + M_2(x, y) = \int M(x, y) \dx$, where $M_1(x)$ has all the terms with only $x$, and $M_2(x, y)$ has all the terms that can't be separated further. $\dstyle N_1(y) + N_2(x, y) = \int N(x, y) \dy$, where $N_1(y)$ has all the terms with only $y$, and $N_2(x, y)$ has all the terms that can't be separated further.

\begin{equation}
	f(x, y) = \int M(x, y) \dx + g(y) = M_1(x) + M_2(x, y) + g(y)
\end{equation}

\begin{equation}
	f(x, y) = \int N(x, y) \dy + h(x) = N_1(y) + N_2(x, y) + h(x)
\end{equation}

Given (3) and (4), by the transitive property, $M_1(x) + M_2(x, y) + g(y) = N_1(y) + N_2(x, y) + h(x)$. Each side has a term of just $x$, a term of just $y$, and a term with both. Separating gives the following three equations:

\begin{equation}
	M_1(x) = h(x)
\end{equation}

\begin{equation}
	M_2(x, y) = N_2(x, y)
\end{equation}

\begin{equation}
	g(y) = N_1(y)
\end{equation}

Equation (3) can be rearranged into $\dstyle \int N(x, y) \dy - N_2(x, y) = N_1(y) = g(y)$.\\
Substituting this into Equation (2) gives:

\begin{equation}
	f(x, y) = \int M(x, y) \dx + \left[\int N(x, y) \dy - N_2(x, y)\right]
\end{equation}

This is then equivalent to the following by equations (1) and (6):

\begin{equation}
	f(x, y) = [M_1(x) + N_2(x, y)] - N_2(x, y) + \int N(x, y) \dy
\end{equation}

\newpage
\restoregeometry

But this is just the sum of the integrals with the duplicates ($M_2$ and $N_2$) subtracted off.

Thus, the original equation holds (also because the integral of $0$ is a constant).

\vspace{20px}

NOTE: stopping after equation 7 is sufficient proof for on exams, so long as the functions are defined concretely.

\end{document}
