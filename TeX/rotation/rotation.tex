% pdflatex %
\documentclass[12pt]{article}
\usepackage{amssymb,latexsym,amsmath}
\usepackage[margin=1in,footskip=0.5in]{geometry}
\begin{document}
\addtolength{\topmargin}{-1in}
\title{Point Rotation Formula Proof}
\author{Daniel E. Janusch}
\maketitle
\noindent preface: $\tan^{-1}(y,x)$\\

\noindent pf:\\

\noindent goal: rotate $(a,b)~\theta$ units around the point $(h,k)$\\
$A_1:=(a,b)$\\
$B_1:=(a,k)$\\
$C_1:=(h,k)$\\
\\
\text{shift everything by $(-h,-k)$}\\
$A_2:=(a-h,b-k)$\\
$B_2:=(a-h,0)$\\
$C_2:=(0,0)$\\
\\
$m\angle A_1B_1C_1=m\angle A_2B_2C_2$\\
\\
$t:=\tan^{-1}\left(b-k,a-h\right)$~~~~(the angle that $(a,b)$ would have if $(h,k)$ was the origin)\\
\\
$\theta_2:=t+\theta$ (the new angle)\\$(\cos\theta_2,\sin\theta_2)$ has the right angle with the distance scaled to 1\\
\\
$d:=\sqrt{\left(b-k\right)^2+\left(a-h\right)^2}$ (the distance between $(a,b)$ and $(h,k)$)\\
\\
the point would then be $(d\cos\theta_2,d\sin\theta_2)$\\
\\
after shifting by $(h,k)$ to counteract the shift from before it becomes $(h+d\cos\theta_2,k+d\sin\theta_2)$\\$=\left(\begin{aligned}h+\cos\left(\tan^{-1}\left(b-k,a-h\right)+\theta\right)\sqrt{\left(b-k\right)^2+\left(a-h\right)^2}~,\\
k+\sin\left(\tan^{-1}\left(b-k,a-h\right)+\theta\right)\sqrt{\left(b-k\right)^2+\left(a-h\right)^2}\end{aligned}\right)$\\
\\
\blacksquare
\end{document}
