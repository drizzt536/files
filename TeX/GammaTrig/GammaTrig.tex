% file for pdflatex.exe

\documentclass[12pt]{article}
\usepackage{amssymb,latexsym,amsmath}

\begin{document}

\title{Trigonometric Functions in Terms of Gamma}
\author{Daniel E. Janusch}
\maketitle

\begin{section}{Definitions}

	Inverse trig functions don't count, but hyperbolic trig functions do.\\
	``$\ni$" means ``such that".\\
	``$\land$" means ``and".\\
	\noindent$0\in\mathbb N_0$

	\begin{equation}
		\mathbb N_k:=\left\{x+k:x\in \mathbb N_0\right\}
	\end{equation}

	\begin{equation}
		\tau:=2\pi
	\end{equation}

	\begin{equation}
		\Gamma(x):=\int_0^\infty t^xe^{-t}dt=(x-1)!~\forall x>-1
	\end{equation}

	\begin{equation}
		\zeta(x):=\sum_{n=0}^\infty\dfrac1{n^x}\forall x>1
	\end{equation}
\end{section}

\pagebreak
\begin{section}{Riemann Zeta Function}
	claim: $\zeta(x)=\dfrac{\zeta(1-x)\tau^x\sec\left(\dfrac{\pi x}2\right)}{2\,\Gamma(x)}$\\
	Proof:\\
	\noindent know: \zeta(x)=2^x\pi^{x-1}\sin\left(\dfrac{\pi x}2\right)\Gamma(1-x)\zeta(1-x)\\

	\begin{equation}
		\zeta(x)=2^x\pi^{x-1}\sin\left(\dfrac{\pi x}2\right)\Gamma\left(1-x\right)\zeta\left(1-x\right)
	\end{equation}

	\begin{equation}
		\zeta(x)=2\cdot2^{x-1}\pi^{x-1}\sin\left(\dfrac{\pi x}2\right)\Gamma(1-x)\zeta(1-x)
	\end{equation}

	\begin{equation}
		\zeta(x)=2\cdot\tau^{x-1}\sin\left(\dfrac{\pi x}2\right)\Gamma(1-x)\zeta(1-x)
	\end{equation}

	\begin{equation}
		\zeta(x)=2\cdot\tau^{x-1}\cos\left(\dfrac\pi2(1-x)\right)\Gamma(1-x)\zeta(1-x)
	\end{equation}

	\begin{equation}
		\zeta(x)=\dfrac2{\tau^{1-x}}\cos\left(\dfrac\pi2(1-x)\right)\Gamma(1-x)\zeta(1-x)
	\end{equation}

	\begin{equation}
		\zeta(1-x)=\dfrac2{\tau^x}\cos\left(\dfrac{\pi x}2\right)\Gamma(x)\zeta(x)
	\end{equation}

	\begin{equation}
		\dfrac{\zeta(1-x)\tau^x}{2\cos\left(\dfrac{\pi x}2\right)\Gamma(x)}=\zeta(x)
	\end{equation}

	\begin{equation}
		\zeta(x)=\dfrac{\zeta(1-x)\tau^x\sec\left(\dfrac{\pi x}2\right)}{2\,\Gamma(x)}
	\end{equation}

	\blacksquare
\end{section}

\pagebreak
\begin{section}{Cosecant}
	claim: $\csc x=\dfrac1\pi\Gamma\left(\dfrac x\pi\right)\Gamma\left(1-\dfrac x\pi\right)$\\
	Proof:\\
	\noindent know: \zeta(x)=2^x\pi^{x-1}\sin\left(\dfrac{\pi x}2\right)\Gamma(1-x)\zeta(1-x)\\

	\noindent know: \zeta(x)=\dfrac{\zeta(1-x)\tau^x\sec\left(\dfrac{\pi x}2\right)}{2\,\Gamma(x)}\\

	\begin{equation}
		2^x\pi^{x-1}\sin\left(\dfrac{\pi x}2\right)\Gamma(1-x)\zeta(1-x)=\dfrac{\zeta(1-x)\tau^x\sec\left(\dfrac{\pi x}2\right)}{2\,\Gamma(x)}
	\end{equation}

	\begin{equation}
		2^x\pi^{x-1}\sin\left(\dfrac{\pi x}2\right)\cos\left(\dfrac{\pi x}2\right)\Gamma(1-x)=\dfrac{\tau^x}{2\,\Gamma(x)}
	\end{equation}

	\begin{equation}
		4\cdot\tau^{x-1}\sin\left(\dfrac{\pi x}2\right)\cos\left(\dfrac{\pi x}2\right)\Gamma(1-x)=\dfrac{\tau^x}{\Gamma(x)}
	\end{equation}

	\begin{equation}
		2\sin\left(\dfrac{\pi x}2\right)\cos\left(\dfrac{\pi x}2\right)\Gamma(1-x)=\dfrac{\pi}{\Gamma(x)}
	\end{equation}

	\begin{equation}
		\sin(\pi x)\Gamma(1-x)\Gamma(x)=\pi
	\end{equation}

	\begin{equation}
		\Gamma(1-x)\Gamma(x)=\pi\csc(\pi x)
	\end{equation}

	\begin{equation}
		\csc x=\dfrac1\pi\Gamma\left(\dfrac x\pi\right)\Gamma\left(1-\dfrac x\pi\right)\ni\zeta(1-x)\ne0\land\sin\pi x\ne0
	\end{equation}

	\indent$x\ne2n+1\ni n\in\mathbb N_1\land x\ne k\ni k\in\mathbb Z\Longrightarrow x\ne\mathbb Z$~~\text{(same domain as $\csc\pi x$)}\\

	\blacksquare
\end{section}

\pagebreak
\begin{section}{The remaining Functions}
	\begin{equation}\sin x=\dfrac1{\csc x}\end{equation}

	\begin{equation}\cos x=\dfrac1{\csc\left(\dfrac\pi2-x\right)}\end{equation}

	\begin{equation}\tan x=\dfrac{\csc\left(\dfrac\pi2-x\right)}{\csc x}\end{equation}

	\begin{equation}\csc x=\csc x\end{equation}
	
	\begin{equation}\sec x=\csc\left(\dfrac\pi2-x\right)\end{equation}

	\begin{equation}\cot x=\dfrac{\csc x}{\csc\left(\dfrac\pi2-x\right)}\end{equation}


	\begin{equation}\sinh x=-\dfrac i{\csc x}\end{equation}

	\begin{equation}\cosh x=\dfrac1{\csc\left(\dfrac\pi2-ix\right)}\end{equation}

	\begin{equation}\tanh x=-i\dfrac{\csc\left(\dfrac\pi2-ix\right)}{\csc ix}\end{equation}

	\begin{equation}\text{csch}\,x=i\csc ix\end{equation}

	\begin{equation}\text{sech}\,x=\csc\left(\dfrac\pi2-ix\right)\end{equation}

	\begin{equation}\coth x=\dfrac{i\csc ix}{\csc\left(\dfrac\pi2-ix\right)}\end{equation}
\end{section}


\end{document}
