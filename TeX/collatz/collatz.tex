% file for pdflatex.exe

\documentclass[12pt]{article}
\usepackage{amssymb, latexsym, amsmath, amsthm}
\usepackage[margin=1in]{geometry}

\renewcommand{\qedsymbol}{\blacksquare}

\begin{document}


\title{Collatz Conjecture Proof (Work in Progress)}
\author{Daniel E. Janusch}
\date{January 16, 2023} % 11:21pm MST
\maketitle

\begin{section}{Preface}\label{sec:preface}
	\begin{align}
		& 0\in\mathbb N\\
		a\,\text{mod}_k\,b & :=a-b\left\lfloor\frac ab\right\rfloor+k\\
		a\,\text{mod}\,b & := a~\text{mod}_0~b
	\end{align}
\end{section}

\begin{section}{Proof}
	\begin{proof}
		let $n,m,k,p~\in\mathbb N,~f(x)=\begin{cases}
			3x+1,~x\bmod2\equiv1\\
			\frac x2,~x\bmod2\equiv0\\
			\text{undef, otherwise}~~(x\notin\mathbb Z)
		\end{cases}$\\
		$\log_2x\in\mathbb N\Longleftrightarrow1$ is reached after $\log_2x$ iterations.\\
		$3^pn\ne2^k\forall n\ne2^m$. Therefore multiplying by 3 has no effect on the ability to reach $2^n$
		and neither does dividing by 2.
		$3n+1\stackrel?=2^k\land3n+2\stackrel?=2^k$~\therefore~\text{adding does have an effect.}\\
		$\left|\mathbb N\right|=\infty=\left|2^\mathbb N\right|$ (there are infinite $2^n$s).\\\
		One of the $2^n$s will always be reached because iterating through $f$ will add 1 every time it is odd.
	\end{proof}
\end{section}

\begin{section}{Footnotes}\label{sec:footnotes}

	This will only prove for $x\in\mathbb N$.\\
	The formulas for $f$ or mod have to change for $x\in\mathbb I,\mathbb C$\\
	~~~~ie: $3i\bmod2=i\notin\{1,0\}$.\\
	The conjecture does not hold $\forall x\in-\mathbb N$ because $x=-5$ loops.
\end{section}


\end{document}
