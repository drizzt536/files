% file for pdflatex.exe

\documentclass[12pt]{article}
\usepackage{amssymb,latexsym,amsmath}

\addtolength{\evensidemargin}{-1in}
\addtolength{\oddsidemargin}{0in}
\addtolength{\topmargin}{-1.75in}

\begin{document}
\title{Collatz Conjecture Proof (Work in Progress)}\author{Daniel E. Janusch}\maketitle

\noindent preface:\\
\indent0~\in\mathbb N\\
\indent$$a~\text{mod}_k~b\equiv a-b\left\lfloor\frac ab\right\rfloor+k$$\\
\indent a\bmod b\equiv a~\text{mod}_0~b$$

\noindent pf:\\
\indent let $n,m,k,p$ \in\mathbb N,~~f(x)=\begin{cases}
	3x+1,~x\bmod2\equiv1\\
	\frac x2,~x\bmod2\equiv0\\
	\text{undef, otherwise}~~(x\notin\mathbb Z)
\end{cases}\\
\indent3^p\ne2^k\foralln\ne2^m

therefore multiplying by 3 has no effect on the ability to reach $2^n$.

neither does dividing by 2.\\
\indent$3n+1\stackrel?=2^k$~\therefore~\text{adding does have an effect.}\\
\indent\log_2x\in\mathbb N\Longleftrightarrow1\text{ is reached after $\log_2x$ iterations.}\\
\indent\left|\mathbb N\right|=\infty\Longleftrightarrow\left|2^\mathbb N\right|=\infty~~(\text{there are infinite $2^n$s.})

one of the $2^n$s will always be reached because iterating through $f$

will add 1 every time it is odd.
\\\indent\blacksquare?

\noindent footnotes:

will only prove for $x\in\mathbb N$.

the formulas for $f$ or mod have to change for $x\in\mathbb I,\mathbb C$

\indent\indent ie: $3i\bmod2=i\notin\{1,0\}$.

the conjecture does not hold $\forall x\in-\mathbb N$ because $x=-5$ loops.

\end{document}
