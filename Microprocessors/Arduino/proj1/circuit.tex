\documentclass{standalone}

\usepackage{circuitikz}
\usepackage[dvipsnames]{xcolor}

\color[RGB]{170, 170, 170}
\pagecolor{black}

\begin{document}

\begin{circuitikz}
	\node at ( 0, 4) [circ, label=above:{GPIO 48}] {};
	\node at ( 2, 4) [circ, label=above:{GPIO 46}] {};
	\node at ( 4, 4) [circ, label=above:{GPIO 44}] {};
	\node at ( 6, 4) [circ, label=above:{GPIO 42}] {};
	\node at ( 8, 4) [circ, label=above:{GPIO 40}] {};
	\node at (10, 4) [circ, label=above:{GPIO 38}] {};
	\node at (12, 4) [circ, label=above:{GPIO 36}] {};
	\node at (14, 4) [circ, label=above:{GPIO 34}] {};
	\node at (16, 4) [circ, label=above:{GPIO 32}] {};
	\node at (18, 4) [circ, label=above:{GPIO 30}] {};
	\node at (20, 4) [circ, label=above:{GPIO 2}] {};
	\node at (22, 4) [circ, label=above:GND] {};

	\draw ( 0, 4) to[led, color=white]     ( 0, 2) to[R, l=$330\,\Omega$] ( 0, 0);
	\draw ( 2, 4) to[led, color=RoyalBlue] ( 2, 2) to[R, l=$330\,\Omega$] ( 2, 0);
	\draw ( 4, 4) to[led, color=RoyalBlue] ( 4, 2) to[R, l=$330\,\Omega$] ( 4, 0);
	\draw ( 6, 4) to[led, color=RoyalBlue] ( 6, 2) to[R, l=$330\,\Omega$] ( 6, 0);
	\draw ( 8, 4) to[led, color=RoyalBlue] ( 8, 2) to[R, l=$330\,\Omega$] ( 8, 0);
	\draw (10, 4) to[led, color=RoyalBlue] (10, 2) to[R, l=$330\,\Omega$] (10, 0);
	\draw (12, 4) to[led, color=RoyalBlue] (12, 2) to[R, l=$330\,\Omega$] (12, 0);
	\draw (14, 4) to[led, color=RoyalBlue] (14, 2) to[R, l=$330\,\Omega$] (14, 0);
	\draw (16, 4) to[led, color=RoyalBlue] (16, 2) to[R, l=$330\,\Omega$] (16, 0);
	\draw (18, 4) to[led, color=RoyalBlue] (18, 2) to[R, l=$330\,\Omega$] (18, 0);

	\draw (20, 4) -- (20, 2) to[push button, fill=black] (20, 0);

	\draw (0, 0) -- (20, 0) to[R, l=$2\,{\rm k}\Omega$] (22, 0) -- (22, 4);
	\node [ground] at (22, 0) {};

	\node [right, align=left] at (-0.25, -1.25) {
		Equivalent resistance is $2033\,\Omega$. the $2\,{\rm k}\Omega$ resistor is actualy two $1\,{\rm k}\Omega$ resistors in series. \\
		Intended for Ardino Mega 2560. Pins are all digital, but are arbitrary otherwise.\\
		Only two out of the four terminals on the push button are used.\\
		The first LED is white because the other blue LED burned out.
	};

	\node at (-0.75, -2.25) {};
	\node at (22.75, 4.75) {};

\end{circuitikz}

\end{document}
